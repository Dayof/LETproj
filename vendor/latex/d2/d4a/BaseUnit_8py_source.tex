\hypertarget{BaseUnit_8py}{\subsection{Base\-Unit.\-py}
\label{BaseUnit_8py}\index{E\-L\-O/\-E\-L\-O/\-Base\-Unit.\-py@{E\-L\-O/\-E\-L\-O/\-Base\-Unit.\-py}}
}

\begin{DoxyCode}
\hypertarget{BaseUnit_8py_source_l00001}{}\hyperlink{namespaceELO_1_1BaseUnit}{00001} \textcolor{comment}{#coding: utf-8}
00002 
00003 \textcolor{comment}{## @file BaseUnit.py}
00004 \textcolor{comment}{#  Implementa os tipos básicos do programa.}
00005 \textcolor{comment}{#}
00006 \textcolor{comment}{#   Os tipos básicos são os blocos formadores do programa. Eles contêm}
00007 \textcolor{comment}{#   informações atômicas como Nome, Senha ou Matrícula, e devem ser capazes}
00008 \textcolor{comment}{#   de validá-los no momento de sua criação. Ou seja, caso o valor fornecido}
00009 \textcolor{comment}{#   não esteja de acordo com as especificações, o tipo básico não é criado}
00010 \textcolor{comment}{#   e lança uma exceção do tipo ValueError.}
00011 
00012 \textcolor{keyword}{from} abc \textcolor{keyword}{import} *
00013 \textcolor{keyword}{import} hashlib
00014 
00015 \textcolor{keyword}{import} \hyperlink{namespaceELO_1_1locale_1_1index}{ELO.locale.index} \textcolor{keyword}{as} lang
00016 
00017 \textcolor{comment}{## Interface para todos os tipos básicos.}
00018 \textcolor{comment}{#}
00019 \textcolor{comment}{#   Classe abstrata que contém os métodos get, set e del de todos os tipos}
00020 \textcolor{comment}{#   básicos. Estes métodos são chamados implicitamente através dos processos}
00021 \textcolor{comment}{#   de requisição de valor, atribuição de valor, e deleção do objeto,}
00022 \textcolor{comment}{#   respectivamente.}
00023 \textcolor{comment}{#}
00024 \textcolor{comment}{#   Possui um único método abstrato, o validador do tipo básico, que varia}
00025 \textcolor{comment}{#   de tipo para tipo, e deve ser obrigatoriamente implementado, sob a pena}
00026 \textcolor{comment}{#   do tipo básico não poder ser instanciado.}
\hypertarget{BaseUnit_8py_source_l00027}{}\hyperlink{classELO_1_1BaseUnit_1_1IfBaseType}{00027} \textcolor{keyword}{class }\hyperlink{classELO_1_1BaseUnit_1_1IfBaseType}{IfBaseType}:
00028 
00029     \textcolor{comment}{## Especifica que IfBaseType é uma classe base abstrata (abc). }
00030     \textcolor{comment}{#   Isso significa que uma classe derivada pode ser instanciada se}
00031     \textcolor{comment}{#   - e somente se - ela der "override" em todos os métodos e}
00032     \textcolor{comment}{#   propriedades abstratas.}
00033     \_\_metaclass\_\_ = ABCMeta
00034 
00035     \textcolor{comment}{## Método que retorna o valor contido no tipo básico.}
00036     \textcolor{comment}{#   Método GET.}
00037     @property
\hypertarget{BaseUnit_8py_source_l00038}{}\hyperlink{classELO_1_1BaseUnit_1_1IfBaseType_ae4974528321f9314afd17c3da0e9d676}{00038}     \textcolor{keyword}{def }\hyperlink{classELO_1_1BaseUnit_1_1IfBaseType_ae4974528321f9314afd17c3da0e9d676}{value}(self):
00039         \textcolor{keywordflow}{return} self.\hyperlink{classELO_1_1BaseUnit_1_1IfBaseType_ad05d9d377fc4b99743c022cc8f6019d7}{\_value}
00040 
00041     \textcolor{comment}{## Método que fixa e valida o conteúdo do tipo básico.}
00042     \textcolor{comment}{#   Método SET.}
00043     \textcolor{comment}{#}
00044     \textcolor{comment}{#   @arg aux Recebe o valor que será validado.}
00045     @value.setter
\hypertarget{BaseUnit_8py_source_l00046}{}\hyperlink{classELO_1_1BaseUnit_1_1IfBaseType_ae4974528321f9314afd17c3da0e9d676}{00046}     \textcolor{keyword}{def }\hyperlink{classELO_1_1BaseUnit_1_1IfBaseType_ae4974528321f9314afd17c3da0e9d676}{value}(self, aux):
00047         self.\hyperlink{classELO_1_1BaseUnit_1_1IfBaseType_a11c68b128a7069e27c1c2fcb782269ea}{\_validate}(aux)
00048         self.\hyperlink{classELO_1_1BaseUnit_1_1IfBaseType_ad05d9d377fc4b99743c022cc8f6019d7}{\_value} = aux
00049 
00050     \textcolor{comment}{## Método que deleta o conteúdo do tipo básico.}
00051     \textcolor{comment}{#   Método DEL.}
00052     @value.deleter
\hypertarget{BaseUnit_8py_source_l00053}{}\hyperlink{classELO_1_1BaseUnit_1_1IfBaseType_ae4974528321f9314afd17c3da0e9d676}{00053}     \textcolor{keyword}{def }\hyperlink{classELO_1_1BaseUnit_1_1IfBaseType_ae4974528321f9314afd17c3da0e9d676}{value}(self):
00054         del self.\hyperlink{classELO_1_1BaseUnit_1_1IfBaseType_ad05d9d377fc4b99743c022cc8f6019d7}{\_value}
00055 
00056     \textcolor{comment}{## Define que toda classe derivada terá um método \_validate.}
00057     @abstractmethod
00058     \textcolor{keyword}{def }\_validate(self, value): \textcolor{keyword}{pass}
00059 
00060     \textcolor{comment}{## Método que compara o conteúdo de dois tipos básicos.}
00061     \textcolor{comment}{#   Será chamado toda vez que dois tipos básicos forem}
00062     \textcolor{comment}{#   comparados através dos operadores >, <, >=, <=, == ou !=.}
00063     \textcolor{comment}{#}
00064     \textcolor{comment}{#   @return Retornos predefinidos do próprio python.}
\hypertarget{BaseUnit_8py_source_l00065}{}\hyperlink{classELO_1_1BaseUnit_1_1IfBaseType_a69c338f6f1574bd3524e9d59ebc17a7c}{00065}     \textcolor{keyword}{def }\hyperlink{classELO_1_1BaseUnit_1_1IfBaseType_a69c338f6f1574bd3524e9d59ebc17a7c}{\_\_cmp\_\_}(self, other):
00066         \textcolor{keywordflow}{if} other.value == self.\hyperlink{classELO_1_1BaseUnit_1_1IfBaseType_ae4974528321f9314afd17c3da0e9d676}{value}:
00067             \textcolor{keywordflow}{return} 0
00068         \textcolor{keywordflow}{elif} other.value > self.\hyperlink{classELO_1_1BaseUnit_1_1IfBaseType_ae4974528321f9314afd17c3da0e9d676}{value}:
00069             \textcolor{keywordflow}{return} 1
00070         \textcolor{keywordflow}{else}:
00071             \textcolor{keywordflow}{return} -1
00072 
00073 \textcolor{comment}{## Classe container de senhas.}
00074 \textcolor{comment}{#   Deve ser responsável não somente por armazenar, mas também por}
00075 \textcolor{comment}{#   encriptar a string recebida.}
\hypertarget{BaseUnit_8py_source_l00076}{}\hyperlink{classELO_1_1BaseUnit_1_1Password}{00076} \textcolor{keyword}{class }\hyperlink{classELO_1_1BaseUnit_1_1Password}{Password}(\hyperlink{classELO_1_1BaseUnit_1_1IfBaseType}{IfBaseType}):
00077     \_value = \textcolor{keywordtype}{None}
00078 
00079     \textcolor{comment}{## Construtor da classe.}
00080     \textcolor{comment}{#   Valida e encripta a string recebida.}
00081     \textcolor{comment}{#}
00082     \textcolor{comment}{#   @arg        value       String a ser armazenada.}
00083     \textcolor{comment}{#}
00084     \textcolor{comment}{#   @exception  ValueError  Exceção lançada no caso do}
00085     \textcolor{comment}{#                           valor de entrada não passar pelos}
00086     \textcolor{comment}{#                           critérios do método \_validate().}
\hypertarget{BaseUnit_8py_source_l00087}{}\hyperlink{classELO_1_1BaseUnit_1_1Password_a01568369a90b21117ba3beb05cbf9dfe}{00087}     \textcolor{keyword}{def }\hyperlink{classELO_1_1BaseUnit_1_1Password_a01568369a90b21117ba3beb05cbf9dfe}{\_\_init\_\_}(self, value):
00088         \textcolor{keywordflow}{try}:
00089             self.\hyperlink{classELO_1_1BaseUnit_1_1IfBaseType_a11c68b128a7069e27c1c2fcb782269ea}{\_validate}(value)
00090         \textcolor{keywordflow}{except} ValueError \textcolor{keyword}{as} exc:
00091             del self
00092             \textcolor{keywordflow}{raise} exc
00093         self.\hyperlink{classELO_1_1BaseUnit_1_1IfBaseType_ad05d9d377fc4b99743c022cc8f6019d7}{\_value} = hashlib.sha256(value).hexdigest()
00094         self.\hyperlink{classELO_1_1BaseUnit_1_1IfBaseType_ad05d9d377fc4b99743c022cc8f6019d7}{\_value} = hashlib.md5(self.\hyperlink{classELO_1_1BaseUnit_1_1IfBaseType_ad05d9d377fc4b99743c022cc8f6019d7}{\_value}).hexdigest()
00095 
00096     \textcolor{comment}{## Validator da classe.}
00097     \textcolor{comment}{#   Verifica se o tamanho da string recebida (value) está correta.}
00098     \textcolor{comment}{#}
00099     \textcolor{comment}{#   @arg        value       String a ser validada.}
00100     \textcolor{comment}{#}
00101     \textcolor{comment}{#   @exception  ValueError  Lançada no caso do tamanho de}
00102     \textcolor{comment}{#                           value ser menor do que 6}
00103     \textcolor{comment}{#                           caracteres.}
00104     \textcolor{keyword}{def }\_validate(self, value):
00105         if(len(value) < 6):
00106             \textcolor{keywordflow}{raise} ValueError(lang.DICT[\textcolor{stringliteral}{'EXCEPTION\_INV\_PW\_S'}])
00107 
00108     @property
\hypertarget{BaseUnit_8py_source_l00109}{}\hyperlink{classELO_1_1BaseUnit_1_1Password_a42dcd63788a2eed3780c39f368356546}{00109}     \textcolor{keyword}{def }\hyperlink{classELO_1_1BaseUnit_1_1Password_a42dcd63788a2eed3780c39f368356546}{value}(self):
00110         \textcolor{keywordflow}{return} self.\hyperlink{classELO_1_1BaseUnit_1_1IfBaseType_ad05d9d377fc4b99743c022cc8f6019d7}{\_value}
00111 
00112     \textcolor{comment}{## Método setter específico de password.}
00113     \textcolor{comment}{#   Sobrescreve o setter definido na interface para que ocorra}
00114     \textcolor{comment}{#   a encriptação de forma correta.}
00115     \textcolor{comment}{#}
00116     \textcolor{comment}{#   @arg value String a ser validada e encriptada.}
00117     @value.setter
\hypertarget{BaseUnit_8py_source_l00118}{}\hyperlink{classELO_1_1BaseUnit_1_1Password_a42dcd63788a2eed3780c39f368356546}{00118}     \textcolor{keyword}{def }\hyperlink{classELO_1_1BaseUnit_1_1Password_a42dcd63788a2eed3780c39f368356546}{value}(self, value):
00119         self.\hyperlink{classELO_1_1BaseUnit_1_1IfBaseType_a11c68b128a7069e27c1c2fcb782269ea}{\_validate}(value)
00120         self.\hyperlink{classELO_1_1BaseUnit_1_1IfBaseType_ad05d9d377fc4b99743c022cc8f6019d7}{\_value} = hashlib.sha256(value).hexdigest()
00121         self.\hyperlink{classELO_1_1BaseUnit_1_1IfBaseType_ad05d9d377fc4b99743c022cc8f6019d7}{\_value} = hashlib.md5(self.\hyperlink{classELO_1_1BaseUnit_1_1IfBaseType_ad05d9d377fc4b99743c022cc8f6019d7}{\_value}).hexdigest()
00122 
00123 \textcolor{comment}{## Classe container de nomes.}
00124 \textcolor{comment}{#   Responsável por armazenar uma string que servirá para identificação}
00125 \textcolor{comment}{#   dos usuários do sistema.}
\hypertarget{BaseUnit_8py_source_l00126}{}\hyperlink{classELO_1_1BaseUnit_1_1Name}{00126} \textcolor{keyword}{class }\hyperlink{classELO_1_1BaseUnit_1_1Name}{Name}(\hyperlink{classELO_1_1BaseUnit_1_1IfBaseType}{IfBaseType}):
00127 
00128     \_value = \textcolor{keywordtype}{None}
00129 
00130     \textcolor{comment}{## Construtor da classe.}
00131     \textcolor{comment}{#   Responsável por impedir a criação do tipo básico em caso de}
00132     \textcolor{comment}{#   falha.}
00133     \textcolor{comment}{#}
00134     \textcolor{comment}{#   @arg value String a ser armazenada.}
\hypertarget{BaseUnit_8py_source_l00135}{}\hyperlink{classELO_1_1BaseUnit_1_1Name_ad73fdab6426635b79b4447ac53c10279}{00135}     \textcolor{keyword}{def }\hyperlink{classELO_1_1BaseUnit_1_1Name_ad73fdab6426635b79b4447ac53c10279}{\_\_init\_\_}(self, value):
00136         \textcolor{keywordflow}{try}:
00137             self.\hyperlink{classELO_1_1BaseUnit_1_1IfBaseType_a11c68b128a7069e27c1c2fcb782269ea}{\_validate}(value)
00138         \textcolor{keywordflow}{except} ValueError \textcolor{keyword}{as} exc:
00139             del self
00140             \textcolor{keywordflow}{raise} exc
00141         self.\hyperlink{classELO_1_1BaseUnit_1_1IfBaseType_ad05d9d377fc4b99743c022cc8f6019d7}{\_value} = value
00142 
00143     \textcolor{comment}{## Validator da classe.}
00144     \textcolor{comment}{#   Verifica se a string de entrada (value) está de acordo com as}
00145     \textcolor{comment}{#   especificações.}
00146     \textcolor{comment}{#}
00147     \textcolor{comment}{#   @arg       value        String a ser validada.}
00148     \textcolor{comment}{#}
00149     \textcolor{comment}{#   @exception ValueError   Exceção lançada no caso da string de}
00150     \textcolor{comment}{#                           entrada ser vazia ou conter algum}
00151     \textcolor{comment}{#                           caractere não-alfanumérico.}
00152     \textcolor{keyword}{def }\_validate(self, value):
00153         \textcolor{keywordflow}{if} len(value) > 32:
00154             \textcolor{keywordflow}{raise} ValueError(lang.DICT[\textcolor{stringliteral}{'EXCEPTION\_INV\_NM\_B'}])
00155         \textcolor{keywordflow}{else}: 
00156             \textcolor{keywordflow}{if} len(value) == 0:
00157                 \textcolor{keywordflow}{raise} ValueError(lang.DICT[\textcolor{stringliteral}{'EXCEPTION\_INV\_NM\_S'}])
00158             \textcolor{keywordflow}{else}:
00159                 temp = unicode.isalnum(value.replace(\textcolor{stringliteral}{" "}, \textcolor{stringliteral}{""}))
00160                 \textcolor{keywordflow}{if} temp == \textcolor{keyword}{False}:
00161                     \textcolor{keywordflow}{raise} ValueError(lang.DICT[\textcolor{stringliteral}{'EXCEPTION\_INV\_NM\_F'}])
00162 
00163 \textcolor{comment}{## Classe container de matrículas.}
00164 \textcolor{comment}{#   Responsável por armazenar um número inteiro que identifica}
00165 \textcolor{comment}{#   univocamente alguns usuários do sistema.}
\hypertarget{BaseUnit_8py_source_l00166}{}\hyperlink{classELO_1_1BaseUnit_1_1Matric}{00166} \textcolor{keyword}{class }\hyperlink{classELO_1_1BaseUnit_1_1Matric}{Matric}(\hyperlink{classELO_1_1BaseUnit_1_1IfBaseType}{IfBaseType}):
00167 
00168     \_value = \textcolor{keywordtype}{None}
00169 
00170     \textcolor{comment}{## Construtor da classe.}
00171     \textcolor{comment}{#   Responsável por impedir a criação de um tipo básico com}
00172     \textcolor{comment}{#   valor inválido.}
00173     \textcolor{comment}{#}
00174     \textcolor{comment}{#   @arg value Número inteiro a ser armazenado no tipo básico.}
\hypertarget{BaseUnit_8py_source_l00175}{}\hyperlink{classELO_1_1BaseUnit_1_1Matric_a050299f6e8d99a530a4598270b93b8d9}{00175}     \textcolor{keyword}{def }\hyperlink{classELO_1_1BaseUnit_1_1Matric_a050299f6e8d99a530a4598270b93b8d9}{\_\_init\_\_}(self, value):
00176         \textcolor{keywordflow}{try}:
00177             self.\hyperlink{classELO_1_1BaseUnit_1_1IfBaseType_a11c68b128a7069e27c1c2fcb782269ea}{\_validate}(value)
00178         \textcolor{keywordflow}{except} ValueError \textcolor{keyword}{as} exc:
00179             del self
00180             \textcolor{keywordflow}{raise} exc
00181         self.\hyperlink{classELO_1_1BaseUnit_1_1IfBaseType_ad05d9d377fc4b99743c022cc8f6019d7}{\_value} = value
00182 
00183     \textcolor{comment}{## Validator da classe.}
00184     \textcolor{comment}{#   Responsável por garantir a coerência da matrícula armazenada}
00185     \textcolor{comment}{#   com as especificações do sistema.}
00186     \textcolor{comment}{#}
00187     \textcolor{comment}{#   @arg       value    Inteiro a ser validado.}
00188     \textcolor{comment}{#}
00189     \textcolor{comment}{#   @exception ValueError   Exceção lançada no caso de value}
00190     \textcolor{comment}{#                           exceder a faixa dinâmica permitida}
00191     \textcolor{comment}{#                           para a matrícula (1-9 dígitos).}
00192     \textcolor{keyword}{def }\_validate(self, value):
00193         \textcolor{keywordflow}{try}: int(value)
00194         \textcolor{keywordflow}{except} ValueError: \textcolor{keywordflow}{raise} ValueError(lang.DICT[\textcolor{stringliteral}{'EXCEPTION\_INT\_MT\_F'}])
00195 
00196         \textcolor{keywordflow}{if} value > 999999999:
00197             \textcolor{keywordflow}{raise} ValueError(lang.DICT[\textcolor{stringliteral}{'EXCEPTION\_INV\_MT\_B'}])
00198         \textcolor{keywordflow}{elif} value < 1:
00199             \textcolor{keywordflow}{raise} ValueError(lang.DICT[\textcolor{stringliteral}{'EXCEPTION\_INV\_MT\_S'}])
00200         
00201 \textcolor{comment}{## Classe container de texto.}
00202 \textcolor{comment}{#   Responsável por armazenar texto corrido, como em descrições,}
00203 \textcolor{comment}{#   biografias, resumos etc.}
\hypertarget{BaseUnit_8py_source_l00204}{}\hyperlink{classELO_1_1BaseUnit_1_1PlainText}{00204} \textcolor{keyword}{class }\hyperlink{classELO_1_1BaseUnit_1_1PlainText}{PlainText}(\hyperlink{classELO_1_1BaseUnit_1_1IfBaseType}{IfBaseType}):
00205     \_value = \textcolor{keywordtype}{None}
00206 
00207     \textcolor{comment}{## Construtor da classe.}
00208     \textcolor{comment}{#   Responsável por impedir a criação de um tipo básico com}
00209     \textcolor{comment}{#   conteúdo inválido.}
00210     \textcolor{comment}{#}
00211     \textcolor{comment}{#   @arg value String a ser validada e armazenada.}
\hypertarget{BaseUnit_8py_source_l00212}{}\hyperlink{classELO_1_1BaseUnit_1_1PlainText_a2cd4f19585b5e8e8279f8052307d031f}{00212}     \textcolor{keyword}{def }\hyperlink{classELO_1_1BaseUnit_1_1PlainText_a2cd4f19585b5e8e8279f8052307d031f}{\_\_init\_\_}(self, value):
00213 
00214         \textcolor{keywordflow}{try}:
00215             self.\hyperlink{classELO_1_1BaseUnit_1_1IfBaseType_a11c68b128a7069e27c1c2fcb782269ea}{\_validate}(value)
00216         \textcolor{keywordflow}{except} ValueError \textcolor{keyword}{as} exc:
00217             del self
00218             \textcolor{keywordflow}{raise} exc
00219         self.\hyperlink{classELO_1_1BaseUnit_1_1IfBaseType_ad05d9d377fc4b99743c022cc8f6019d7}{\_value} = value
00220 
00221     \textcolor{comment}{## Validator da classe.}
00222     \textcolor{comment}{#   Responsável pela devida verificação da validade de value.}
00223     \textcolor{comment}{#}
00224     \textcolor{comment}{#   @arg        value       String a ser validada.}
00225     \textcolor{comment}{#}
00226     \textcolor{comment}{#   @exception  ValueError  Exceção lançada no caso de}
00227     \textcolor{comment}{#                           value exceder 1024 caracteres.}
00228     \textcolor{keyword}{def }\_validate(self, value):
00229         \textcolor{keywordflow}{if} len(value) > 1024:
00230             \textcolor{keywordflow}{raise} ValueError(lang.DICT[\textcolor{stringliteral}{'EXCEPTION\_INV\_PT\_B'}])
00231 
00232 \textcolor{comment}{## Classe container de Campus.}
00233 \textcolor{comment}{#   Irá armazenar um inteiro que identifica univocamente um Campus.}
00234 \textcolor{comment}{#   A associação Campus-Id será tratada em uma tabela à parte.}
\hypertarget{BaseUnit_8py_source_l00235}{}\hyperlink{classELO_1_1BaseUnit_1_1Campus}{00235} \textcolor{keyword}{class }\hyperlink{classELO_1_1BaseUnit_1_1Campus}{Campus}(\hyperlink{classELO_1_1BaseUnit_1_1IfBaseType}{IfBaseType}):
00236 
00237     \_value = \textcolor{keywordtype}{None}
00238 
00239     \textcolor{comment}{## Construtor da classe.}
00240     \textcolor{comment}{#   Responsável por impedir a criação de um tipo básico com}
00241     \textcolor{comment}{#   conteúdo inválido.}
00242     \textcolor{comment}{#}
00243     \textcolor{comment}{#   @arg value Inteiro a ser armazenado.}
\hypertarget{BaseUnit_8py_source_l00244}{}\hyperlink{classELO_1_1BaseUnit_1_1Campus_abd800984df4eb26836d305682d84dfa9}{00244}     \textcolor{keyword}{def }\hyperlink{classELO_1_1BaseUnit_1_1Campus_abd800984df4eb26836d305682d84dfa9}{\_\_init\_\_}(self, value):
00245         \textcolor{keywordflow}{try}:
00246             self.\hyperlink{classELO_1_1BaseUnit_1_1IfBaseType_a11c68b128a7069e27c1c2fcb782269ea}{\_validate}(value)
00247         \textcolor{keywordflow}{except} ValueError \textcolor{keyword}{as} exc:
00248             del self
00249             \textcolor{keywordflow}{raise} exc
00250         self.\hyperlink{classELO_1_1BaseUnit_1_1IfBaseType_ad05d9d377fc4b99743c022cc8f6019d7}{\_value} = value
00251 
00252     \textcolor{comment}{## Validador da classe.}
00253     \textcolor{comment}{#   Capaz de identificar se o valor de entrada está dentro dos}
00254     \textcolor{comment}{#   requisitos do sistema.}
00255     \textcolor{comment}{#}
00256     \textcolor{comment}{#   @value      value       Inteiro a ser validado.}
00257     \textcolor{comment}{#}
00258     \textcolor{comment}{#   @exception  ValueError  Exceção lançada no caso de}
00259     \textcolor{comment}{#                           value ser negativo.}
00260     \textcolor{keyword}{def }\_validate(self, value):
00261         \textcolor{keywordflow}{try}: int(value)
00262         \textcolor{keywordflow}{except} ValueError: \textcolor{keywordflow}{raise} ValueError(lang.DICT[\textcolor{stringliteral}{'EXCEPTION\_INV\_CP\_F'}])
00263 
00264         \textcolor{keywordflow}{if} value <= 0:
00265             \textcolor{keywordflow}{raise} ValueError(lang.DICT[\textcolor{stringliteral}{'EXCEPTION\_INV\_CP\_S'}])
00266 
00267 \textcolor{comment}{## Classe container de sexo.}
00268 \textcolor{comment}{#   Responsável por armazenar um caractere relativo a identificação}
00269 \textcolor{comment}{#   do sexo do usuário.}
\hypertarget{BaseUnit_8py_source_l00270}{}\hyperlink{classELO_1_1BaseUnit_1_1Sex}{00270} \textcolor{keyword}{class }\hyperlink{classELO_1_1BaseUnit_1_1Sex}{Sex}(\hyperlink{classELO_1_1BaseUnit_1_1IfBaseType}{IfBaseType}):
00271     \_value = \textcolor{keywordtype}{None}
00272     
00273     \textcolor{comment}{## Construtor da classe.}
00274     \textcolor{comment}{#   Capaz de impedir a criação de um tipo básico com conteúdo}
00275     \textcolor{comment}{#   inválido.}
00276     \textcolor{comment}{#}
00277     \textcolor{comment}{#   @arg        value       Caractere a ser armazenado no}
00278     \textcolor{comment}{#                           tipo básico.}
00279     \textcolor{comment}{#}
00280     \textcolor{comment}{#   @exception  ValueError  Exceção lançada para advertir}
00281     \textcolor{comment}{#                           que value está inválido.}
\hypertarget{BaseUnit_8py_source_l00282}{}\hyperlink{classELO_1_1BaseUnit_1_1Sex_a2a5214adaf2a657f70cd8cf579e420b5}{00282}     \textcolor{keyword}{def }\hyperlink{classELO_1_1BaseUnit_1_1Sex_a2a5214adaf2a657f70cd8cf579e420b5}{\_\_init\_\_}(self, value):
00283         \textcolor{keywordflow}{try}:
00284             self.\hyperlink{classELO_1_1BaseUnit_1_1IfBaseType_a11c68b128a7069e27c1c2fcb782269ea}{\_validate}(value)
00285         \textcolor{keywordflow}{except} ValueError \textcolor{keyword}{as} exc:
00286             del self
00287             \textcolor{keywordflow}{raise} exc
00288         self.\hyperlink{classELO_1_1BaseUnit_1_1IfBaseType_ad05d9d377fc4b99743c022cc8f6019d7}{\_value} = value
00289 
00290     \textcolor{comment}{## Validador da classe.}
00291     \textcolor{comment}{#   Deve ser capaz de lançar uma exceção no caso de value não ser}
00292     \textcolor{comment}{#   compatível com as especificações do sistema.}
00293     \textcolor{comment}{#}
00294     \textcolor{comment}{#   @arg        value       Caractere a ser validado.}
00295     \textcolor{comment}{#}
00296     \textcolor{comment}{#   @exception  ValueError  Exceção a ser lançada no caso do}
00297     \textcolor{comment}{#                           caractere validado não ser F ou M.}
00298     \textcolor{keyword}{def }\_validate(self, value):
00299         \textcolor{keywordflow}{if} value.upper() != \textcolor{stringliteral}{u"M"} \textcolor{keywordflow}{and} value.upper() != \textcolor{stringliteral}{u"F"}:
00300             \textcolor{keywordflow}{raise} ValueError(lang.DICT[\textcolor{stringliteral}{'EXCEPTION\_INV\_SX\_F'}])
00301 
00302 \textcolor{comment}{## Classe container de endereço do sistema.}
00303 \textcolor{comment}{#   Capaz de armazenar uma string que representa o endereço de um arquivo}
00304 \textcolor{comment}{#   na árvore de diretórios do servidor.}
\hypertarget{BaseUnit_8py_source_l00305}{}\hyperlink{classELO_1_1BaseUnit_1_1Link}{00305} \textcolor{keyword}{class }\hyperlink{classELO_1_1BaseUnit_1_1Link}{Link}(\hyperlink{classELO_1_1BaseUnit_1_1IfBaseType}{IfBaseType}):
00306     \_value = \textcolor{keywordtype}{None}
00307 
00308     \textcolor{comment}{## Construtor da classe.}
00309     \textcolor{comment}{#   Impede a criação de um tipo básico com conteúdo inválido.}
00310     \textcolor{comment}{#}
00311     \textcolor{comment}{#   @arg        value       String a ser armazenada.}
00312     \textcolor{comment}{#}
00313     \textcolor{comment}{#   @exception  ValueError  Exceção lançada no caso da}
00314     \textcolor{comment}{#                           invalidez de value.}
\hypertarget{BaseUnit_8py_source_l00315}{}\hyperlink{classELO_1_1BaseUnit_1_1Link_a0c2308097022f21fc0ded45af15e9172}{00315}     \textcolor{keyword}{def }\hyperlink{classELO_1_1BaseUnit_1_1Link_a0c2308097022f21fc0ded45af15e9172}{\_\_init\_\_}(self, value):
00316         \textcolor{keywordflow}{try}:
00317             self.\hyperlink{classELO_1_1BaseUnit_1_1IfBaseType_a11c68b128a7069e27c1c2fcb782269ea}{\_validate}(value)
00318         \textcolor{keywordflow}{except} ValueError \textcolor{keyword}{as} exc:
00319             del self
00320             \textcolor{keywordflow}{raise} exc
00321         self.\hyperlink{classELO_1_1BaseUnit_1_1IfBaseType_ad05d9d377fc4b99743c022cc8f6019d7}{\_value} = value
00322 
00323     \textcolor{comment}{## Validador da classe.}
00324     \textcolor{comment}{#   Verifica se a string de entrada está de acordo com as}
00325     \textcolor{comment}{#   especificações do sistema.}
00326     \textcolor{comment}{#}
00327     \textcolor{comment}{#   @arg        value       Stringa  ser validada.}
00328     \textcolor{comment}{#   }
00329     \textcolor{comment}{#   @exception  ValueError  Exceção lançada no caso de}
00330     \textcolor{comment}{#                           value não ser composto de}
00331     \textcolor{comment}{#                           caracteres exclusivamente}
00332     \textcolor{comment}{#                           alfanuméricos, com exceção}
00333     \textcolor{comment}{#                           das barras do sistema.}
00334     \textcolor{keyword}{def }\_validate(self, value):
00335         nobar = value.replace(\textcolor{stringliteral}{"/"}, \textcolor{stringliteral}{""})
00336         nobar = value.replace(\textcolor{stringliteral}{"."}, \textcolor{stringliteral}{""})
00337         \textcolor{keywordflow}{if} len(value) == 0:
00338             \textcolor{keywordflow}{raise} ValueError(lang.DICT[\textcolor{stringliteral}{'EXCEPTION\_INV\_LK\_S'}])
00339         \textcolor{keywordflow}{elif} unicode.isalnum(nobar) != \textcolor{keyword}{False} \textcolor{keywordflow}{or} len(nobar) == 0:
00340             \textcolor{keywordflow}{raise} ValueError(lang.DICT[\textcolor{stringliteral}{'EXCEPTION\_INV\_LK\_F'}])
00341 
00342 \textcolor{comment}{## Classe container de notas.}
00343 \textcolor{comment}{#   Armazena números que representam notas dos alunos nas atividades.}
\hypertarget{BaseUnit_8py_source_l00344}{}\hyperlink{classELO_1_1BaseUnit_1_1Grades}{00344} \textcolor{keyword}{class }\hyperlink{classELO_1_1BaseUnit_1_1Grades}{Grades}(\hyperlink{classELO_1_1BaseUnit_1_1IfBaseType}{IfBaseType}):
00345     \_value = \textcolor{keywordtype}{None}
00346 
00347     \textcolor{comment}{## Contrutor da classe.}
00348     \textcolor{comment}{#   Deve ser capaz de abortar a criação do tipo básico no caso do}
00349     \textcolor{comment}{#   valor passado ser inválido.}
00350     \textcolor{comment}{#}
00351     \textcolor{comment}{#   @arg        value       Inteiro a ser armazenado.}
00352     \textcolor{comment}{#}
00353     \textcolor{comment}{#   @exception  ValueError  Exceção lançada no caso de}
00354     \textcolor{comment}{#                           value não atender as}
00355     \textcolor{comment}{#                           especificações do sistema.}
\hypertarget{BaseUnit_8py_source_l00356}{}\hyperlink{classELO_1_1BaseUnit_1_1Grades_af948e06e9c08a7d99da82eae5f851839}{00356}     \textcolor{keyword}{def }\hyperlink{classELO_1_1BaseUnit_1_1Grades_af948e06e9c08a7d99da82eae5f851839}{\_\_init\_\_}(self, value):
00357         \textcolor{keywordflow}{try}:
00358             self.\hyperlink{classELO_1_1BaseUnit_1_1IfBaseType_a11c68b128a7069e27c1c2fcb782269ea}{\_validate}(value)
00359         \textcolor{keywordflow}{except} ValueError \textcolor{keyword}{as} exc:
00360             del self
00361             \textcolor{keywordflow}{raise} exc
00362         self.\hyperlink{classELO_1_1BaseUnit_1_1IfBaseType_ad05d9d377fc4b99743c022cc8f6019d7}{\_value} = value
00363 
00364     \textcolor{comment}{## Validator da classe.}
00365     \textcolor{comment}{#   Deve verificar a validade do argumento a ele passado, lançando}
00366     \textcolor{comment}{#   uma exceção em caso contrário.}
00367     \textcolor{comment}{#}
00368     \textcolor{comment}{#   @arg        value       Inteiro a ser validado.}
00369     \textcolor{comment}{#}
00370     \textcolor{comment}{#   @exception  ValueError  Exceção lançada no caso de}
00371     \textcolor{comment}{#                           value ser negativo ou maior}
00372     \textcolor{comment}{#                           do que 100.}
00373     \textcolor{keyword}{def }\_validate(self, value):
00374         \textcolor{keywordflow}{try}: int(value)
00375         \textcolor{keywordflow}{except} ValueError: \textcolor{keywordflow}{raise} ValueError(lang.DICT[\textcolor{stringliteral}{'EXCEPTION\_INV\_GR\_F'}])
00376 
00377         \textcolor{keywordflow}{if} value < 0:
00378             \textcolor{keywordflow}{raise} ValueError(lang.DICT[\textcolor{stringliteral}{'EXCEPTION\_INV\_GR\_S'}])
00379         \textcolor{keywordflow}{elif} value > 100:
00380             \textcolor{keywordflow}{raise} ValueError(lang.DICT[\textcolor{stringliteral}{'EXCEPTION\_INV\_GR\_B'}])
00381 
00382 \textcolor{comment}{## Classe container de email.}
00383 \textcolor{comment}{#   Este tipo básico irá armazenar emails para contato e identificação}
00384 \textcolor{comment}{#   dos usuários.}
\hypertarget{BaseUnit_8py_source_l00385}{}\hyperlink{classELO_1_1BaseUnit_1_1Mail}{00385} \textcolor{keyword}{class }\hyperlink{classELO_1_1BaseUnit_1_1Mail}{Mail}(\hyperlink{classELO_1_1BaseUnit_1_1IfBaseType}{IfBaseType}):
00386     \_value = \textcolor{keywordtype}{None}
00387 
00388     \textcolor{comment}{## Construtor da classe.}
00389     \textcolor{comment}{#   Deve impedir a criação do tipo básico caso seu conteúdo não}
00390     \textcolor{comment}{#   esteja de acordo com as especificações.}
00391     \textcolor{comment}{#}
00392     \textcolor{comment}{#   @arg        value       String a ser armazenada no}
00393     \textcolor{comment}{#                           tipo básico.}
00394     \textcolor{comment}{#}
00395     \textcolor{comment}{#   @exception  ValueError  Exceção lançada no caso do}
00396     \textcolor{comment}{#                           valor passado não ser válido.}
\hypertarget{BaseUnit_8py_source_l00397}{}\hyperlink{classELO_1_1BaseUnit_1_1Mail_a7b9e31959dacc8c25ecae3e9bc0321aa}{00397}     \textcolor{keyword}{def }\hyperlink{classELO_1_1BaseUnit_1_1Mail_a7b9e31959dacc8c25ecae3e9bc0321aa}{\_\_init\_\_}(self, value):
00398         \textcolor{keywordflow}{try}:
00399             self.\hyperlink{classELO_1_1BaseUnit_1_1IfBaseType_a11c68b128a7069e27c1c2fcb782269ea}{\_validate}(value)
00400         \textcolor{keywordflow}{except} ValueError \textcolor{keyword}{as} exc:
00401             del self
00402             \textcolor{keywordflow}{raise} exc
00403         self.\hyperlink{classELO_1_1BaseUnit_1_1IfBaseType_ad05d9d377fc4b99743c022cc8f6019d7}{\_value} = value
00404 
00405     \textcolor{comment}{## Validator da classe.}
00406     \textcolor{comment}{#   Verifica a validade do argumento passado, lançando uma exceção}
00407     \textcolor{comment}{#   no caso de ser negativa.}
00408     \textcolor{comment}{#}
00409     \textcolor{comment}{#   @arg        value       String a ser validada.}
00410     \textcolor{comment}{#}
00411     \textcolor{comment}{#   @exception  ValueError  Exceção lançada no caso de}
00412     \textcolor{comment}{#                           value possuir mais de um, ou}
00413     \textcolor{comment}{#                           nenhum, arroba; no caso de}
00414     \textcolor{comment}{#                           value possuir caracteres}
00415     \textcolor{comment}{#                           não-alfa-numéricos ou caso a}
00416     \textcolor{comment}{#                           contagem de pontos esteja}
00417     \textcolor{comment}{#                           incorreta.}
00418     \textcolor{keyword}{def }\_validate(self, value):
00419         notag = value.replace(\textcolor{stringliteral}{"@"}, \textcolor{stringliteral}{""}).replace(\textcolor{stringliteral}{"."}, \textcolor{stringliteral}{""})
00420         \textcolor{keywordflow}{if} len(value) == 0:
00421             \textcolor{keywordflow}{raise} ValueError(lang.DICT[\textcolor{stringliteral}{'EXCEPTION\_INV\_ML\_S'}])
00422         \textcolor{keywordflow}{elif} unicode.isalnum(notag) == \textcolor{keyword}{False}:
00423             \textcolor{keywordflow}{raise} ValueError(lang.DICT[\textcolor{stringliteral}{'EXCEPTION\_INV\_ML\_F'}])
00424         \textcolor{keywordflow}{elif} value.count(\textcolor{stringliteral}{'@'}) == 1:
00425             \textcolor{keywordflow}{if} value[value.index(\textcolor{stringliteral}{'@'}):].count(\textcolor{stringliteral}{'.'}) < 1:
00426                 \textcolor{keywordflow}{raise} ValueError(lang.DICT[\textcolor{stringliteral}{'EXCEPTION\_INV\_ML\_F'}])
00427         \textcolor{keywordflow}{else}:
00428             \textcolor{keywordflow}{raise} ValueError(lang.DICT[\textcolor{stringliteral}{'EXCEPTION\_INV\_ML\_F'}])
00429         
00430 \textcolor{comment}{## Classe container de tipo de exercício.}
00431 \textcolor{comment}{#   Este tipo básico contém um inteiro que representa um tipo de exercício.}
00432 \textcolor{comment}{#   Cada tipo de exercício funciona de uma forma diferente, e eles são}
00433 \textcolor{comment}{#   escaláveis.}
\hypertarget{BaseUnit_8py_source_l00434}{}\hyperlink{classELO_1_1BaseUnit_1_1ExType}{00434} \textcolor{keyword}{class }\hyperlink{classELO_1_1BaseUnit_1_1ExType}{ExType}(\hyperlink{classELO_1_1BaseUnit_1_1IfBaseType}{IfBaseType}):
00435     \_value = \textcolor{keywordtype}{None}
00436 
00437     \textcolor{comment}{## Contrutor da classe,.}
00438     \textcolor{comment}{#   Impede a criação do tipo básico para valores de entrada}
00439     \textcolor{comment}{#   inválidos.}
00440     \textcolor{comment}{#}
00441     \textcolor{comment}{#   @arg        value       Valor de entrada, que será}
00442     \textcolor{comment}{#                           armazenado no tipo básico.}
00443     \textcolor{comment}{#}
00444     \textcolor{comment}{#   @exception  ValueError  Será lançada no caso do valor}
00445     \textcolor{comment}{#                           de entrada ser inválido.}
\hypertarget{BaseUnit_8py_source_l00446}{}\hyperlink{classELO_1_1BaseUnit_1_1ExType_aca9932deaf8fe247d16ce17909d3dfbf}{00446}     \textcolor{keyword}{def }\hyperlink{classELO_1_1BaseUnit_1_1ExType_aca9932deaf8fe247d16ce17909d3dfbf}{\_\_init\_\_}(self, value):
00447         \textcolor{keywordflow}{try}:
00448             self.\hyperlink{classELO_1_1BaseUnit_1_1IfBaseType_a11c68b128a7069e27c1c2fcb782269ea}{\_validate}(value)
00449         \textcolor{keywordflow}{except} ValueError \textcolor{keyword}{as} exc:
00450             del self
00451             \textcolor{keywordflow}{raise} exc
00452         self.\hyperlink{classELO_1_1BaseUnit_1_1IfBaseType_ad05d9d377fc4b99743c022cc8f6019d7}{\_value} = value
00453 
00454     \textcolor{comment}{## Validador da classe.}
00455     \textcolor{comment}{#   Lança uma excessão no caso de value não atender aos requisitos}
00456     \textcolor{comment}{#   do sistema.}
00457     \textcolor{comment}{#}
00458     \textcolor{comment}{#   @arg        value       Inteiro que será testado.}
00459     \textcolor{comment}{#}
00460     \textcolor{comment}{#   @exception  ValueError  Será lançado no caso de value}
00461     \textcolor{comment}{#                           ser negativo ou não inteiro.}
00462     \textcolor{keyword}{def }\_validate(self, value):
00463         \textcolor{keywordflow}{try}: int(value)
00464         \textcolor{keywordflow}{except} ValueError: \textcolor{keywordflow}{raise} ValueError(lang.DICT[\textcolor{stringliteral}{'EXCEPTION\_INV\_ET\_F'}])
00465 
00466         \textcolor{keywordflow}{if} value <= 0:
00467             \textcolor{keywordflow}{raise} ValueError(lang.DICT[\textcolor{stringliteral}{'EXCEPTION\_INV\_ET\_S'}])
00468 
00469 \textcolor{comment}{## Classe container de Id.}
00470 \textcolor{comment}{#   Este tipo básico irá armazenar um inteiro, que identifica univocamente}
00471 \textcolor{comment}{#   diversas estruturas dentro do sistema.}
\hypertarget{BaseUnit_8py_source_l00472}{}\hyperlink{classELO_1_1BaseUnit_1_1Id}{00472} \textcolor{keyword}{class }\hyperlink{classELO_1_1BaseUnit_1_1Id}{Id}(\hyperlink{classELO_1_1BaseUnit_1_1IfBaseType}{IfBaseType}):
00473     \_value = \textcolor{keywordtype}{None}
00474 
00475     \textcolor{comment}{## Construtor da classe.}
00476     \textcolor{comment}{#   Responsável por impedir a criação de um tipo básico com}
00477     \textcolor{comment}{#   conteúdo inválido.}
00478     \textcolor{comment}{#}
00479     \textcolor{comment}{#   @arg        value       Inteiro a ser salvo no tipo}
00480     \textcolor{comment}{#                           básico.}
00481     \textcolor{comment}{#}
00482     \textcolor{comment}{#   @exception  ValueError  Será lançada no caso de value}
00483     \textcolor{comment}{#                           não corresponder aos requisitos}
00484     \textcolor{comment}{#                           do sistema.}
\hypertarget{BaseUnit_8py_source_l00485}{}\hyperlink{classELO_1_1BaseUnit_1_1Id_adc3e24ef020fb79dda540cdbec85b7a8}{00485}     \textcolor{keyword}{def }\hyperlink{classELO_1_1BaseUnit_1_1Id_adc3e24ef020fb79dda540cdbec85b7a8}{\_\_init\_\_}(self, value):
00486         \textcolor{keywordflow}{try}:
00487             self.\hyperlink{classELO_1_1BaseUnit_1_1IfBaseType_a11c68b128a7069e27c1c2fcb782269ea}{\_validate}(value)
00488         \textcolor{keywordflow}{except} ValueError \textcolor{keyword}{as} exc:
00489             del self
00490             \textcolor{keywordflow}{raise} exc
00491         self.\hyperlink{classELO_1_1BaseUnit_1_1IfBaseType_ad05d9d377fc4b99743c022cc8f6019d7}{\_value} = value
00492 
00493     \textcolor{comment}{## Validador da classe.}
00494     \textcolor{comment}{#   Lança uma exceção no caso do valor de entrada não bater com os}
00495     \textcolor{comment}{#   requisitos para um Id.}
00496     \textcolor{comment}{#}
00497     \textcolor{comment}{#   @arg        value       Inteiro a ser testado.}
00498     \textcolor{comment}{#}
00499     \textcolor{comment}{#   @exception  ValueError  Exceção a ser lançada no caso}
00500     \textcolor{comment}{#                           de value não ser inteiro ou,}
00501     \textcolor{comment}{#                           caso seja, negativo.}
00502     \textcolor{keyword}{def }\_validate(self, value):
00503         \textcolor{keywordflow}{try}: int(value)
00504         \textcolor{keywordflow}{except} ValueError: \textcolor{keywordflow}{raise} ValueError(lang.DICT[\textcolor{stringliteral}{'EXCEPTION\_INV\_ID\_F'}])
00505 
00506         \textcolor{keywordflow}{if} value < 1:
00507             \textcolor{keywordflow}{raise} ValueError(lang.DICT[\textcolor{stringliteral}{'EXCEPTION\_INV\_ID\_S'}])
00508 
00509 \textcolor{comment}{## Classe container de linguagem.}
00510 \textcolor{comment}{#   Irá armazenar um número que corresponde à língua de preferência do}
00511 \textcolor{comment}{#   usuário. A relação linguagem-id será armazenada numa tabela à parte.}
\hypertarget{BaseUnit_8py_source_l00512}{}\hyperlink{classELO_1_1BaseUnit_1_1Language}{00512} \textcolor{keyword}{class }\hyperlink{classELO_1_1BaseUnit_1_1Language}{Language}(\hyperlink{classELO_1_1BaseUnit_1_1IfBaseType}{IfBaseType}):
00513     \_value = \textcolor{keywordtype}{None}
00514 
00515     \textcolor{comment}{## Construtor da classe.}
00516     \textcolor{comment}{#   Impede a criação de tipos básicos com conteúdo inválido.}
00517     \textcolor{comment}{#}
00518     \textcolor{comment}{#   @arg        value       Inteiro a ser armazenado.}
00519     \textcolor{comment}{#}
00520     \textcolor{comment}{#   @exception  ValueError  Exceção a ser lançada no caso}
00521     \textcolor{comment}{#                           de value não corresponder às}
00522     \textcolor{comment}{#                           expectativas.}
\hypertarget{BaseUnit_8py_source_l00523}{}\hyperlink{classELO_1_1BaseUnit_1_1Language_a0b717ad014a17f87cbe20caf5be56fb8}{00523}     \textcolor{keyword}{def }\hyperlink{classELO_1_1BaseUnit_1_1Language_a0b717ad014a17f87cbe20caf5be56fb8}{\_\_init\_\_}(self, value):
00524         \textcolor{keywordflow}{try}:
00525             self.\hyperlink{classELO_1_1BaseUnit_1_1IfBaseType_a11c68b128a7069e27c1c2fcb782269ea}{\_validate}(value)
00526         \textcolor{keywordflow}{except} ValueError \textcolor{keyword}{as} exc:
00527             del self
00528             \textcolor{keywordflow}{raise} exc
00529         self.\hyperlink{classELO_1_1BaseUnit_1_1IfBaseType_ad05d9d377fc4b99743c022cc8f6019d7}{\_value} = value
00530 
00531     \textcolor{comment}{## Validador da classe.}
00532     \textcolor{comment}{#   Recebe um inteiro e verifica se ele pode ser utilizado como}
00533     \textcolor{comment}{#   Linguagem.}
00534     \textcolor{comment}{#}
00535     \textcolor{comment}{#   @arg        value       Inteiro a ser analisado.}
00536     \textcolor{comment}{#}
00537     \textcolor{comment}{#   @exception  ValueError  Exceção a ser lançada caso}
00538     \textcolor{comment}{#                           value não seja inteiro ou}
00539     \textcolor{comment}{#                           seja negativo.}
00540     \textcolor{keyword}{def }\_validate(self, value):
00541         \textcolor{keywordflow}{try}: int(value)
00542         \textcolor{keywordflow}{except} ValueError: \textcolor{keywordflow}{raise} ValueError(lang.DICT[\textcolor{stringliteral}{'EXCEPTION\_INV\_LG\_F'}])
00543 
00544         \textcolor{keywordflow}{if} value < 1:
00545             \textcolor{keywordflow}{raise} ValueError(lang.DICT[\textcolor{stringliteral}{'EXCEPTION\_INV\_LG\_F'}])
00546 
00547 \textcolor{comment}{## Classe container de Data.}
00548 \textcolor{comment}{#   Este tipo básico armazena uma data para os mais diversos usos.}
\hypertarget{BaseUnit_8py_source_l00549}{}\hyperlink{classELO_1_1BaseUnit_1_1Date}{00549} \textcolor{keyword}{class }\hyperlink{classELO_1_1BaseUnit_1_1Date}{Date}(\hyperlink{classELO_1_1BaseUnit_1_1IfBaseType}{IfBaseType}):
00550     \_value = \textcolor{keywordtype}{None}
00551     
00552     \textcolor{comment}{## Construtor da classe.}
00553     \textcolor{comment}{#   Responsável por impedir a criação de um tipo básico que não}
00554     \textcolor{comment}{#   atenda aos requisitos do sistema.}
00555     \textcolor{comment}{#}
00556     \textcolor{comment}{#   @arg        day     Dia a ser armazenado.}
00557     \textcolor{comment}{#}
00558     \textcolor{comment}{#   @arg        month       Mês a ser armazenado.}
00559     \textcolor{comment}{#}
00560     \textcolor{comment}{#   @arg        year        Ano a ser armazenado.}
00561     \textcolor{comment}{#}
00562     \textcolor{comment}{#   @exception  ValueError  Exceção a ser lançada caso a}
00563     \textcolor{comment}{#                           data seja inválida.}
\hypertarget{BaseUnit_8py_source_l00564}{}\hyperlink{classELO_1_1BaseUnit_1_1Date_a97d924fa5f1b2a1d8afbdbfe17b6a852}{00564}     \textcolor{keyword}{def }\hyperlink{classELO_1_1BaseUnit_1_1Date_a97d924fa5f1b2a1d8afbdbfe17b6a852}{\_\_init\_\_}(self, day, month, year):
00565         \textcolor{keywordflow}{try}:
00566             value[\textcolor{stringliteral}{'day'}] = day
00567             value[\textcolor{stringliteral}{'month'}] = month
00568             value[\textcolor{stringliteral}{'year'}] = year
00569             self.\hyperlink{classELO_1_1BaseUnit_1_1IfBaseType_a11c68b128a7069e27c1c2fcb782269ea}{\_validate}(value)
00570         \textcolor{keywordflow}{except} ValueError \textcolor{keyword}{as} exc:
00571             del self
00572             \textcolor{keywordflow}{raise} exc
00573         self.\hyperlink{classELO_1_1BaseUnit_1_1IfBaseType_ad05d9d377fc4b99743c022cc8f6019d7}{\_value} = value
00574 
00575     \textcolor{comment}{## Validator da classe.}
00576     \textcolor{comment}{#   Método que recebe um dicionário e verifica se a data nele}
00577     \textcolor{comment}{#   contida é válida.}
00578     \textcolor{comment}{#}
00579     \textcolor{comment}{#   @arg        value       Dicionário a ser testado. Deve}
00580     \textcolor{comment}{#                           conter no mínimo três campos:}
00581     \textcolor{comment}{#                           year, month e day.}
00582     \textcolor{comment}{#}
00583     \textcolor{comment}{#   @exception  ValueError  Exceção que será lançada na}
00584     \textcolor{comment}{#                           eventualidade da data inserida}
00585     \textcolor{comment}{#                           ser impossível de ocorrer, ou}
00586     \textcolor{comment}{#                           caso ela seja anterior a 2014.}
00587     \textcolor{keyword}{def }\_validate(self, value):
00588         year = value[\textcolor{stringliteral}{'year'}]
00589         month = value[\textcolor{stringliteral}{'month'}]
00590         day = value[\textcolor{stringliteral}{'day'}]
00591 
00592         \textcolor{keywordflow}{if} year < 2014:
00593             \textcolor{keywordflow}{raise} ValueError(lang.DICT[\textcolor{stringliteral}{'EXCEPTION\_INV\_DT\_Y'}])
00594 
00595         \textcolor{keywordflow}{if} month < 1 \textcolor{keywordflow}{or} month > 12:
00596             \textcolor{keywordflow}{raise} ValueError(lang.DICT[\textcolor{stringliteral}{'EXCEPTION\_INV\_DT\_M'}])
00597 
00598         \textcolor{keywordflow}{if} day < 1:
00599             \textcolor{keywordflow}{raise} ValueError(lang.DICT[\textcolor{stringliteral}{'EXCEPTION\_INV\_DT\_D'}])
00600 
00601         \textcolor{keywordflow}{if} month == 2:
00602             \textcolor{keywordflow}{if} day > 29:
00603                 \textcolor{keywordflow}{raise} ValueError(lang.DICT[\textcolor{stringliteral}{'EXCEPTION\_INV\_DT\_D'}])
00604         \textcolor{keywordflow}{elif} month <= 7:
00605             \textcolor{keywordflow}{if} day > 30 + (1 \textcolor{keywordflow}{if} month % 2 \textcolor{keywordflow}{else} 0):
00606                 \textcolor{keywordflow}{raise} ValueError(lang.DICT[\textcolor{stringliteral}{'EXCEPTION\_INV\_DT\_D'}])
00607         \textcolor{keywordflow}{else}:
00608             \textcolor{keywordflow}{if} day > 30 + (0 \textcolor{keywordflow}{if} month % 2 \textcolor{keywordflow}{else} 1):
00609                 \textcolor{keywordflow}{raise} ValueError(lang.DICT[\textcolor{stringliteral}{'EXCEPTION\_INV\_DT\_D'}])
\end{DoxyCode}
