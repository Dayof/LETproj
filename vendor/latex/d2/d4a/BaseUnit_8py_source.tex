\hypertarget{BaseUnit_8py}{\subsection{Base\-Unit.\-py}
\label{BaseUnit_8py}\index{E\-L\-O/\-E\-L\-O/\-Base\-Unit.\-py@{E\-L\-O/\-E\-L\-O/\-Base\-Unit.\-py}}
}

\begin{DoxyCode}
\hypertarget{BaseUnit_8py_source_l00001}{}\hyperlink{namespaceELO_1_1BaseUnit}{00001} \textcolor{comment}{#coding: utf-8}
00002 
00003 \textcolor{comment}{## @file BaseUnit.py}
00004 \textcolor{comment}{#  Implementa os tipos básicos do programa.}
00005 \textcolor{comment}{#}
00006 \textcolor{comment}{#   Cada classe deve conter o validator, que garante que os tipos básicos}
00007 \textcolor{comment}{#   são compatíveis com o formato especificado nos requisitos.}
00008 
00009 \textcolor{keyword}{from} abc \textcolor{keyword}{import} *
00010 \textcolor{keyword}{import} hashlib
00011 
00012 \textcolor{keyword}{import} \hyperlink{namespaceELO_1_1locale_1_1index}{ELO.locale.index} \textcolor{keyword}{as} lang
00013 
00014 \textcolor{comment}{## Interface para qualquer tipo básico (Base Type) pertencente ao projeto.}
00015 \textcolor{comment}{#   Sua descrição implica que todos os tipos básicos devem ter um método}
00016 \textcolor{comment}{#   \_validate(self, value).}
\hypertarget{BaseUnit_8py_source_l00017}{}\hyperlink{classELO_1_1BaseUnit_1_1IfBaseType}{00017} \textcolor{keyword}{class }\hyperlink{classELO_1_1BaseUnit_1_1IfBaseType}{IfBaseType}:
00018 
00019     \textcolor{comment}{## Especifica que o IfBaseType é uma classe base abstrata (abc). }
00020     \textcolor{comment}{#   Isso significa que uma classe derivada pode ser instanciada se}
00021     \textcolor{comment}{#   - e somente se - ela der "override" em todos os métodos e}
00022     \textcolor{comment}{#   propriedades abstratas.}
00023     \_\_metaclass\_\_ = ABCMeta
00024 
00025     \textcolor{comment}{## Método que retorna o valor contido no tipo básico.}
00026     @property
\hypertarget{BaseUnit_8py_source_l00027}{}\hyperlink{classELO_1_1BaseUnit_1_1IfBaseType_ae4974528321f9314afd17c3da0e9d676}{00027}     \textcolor{keyword}{def }\hyperlink{classELO_1_1BaseUnit_1_1IfBaseType_ae4974528321f9314afd17c3da0e9d676}{value}(self):
00028         \textcolor{keywordflow}{return} self.\hyperlink{classELO_1_1BaseUnit_1_1IfBaseType_ad05d9d377fc4b99743c022cc8f6019d7}{\_value}
00029 
00030     \textcolor{comment}{## Método que fixa e valida o conteúdo do tipo básico.}
00031     \textcolor{comment}{#   @arg aux Recebe o valor que será validado.}
00032     @value.setter
\hypertarget{BaseUnit_8py_source_l00033}{}\hyperlink{classELO_1_1BaseUnit_1_1IfBaseType_ae4974528321f9314afd17c3da0e9d676}{00033}     \textcolor{keyword}{def }\hyperlink{classELO_1_1BaseUnit_1_1IfBaseType_ae4974528321f9314afd17c3da0e9d676}{value}(self, aux):
00034         self.\hyperlink{classELO_1_1BaseUnit_1_1IfBaseType_a11c68b128a7069e27c1c2fcb782269ea}{\_validate}(aux)
00035         self.\hyperlink{classELO_1_1BaseUnit_1_1IfBaseType_ad05d9d377fc4b99743c022cc8f6019d7}{\_value} = aux
00036 
00037     \textcolor{comment}{## Método que deleta o conteúdo do tipo básico.}
00038     @value.deleter
\hypertarget{BaseUnit_8py_source_l00039}{}\hyperlink{classELO_1_1BaseUnit_1_1IfBaseType_ae4974528321f9314afd17c3da0e9d676}{00039}     \textcolor{keyword}{def }\hyperlink{classELO_1_1BaseUnit_1_1IfBaseType_ae4974528321f9314afd17c3da0e9d676}{value}(self):
00040         del self.\hyperlink{classELO_1_1BaseUnit_1_1IfBaseType_ad05d9d377fc4b99743c022cc8f6019d7}{\_value}
00041 
00042     \textcolor{comment}{## Define que toda classe derivada terá um método \_validate.}
00043     @abstractmethod
00044     \textcolor{keyword}{def }\_validate(self, value): \textcolor{keyword}{pass}
00045 
00046     \textcolor{comment}{## Método que compara o conteúdo de dois tipos básicos.}
00047     \textcolor{comment}{#   Será chamado toda vez que dois tipos básicos forem}
00048     \textcolor{comment}{#   comparados através dos operadores >, <, >=, <=, == ou !=.}
00049     \textcolor{comment}{#}
00050     \textcolor{comment}{#   @return Retornos predefinidos do próprio python.}
\hypertarget{BaseUnit_8py_source_l00051}{}\hyperlink{classELO_1_1BaseUnit_1_1IfBaseType_a69c338f6f1574bd3524e9d59ebc17a7c}{00051}     \textcolor{keyword}{def }\hyperlink{classELO_1_1BaseUnit_1_1IfBaseType_a69c338f6f1574bd3524e9d59ebc17a7c}{\_\_cmp\_\_}(self, other):
00052         \textcolor{keywordflow}{if} other.value == self.\hyperlink{classELO_1_1BaseUnit_1_1IfBaseType_ae4974528321f9314afd17c3da0e9d676}{value}:
00053             \textcolor{keywordflow}{return} 0
00054         \textcolor{keywordflow}{elif} other.value > self.\hyperlink{classELO_1_1BaseUnit_1_1IfBaseType_ae4974528321f9314afd17c3da0e9d676}{value}:
00055             \textcolor{keywordflow}{return} 1
00056         \textcolor{keywordflow}{else}:
00057             \textcolor{keywordflow}{return} -1
00058 
00059 \textcolor{comment}{## Classe container de senhas.}
00060 \textcolor{comment}{#   Deve ser responsável não somente por armazenar, mas também por}
00061 \textcolor{comment}{#   encriptar strings passadas.}
\hypertarget{BaseUnit_8py_source_l00062}{}\hyperlink{classELO_1_1BaseUnit_1_1Password}{00062} \textcolor{keyword}{class }\hyperlink{classELO_1_1BaseUnit_1_1Password}{Password}(\hyperlink{classELO_1_1BaseUnit_1_1IfBaseType}{IfBaseType}):
00063     \_value = \textcolor{keywordtype}{None}
00064 
00065     \textcolor{comment}{## Construtor da classe.}
00066     \textcolor{comment}{#   Valida e encripta a string passada.}
00067     \textcolor{comment}{#}
00068     \textcolor{comment}{#   @arg        value       String a ser armazenada.}
00069     \textcolor{comment}{#}
00070     \textcolor{comment}{#   @exception  ValueError  Exceção lançada no caso do}
00071     \textcolor{comment}{#                           valor de entrada ser inválido.}
\hypertarget{BaseUnit_8py_source_l00072}{}\hyperlink{classELO_1_1BaseUnit_1_1Password_a01568369a90b21117ba3beb05cbf9dfe}{00072}     \textcolor{keyword}{def }\hyperlink{classELO_1_1BaseUnit_1_1Password_a01568369a90b21117ba3beb05cbf9dfe}{\_\_init\_\_}(self, value):
00073         \textcolor{keywordflow}{try}:
00074             self.\hyperlink{classELO_1_1BaseUnit_1_1IfBaseType_a11c68b128a7069e27c1c2fcb782269ea}{\_validate}(value)
00075         \textcolor{keywordflow}{except} ValueError \textcolor{keyword}{as} exc:
00076             del self
00077             \textcolor{keywordflow}{raise} exc
00078         self.\hyperlink{classELO_1_1BaseUnit_1_1IfBaseType_ad05d9d377fc4b99743c022cc8f6019d7}{\_value} = hashlib.sha256(value).hexdigest()
00079         self.\hyperlink{classELO_1_1BaseUnit_1_1IfBaseType_ad05d9d377fc4b99743c022cc8f6019d7}{\_value} = hashlib.md5(self.\hyperlink{classELO_1_1BaseUnit_1_1IfBaseType_ad05d9d377fc4b99743c022cc8f6019d7}{\_value}).hexdigest()
00080 
00081     \textcolor{comment}{## Validator da classe.}
00082     \textcolor{comment}{#   Verifica se o tamanho da string passada (value) está correta.}
00083     \textcolor{comment}{#}
00084     \textcolor{comment}{#   @arg        value       String a ser validada.}
00085     \textcolor{comment}{#}
00086     \textcolor{comment}{#   @exception  ValueError  Lançada no caso do tamanho de}
00087     \textcolor{comment}{#                           value ser menor do que 6}
00088     \textcolor{comment}{#                           caracteres.}
00089     \textcolor{keyword}{def }\_validate(self, value):
00090         if(len(value) < 6):
00091             \textcolor{keywordflow}{raise} ValueError(lang.DICT[\textcolor{stringliteral}{'EXCEPTION\_INV\_PW\_S'}])
00092 
00093     @property
\hypertarget{BaseUnit_8py_source_l00094}{}\hyperlink{classELO_1_1BaseUnit_1_1Password_a42dcd63788a2eed3780c39f368356546}{00094}     \textcolor{keyword}{def }\hyperlink{classELO_1_1BaseUnit_1_1Password_a42dcd63788a2eed3780c39f368356546}{value}(self):
00095         \textcolor{keywordflow}{return} self.\hyperlink{classELO_1_1BaseUnit_1_1IfBaseType_ad05d9d377fc4b99743c022cc8f6019d7}{\_value}
00096 
00097     \textcolor{comment}{## Decorador específico de password.}
00098     \textcolor{comment}{#   Sobrescreve o decorador definido na interface para que ocorra}
00099     \textcolor{comment}{#   a encriptação de forma correta.}
00100     \textcolor{comment}{#}
00101     \textcolor{comment}{#   @arg value String a ser validada e encriptada.}
00102     @value.setter
\hypertarget{BaseUnit_8py_source_l00103}{}\hyperlink{classELO_1_1BaseUnit_1_1Password_a42dcd63788a2eed3780c39f368356546}{00103}     \textcolor{keyword}{def }\hyperlink{classELO_1_1BaseUnit_1_1Password_a42dcd63788a2eed3780c39f368356546}{value}(self, value):
00104         self.\hyperlink{classELO_1_1BaseUnit_1_1IfBaseType_a11c68b128a7069e27c1c2fcb782269ea}{\_validate}(value)
00105         self.\hyperlink{classELO_1_1BaseUnit_1_1IfBaseType_ad05d9d377fc4b99743c022cc8f6019d7}{\_value} = hashlib.sha256(value).hexdigest()
00106         self.\hyperlink{classELO_1_1BaseUnit_1_1IfBaseType_ad05d9d377fc4b99743c022cc8f6019d7}{\_value} = hashlib.md5(self.\hyperlink{classELO_1_1BaseUnit_1_1IfBaseType_ad05d9d377fc4b99743c022cc8f6019d7}{\_value}).hexdigest()
00107 
00108 \textcolor{comment}{## Classe container de nomes.}
00109 \textcolor{comment}{#   Responsável por armazenar uma string que servirá para identificação}
00110 \textcolor{comment}{#   dos usuários do sistema.}
\hypertarget{BaseUnit_8py_source_l00111}{}\hyperlink{classELO_1_1BaseUnit_1_1Name}{00111} \textcolor{keyword}{class }\hyperlink{classELO_1_1BaseUnit_1_1Name}{Name}(\hyperlink{classELO_1_1BaseUnit_1_1IfBaseType}{IfBaseType}):
00112 
00113     \_value = \textcolor{keywordtype}{None}
00114 
00115     \textcolor{comment}{## Construtor da classe.}
00116     \textcolor{comment}{#   Responsável por impedir a criação do tipo básico em caso de}
00117     \textcolor{comment}{#   falha.}
00118     \textcolor{comment}{#}
00119     \textcolor{comment}{#   @arg value String a ser armazenada.}
\hypertarget{BaseUnit_8py_source_l00120}{}\hyperlink{classELO_1_1BaseUnit_1_1Name_ad73fdab6426635b79b4447ac53c10279}{00120}     \textcolor{keyword}{def }\hyperlink{classELO_1_1BaseUnit_1_1Name_ad73fdab6426635b79b4447ac53c10279}{\_\_init\_\_}(self, value):
00121         \textcolor{keywordflow}{try}:
00122             self.\hyperlink{classELO_1_1BaseUnit_1_1IfBaseType_a11c68b128a7069e27c1c2fcb782269ea}{\_validate}(value)
00123         \textcolor{keywordflow}{except} ValueError \textcolor{keyword}{as} exc:
00124             del self
00125             \textcolor{keywordflow}{raise} exc
00126         self.\hyperlink{classELO_1_1BaseUnit_1_1IfBaseType_ad05d9d377fc4b99743c022cc8f6019d7}{\_value} = value
00127 
00128     \textcolor{comment}{## Validator da classe.}
00129     \textcolor{comment}{#   Verifica se a string de entrada (value) está de acordo com as}
00130     \textcolor{comment}{#   especificações.}
00131     \textcolor{comment}{#}
00132     \textcolor{comment}{#   @arg       value    String a ser validada.}
00133     \textcolor{comment}{#}
00134     \textcolor{comment}{#   @exception ValueError   Exceção lançada no caso da string de}
00135     \textcolor{comment}{#               entrada ser vazia ou conter algum}
00136     \textcolor{comment}{#               caractere não-alfanumérico.}
00137     \textcolor{keyword}{def }\_validate(self, value):
00138         \textcolor{keywordflow}{if} len(value) > 32:
00139             \textcolor{keywordflow}{raise} ValueError(lang.DICT[\textcolor{stringliteral}{'EXCEPTION\_INV\_NM\_B'}])
00140         \textcolor{keywordflow}{else}: 
00141             \textcolor{keywordflow}{if} len(value) == 0:
00142                 \textcolor{keywordflow}{raise} ValueError(lang.DICT[\textcolor{stringliteral}{'EXCEPTION\_INV\_NM\_S'}])
00143             \textcolor{keywordflow}{else}:
00144                 temp = unicode.isalnum(value.replace(\textcolor{stringliteral}{" "}, \textcolor{stringliteral}{""}))
00145                 \textcolor{keywordflow}{if} temp == \textcolor{keyword}{False}:
00146                     \textcolor{keywordflow}{raise} ValueError(lang.DICT[\textcolor{stringliteral}{'EXCEPTION\_INV\_NM\_F'}])
00147 
00148 \textcolor{comment}{## Classe container de matrículas.}
00149 \textcolor{comment}{#   Responsável por armazenar um número inteiro que identifica}
00150 \textcolor{comment}{#   alguns usuários do sistema.}
\hypertarget{BaseUnit_8py_source_l00151}{}\hyperlink{classELO_1_1BaseUnit_1_1Matric}{00151} \textcolor{keyword}{class }\hyperlink{classELO_1_1BaseUnit_1_1Matric}{Matric}(\hyperlink{classELO_1_1BaseUnit_1_1IfBaseType}{IfBaseType}):
00152 
00153     \_value = \textcolor{keywordtype}{None}
00154 
00155     \textcolor{comment}{## Construtor da classe.}
00156     \textcolor{comment}{#   Responsável por impedir a criação de um tipo básico com}
00157     \textcolor{comment}{#   valor inválido.}
00158     \textcolor{comment}{#}
00159     \textcolor{comment}{#   @arg value Número inteiro a ser armazenado no tipo básico.}
\hypertarget{BaseUnit_8py_source_l00160}{}\hyperlink{classELO_1_1BaseUnit_1_1Matric_a050299f6e8d99a530a4598270b93b8d9}{00160}     \textcolor{keyword}{def }\hyperlink{classELO_1_1BaseUnit_1_1Matric_a050299f6e8d99a530a4598270b93b8d9}{\_\_init\_\_}(self, value):
00161         \textcolor{keywordflow}{try}:
00162             self.\hyperlink{classELO_1_1BaseUnit_1_1IfBaseType_a11c68b128a7069e27c1c2fcb782269ea}{\_validate}(value)
00163         \textcolor{keywordflow}{except} ValueError \textcolor{keyword}{as} exc:
00164             del self
00165             \textcolor{keywordflow}{raise} exc
00166         self.\hyperlink{classELO_1_1BaseUnit_1_1IfBaseType_ad05d9d377fc4b99743c022cc8f6019d7}{\_value} = value
00167 
00168     \textcolor{comment}{## Validator da classe.}
00169     \textcolor{comment}{#   Responsável por garantir a coerência da matrícula armazenada}
00170     \textcolor{comment}{#   com as especificações do sistema.}
00171     \textcolor{comment}{#}
00172     \textcolor{comment}{#   @arg       value    Inteiro a ser validado.}
00173     \textcolor{comment}{#}
00174     \textcolor{comment}{#   @exception ValueError   Exceção lançada no caso de value}
00175     \textcolor{comment}{#                           exceder a faixa dinâmica permitida}
00176     \textcolor{comment}{#                           para a matrícula (1-9 dígitos).}
00177     \textcolor{keyword}{def }\_validate(self, value):
00178         \textcolor{keywordflow}{try}: int(value)
00179         \textcolor{keywordflow}{except} ValueError: \textcolor{keywordflow}{raise} ValueError(lang.DICT[\textcolor{stringliteral}{'EXCEPTION\_INT\_MT\_F'}])
00180 
00181         \textcolor{keywordflow}{if} value > 999999999:
00182             \textcolor{keywordflow}{raise} ValueError(lang.DICT[\textcolor{stringliteral}{'EXCEPTION\_INV\_MT\_B'}])
00183         \textcolor{keywordflow}{elif} value < 1:
00184             \textcolor{keywordflow}{raise} ValueError(lang.DICT[\textcolor{stringliteral}{'EXCEPTION\_INV\_MT\_S'}])
00185         
00186 \textcolor{comment}{## Classe container de texto.}
00187 \textcolor{comment}{#   Responsável por armazenar texto corrido, como em descrições,}
00188 \textcolor{comment}{#   biografias, resumos, etc.}
\hypertarget{BaseUnit_8py_source_l00189}{}\hyperlink{classELO_1_1BaseUnit_1_1PlainText}{00189} \textcolor{keyword}{class }\hyperlink{classELO_1_1BaseUnit_1_1PlainText}{PlainText}(\hyperlink{classELO_1_1BaseUnit_1_1IfBaseType}{IfBaseType}):
00190     \_value = \textcolor{keywordtype}{None}
00191 
00192     \textcolor{comment}{## Construtor da classe.}
00193     \textcolor{comment}{#   Responsável por impedir a criação de um tipo básico com}
00194     \textcolor{comment}{#   conteúdo inválido.}
00195     \textcolor{comment}{#}
00196     \textcolor{comment}{#   @arg value String a ser validada e armazenada.}
\hypertarget{BaseUnit_8py_source_l00197}{}\hyperlink{classELO_1_1BaseUnit_1_1PlainText_a2cd4f19585b5e8e8279f8052307d031f}{00197}     \textcolor{keyword}{def }\hyperlink{classELO_1_1BaseUnit_1_1PlainText_a2cd4f19585b5e8e8279f8052307d031f}{\_\_init\_\_}(self, value):
00198 
00199         \textcolor{keywordflow}{try}:
00200             self.\hyperlink{classELO_1_1BaseUnit_1_1IfBaseType_a11c68b128a7069e27c1c2fcb782269ea}{\_validate}(value)
00201         \textcolor{keywordflow}{except} ValueError \textcolor{keyword}{as} exc:
00202             del self
00203             \textcolor{keywordflow}{raise} exc
00204         self.\hyperlink{classELO_1_1BaseUnit_1_1IfBaseType_ad05d9d377fc4b99743c022cc8f6019d7}{\_value} = value
00205 
00206     \textcolor{comment}{## Validator da classe.}
00207     \textcolor{comment}{#   Responsável pela devida verificação da validade de value.}
00208     \textcolor{comment}{#}
00209     \textcolor{comment}{#   @arg        value       String a ser validada.}
00210     \textcolor{comment}{#}
00211     \textcolor{comment}{#   @exception  ValueError  Exceção lançada no caso de}
00212     \textcolor{comment}{#                           value exceder 1024 caracteres.}
00213     \textcolor{keyword}{def }\_validate(self, value):
00214         \textcolor{keywordflow}{if} len(value) > 1024:
00215             \textcolor{keywordflow}{raise} ValueError(lang.DICT[\textcolor{stringliteral}{'EXCEPTION\_INV\_PT\_B'}])
00216 
00217 \textcolor{comment}{## Classe container de Campus.}
00218 \textcolor{comment}{#   Irá armazenar um inteiro que identifica univocamente um Campus.}
00219 \textcolor{comment}{#   A associação Campus-Id será tratada em uma tabela à parte.}
\hypertarget{BaseUnit_8py_source_l00220}{}\hyperlink{classELO_1_1BaseUnit_1_1Campus}{00220} \textcolor{keyword}{class }\hyperlink{classELO_1_1BaseUnit_1_1Campus}{Campus}(\hyperlink{classELO_1_1BaseUnit_1_1IfBaseType}{IfBaseType}):
00221 
00222     \_value = \textcolor{keywordtype}{None}
00223 
00224     \textcolor{comment}{## Construtor da classe.}
00225     \textcolor{comment}{#   Responsável por impedir a criação de um tipo básico com}
00226     \textcolor{comment}{#   conteúdo inválido.}
00227     \textcolor{comment}{#}
00228     \textcolor{comment}{#   @arg value Inteiro a ser armazenado.}
\hypertarget{BaseUnit_8py_source_l00229}{}\hyperlink{classELO_1_1BaseUnit_1_1Campus_abd800984df4eb26836d305682d84dfa9}{00229}     \textcolor{keyword}{def }\hyperlink{classELO_1_1BaseUnit_1_1Campus_abd800984df4eb26836d305682d84dfa9}{\_\_init\_\_}(self, value):
00230         \textcolor{keywordflow}{try}:
00231             self.\hyperlink{classELO_1_1BaseUnit_1_1IfBaseType_a11c68b128a7069e27c1c2fcb782269ea}{\_validate}(value)
00232         \textcolor{keywordflow}{except} ValueError \textcolor{keyword}{as} exc:
00233             del self
00234             \textcolor{keywordflow}{raise} exc
00235         self.\hyperlink{classELO_1_1BaseUnit_1_1IfBaseType_ad05d9d377fc4b99743c022cc8f6019d7}{\_value} = value
00236 
00237     \textcolor{comment}{## Validador da classe.}
00238     \textcolor{comment}{#   Capaz de identificar se o valor de entrada está dentro dos}
00239     \textcolor{comment}{#   requisitos do sistema.}
00240     \textcolor{comment}{#}
00241     \textcolor{comment}{#   @value      value       Inteiro a ser validado.}
00242     \textcolor{comment}{#}
00243     \textcolor{comment}{#   @exception  ValueError  Exceção lançada no caso de}
00244     \textcolor{comment}{#                           value ser negativo.}
00245     \textcolor{keyword}{def }\_validate(self, value):
00246         \textcolor{keywordflow}{try}: int(value)
00247         \textcolor{keywordflow}{except} ValueError: \textcolor{keywordflow}{raise} ValueError(lang.DICT[\textcolor{stringliteral}{'EXCEPTION\_INV\_CP\_F'}])
00248 
00249         \textcolor{keywordflow}{if} value <= 0:
00250             \textcolor{keywordflow}{raise} ValueError(lang.DICT[\textcolor{stringliteral}{'EXCEPTION\_INV\_CP\_S'}])
00251 
00252 \textcolor{comment}{## Classe container de sexo.}
00253 \textcolor{comment}{#   Responsável por armazenar um caractere relativo a identificação}
00254 \textcolor{comment}{#   do sexo do usuário.}
\hypertarget{BaseUnit_8py_source_l00255}{}\hyperlink{classELO_1_1BaseUnit_1_1Sex}{00255} \textcolor{keyword}{class }\hyperlink{classELO_1_1BaseUnit_1_1Sex}{Sex}(\hyperlink{classELO_1_1BaseUnit_1_1IfBaseType}{IfBaseType}):
00256     \_value = \textcolor{keywordtype}{None}
00257     
00258     \textcolor{comment}{## Construtor da classe.}
00259     \textcolor{comment}{#   Capaz de impedir a criação de um tipo básico com conteúdo}
00260     \textcolor{comment}{#   inválido.}
00261     \textcolor{comment}{#}
00262     \textcolor{comment}{#   @arg        value       Caractere a ser armazenado no}
00263     \textcolor{comment}{#                           tipo básico.}
00264     \textcolor{comment}{#}
00265     \textcolor{comment}{#   @exception  ValueError  Exceção lançada para advertir}
00266     \textcolor{comment}{#                           que value está inválido.}
\hypertarget{BaseUnit_8py_source_l00267}{}\hyperlink{classELO_1_1BaseUnit_1_1Sex_a2a5214adaf2a657f70cd8cf579e420b5}{00267}     \textcolor{keyword}{def }\hyperlink{classELO_1_1BaseUnit_1_1Sex_a2a5214adaf2a657f70cd8cf579e420b5}{\_\_init\_\_}(self, value):
00268         \textcolor{keywordflow}{try}:
00269             self.\hyperlink{classELO_1_1BaseUnit_1_1IfBaseType_a11c68b128a7069e27c1c2fcb782269ea}{\_validate}(value)
00270         \textcolor{keywordflow}{except} ValueError \textcolor{keyword}{as} exc:
00271             del self
00272             \textcolor{keywordflow}{raise} exc
00273         self.\hyperlink{classELO_1_1BaseUnit_1_1IfBaseType_ad05d9d377fc4b99743c022cc8f6019d7}{\_value} = value
00274 
00275     \textcolor{comment}{## Validador da classe.}
00276     \textcolor{comment}{#   Deve ser capaz de lançar uma exceção no caso de value não ser}
00277     \textcolor{comment}{#   compatível com as especificações do sistema.}
00278     \textcolor{comment}{#}
00279     \textcolor{comment}{#   @arg        value       Caractere a ser validado.}
00280     \textcolor{comment}{#}
00281     \textcolor{comment}{#   @exception  ValueError  Exceção a ser lançada no caso do}
00282     \textcolor{comment}{#                           caractere validado não ser F ou M.}
00283     \textcolor{keyword}{def }\_validate(self, value):
00284         \textcolor{keywordflow}{if} value.upper() != \textcolor{stringliteral}{u"M"} \textcolor{keywordflow}{and} value.upper() != \textcolor{stringliteral}{u"F"}:
00285             \textcolor{keywordflow}{raise} ValueError(lang.DICT[\textcolor{stringliteral}{'EXCEPTION\_INV\_SX\_F'}])
00286 
00287 \textcolor{comment}{## Classe container de endereço do sistema.}
00288 \textcolor{comment}{#   Capaz de armazenar uma string que representa o endereço de um arquivo}
00289 \textcolor{comment}{#   na árvore de diretórios do servidor.}
\hypertarget{BaseUnit_8py_source_l00290}{}\hyperlink{classELO_1_1BaseUnit_1_1Link}{00290} \textcolor{keyword}{class }\hyperlink{classELO_1_1BaseUnit_1_1Link}{Link}(\hyperlink{classELO_1_1BaseUnit_1_1IfBaseType}{IfBaseType}):
00291     \_value = \textcolor{keywordtype}{None}
00292 
00293     \textcolor{comment}{## Construtor da classe.}
00294     \textcolor{comment}{#   Impede a criação de um tipo básico com conteúdo inválido.}
00295     \textcolor{comment}{#}
00296     \textcolor{comment}{#   @arg        value       String a ser armazenada.}
00297     \textcolor{comment}{#}
00298     \textcolor{comment}{#   @exception  ValueError  Exceção lançada no caso da}
00299     \textcolor{comment}{#                           invalidez de value.}
\hypertarget{BaseUnit_8py_source_l00300}{}\hyperlink{classELO_1_1BaseUnit_1_1Link_a0c2308097022f21fc0ded45af15e9172}{00300}     \textcolor{keyword}{def }\hyperlink{classELO_1_1BaseUnit_1_1Link_a0c2308097022f21fc0ded45af15e9172}{\_\_init\_\_}(self, value):
00301         \textcolor{keywordflow}{try}:
00302             self.\hyperlink{classELO_1_1BaseUnit_1_1IfBaseType_a11c68b128a7069e27c1c2fcb782269ea}{\_validate}(value)
00303         \textcolor{keywordflow}{except} ValueError \textcolor{keyword}{as} exc:
00304             del self
00305             \textcolor{keywordflow}{raise} exc
00306         self.\hyperlink{classELO_1_1BaseUnit_1_1IfBaseType_ad05d9d377fc4b99743c022cc8f6019d7}{\_value} = value
00307 
00308     \textcolor{comment}{## Validador da classe.}
00309     \textcolor{comment}{#   Verifica se a string de entrada está de acordo com as}
00310     \textcolor{comment}{#   especificações do sistema.}
00311     \textcolor{comment}{#}
00312     \textcolor{comment}{#   @arg        value       Stringa  ser validada.}
00313     \textcolor{comment}{#   }
00314     \textcolor{comment}{#   @exception  ValueError  Exceção lançada no caso de}
00315     \textcolor{comment}{#                           value não ser composto de}
00316     \textcolor{comment}{#                           caracteres exclusivamente}
00317     \textcolor{comment}{#                           alfanuméricos, com exceção}
00318     \textcolor{comment}{#                           das barras do sistema.}
00319     \textcolor{keyword}{def }\_validate(self, value):
00320         nobar = value.replace(\textcolor{stringliteral}{"/"}, \textcolor{stringliteral}{""})
00321         nobar = value.replace(\textcolor{stringliteral}{"."}, \textcolor{stringliteral}{""})
00322         \textcolor{keywordflow}{if} len(value) == 0:
00323             \textcolor{keywordflow}{raise} ValueError(lang.DICT[\textcolor{stringliteral}{'EXCEPTION\_INV\_LK\_S'}])
00324         \textcolor{keywordflow}{elif} unicode.isalnum(nobar) != \textcolor{keyword}{False} \textcolor{keywordflow}{or} len(nobar) == 0:
00325             \textcolor{keywordflow}{raise} ValueError(lang.DICT[\textcolor{stringliteral}{'EXCEPTION\_INV\_LK\_F'}])
00326 
00327 \textcolor{comment}{## Classe container de notas.}
00328 \textcolor{comment}{#   Armazena números que representam notas dos alunos nas atividades.}
\hypertarget{BaseUnit_8py_source_l00329}{}\hyperlink{classELO_1_1BaseUnit_1_1Grades}{00329} \textcolor{keyword}{class }\hyperlink{classELO_1_1BaseUnit_1_1Grades}{Grades}(\hyperlink{classELO_1_1BaseUnit_1_1IfBaseType}{IfBaseType}):
00330     \_value = \textcolor{keywordtype}{None}
00331 
00332     \textcolor{comment}{## Contrutor da classe.}
00333     \textcolor{comment}{#   Deve ser capaz de abortar a criação do tipo básico no caso do}
00334     \textcolor{comment}{#   valor passado ser inválido.}
00335     \textcolor{comment}{#}
00336     \textcolor{comment}{#   @arg        value       Inteiro a ser armazenado.}
00337     \textcolor{comment}{#}
00338     \textcolor{comment}{#   @exception  ValueError  Exceção lançada no caso de}
00339     \textcolor{comment}{#                           value não atender as}
00340     \textcolor{comment}{#                           especificações do sistema.}
\hypertarget{BaseUnit_8py_source_l00341}{}\hyperlink{classELO_1_1BaseUnit_1_1Grades_af948e06e9c08a7d99da82eae5f851839}{00341}     \textcolor{keyword}{def }\hyperlink{classELO_1_1BaseUnit_1_1Grades_af948e06e9c08a7d99da82eae5f851839}{\_\_init\_\_}(self, value):
00342         \textcolor{keywordflow}{try}:
00343             self.\hyperlink{classELO_1_1BaseUnit_1_1IfBaseType_a11c68b128a7069e27c1c2fcb782269ea}{\_validate}(value)
00344         \textcolor{keywordflow}{except} ValueError \textcolor{keyword}{as} exc:
00345             del self
00346             \textcolor{keywordflow}{raise} exc
00347         self.\hyperlink{classELO_1_1BaseUnit_1_1IfBaseType_ad05d9d377fc4b99743c022cc8f6019d7}{\_value} = value
00348 
00349     \textcolor{comment}{## Validator da classe.}
00350     \textcolor{comment}{#   Deve verificar a validade do argumento a ele passado, lançando}
00351     \textcolor{comment}{#   uma exceção em caso contrário.}
00352     \textcolor{comment}{#}
00353     \textcolor{comment}{#   @arg        value       Inteiro a ser validado.}
00354     \textcolor{comment}{#}
00355     \textcolor{comment}{#   @exception  ValueError  Exceção lançada no caso de}
00356     \textcolor{comment}{#                           value ser negativo ou maior}
00357     \textcolor{comment}{#                           do que 100.}
00358     \textcolor{keyword}{def }\_validate(self, value):
00359         \textcolor{keywordflow}{try}: int(value)
00360         \textcolor{keywordflow}{except} ValueError: \textcolor{keywordflow}{raise} ValueError(lang.DICT[\textcolor{stringliteral}{'EXCEPTION\_INV\_GR\_F'}])
00361 
00362         \textcolor{keywordflow}{if} value < 0:
00363             \textcolor{keywordflow}{raise} ValueError(lang.DICT[\textcolor{stringliteral}{'EXCEPTION\_INV\_GR\_S'}])
00364         \textcolor{keywordflow}{elif} value > 100:
00365             \textcolor{keywordflow}{raise} ValueError(lang.DICT[\textcolor{stringliteral}{'EXCEPTION\_INV\_GR\_B'}])
00366 
00367 \textcolor{comment}{## Classe container de email.}
00368 \textcolor{comment}{#   Este tipo básico irá armazenar emails para contato e identificação}
00369 \textcolor{comment}{#   dos usuários.}
\hypertarget{BaseUnit_8py_source_l00370}{}\hyperlink{classELO_1_1BaseUnit_1_1Mail}{00370} \textcolor{keyword}{class }\hyperlink{classELO_1_1BaseUnit_1_1Mail}{Mail}(\hyperlink{classELO_1_1BaseUnit_1_1IfBaseType}{IfBaseType}):
00371     \_value = \textcolor{keywordtype}{None}
00372 
00373     \textcolor{comment}{## Construtor da classe.}
00374     \textcolor{comment}{#   Deve impedir a criação do tipo básico caso seu conteúdo não}
00375     \textcolor{comment}{#   esteja de acordo com as especificações.}
00376     \textcolor{comment}{#}
00377     \textcolor{comment}{#   @arg        value       String a ser armazenada no}
00378     \textcolor{comment}{#                           tipo básico.}
00379     \textcolor{comment}{#}
00380     \textcolor{comment}{#   @exception  ValueError  Exceção lançada no caso do}
00381     \textcolor{comment}{#                           valor passado não ser válido.}
\hypertarget{BaseUnit_8py_source_l00382}{}\hyperlink{classELO_1_1BaseUnit_1_1Mail_a7b9e31959dacc8c25ecae3e9bc0321aa}{00382}     \textcolor{keyword}{def }\hyperlink{classELO_1_1BaseUnit_1_1Mail_a7b9e31959dacc8c25ecae3e9bc0321aa}{\_\_init\_\_}(self, value):
00383         \textcolor{keywordflow}{try}:
00384             self.\hyperlink{classELO_1_1BaseUnit_1_1IfBaseType_a11c68b128a7069e27c1c2fcb782269ea}{\_validate}(value)
00385         \textcolor{keywordflow}{except} ValueError \textcolor{keyword}{as} exc:
00386             del self
00387             \textcolor{keywordflow}{raise} exc
00388         self.\hyperlink{classELO_1_1BaseUnit_1_1IfBaseType_ad05d9d377fc4b99743c022cc8f6019d7}{\_value} = value
00389 
00390     \textcolor{comment}{## Validator da classe.}
00391     \textcolor{comment}{#   Verifica a validade do argumento passado, lançando uma exceção}
00392     \textcolor{comment}{#   no caso de ser negativa.}
00393     \textcolor{comment}{#}
00394     \textcolor{comment}{#   @arg        value       String a ser validada.}
00395     \textcolor{comment}{#}
00396     \textcolor{comment}{#   @exception  ValueError  Exceção lançada no caso de}
00397     \textcolor{comment}{#                           value possuir mais de um, ou}
00398     \textcolor{comment}{#                           nenhum, arroba; no caso de}
00399     \textcolor{comment}{#                           value possuir caracteres}
00400     \textcolor{comment}{#                           não-alfa-numéricos ou caso a}
00401     \textcolor{comment}{#                           contagem de pontos esteja}
00402     \textcolor{comment}{#                           incorreta.}
00403     \textcolor{keyword}{def }\_validate(self, value):
00404         notag = value.replace(\textcolor{stringliteral}{"@"}, \textcolor{stringliteral}{""}).replace(\textcolor{stringliteral}{"."}, \textcolor{stringliteral}{""})
00405         \textcolor{keywordflow}{if} len(value) == 0:
00406             \textcolor{keywordflow}{raise} ValueError(lang.DICT[\textcolor{stringliteral}{'EXCEPTION\_INV\_ML\_S'}])
00407         \textcolor{keywordflow}{elif} unicode.isalnum(notag) == \textcolor{keyword}{False}:
00408             \textcolor{keywordflow}{raise} ValueError(lang.DICT[\textcolor{stringliteral}{'EXCEPTION\_INV\_ML\_F'}])
00409         \textcolor{keywordflow}{elif} value.count(\textcolor{stringliteral}{'@'}) == 1:
00410             \textcolor{keywordflow}{if} value[value.index(\textcolor{stringliteral}{'@'}):].count(\textcolor{stringliteral}{'.'}) < 1:
00411                 \textcolor{keywordflow}{raise} ValueError(lang.DICT[\textcolor{stringliteral}{'EXCEPTION\_INV\_ML\_F'}])
00412         \textcolor{keywordflow}{else}:
00413             \textcolor{keywordflow}{raise} ValueError(lang.DICT[\textcolor{stringliteral}{'EXCEPTION\_INV\_ML\_F'}])
00414         
00415 \textcolor{comment}{## Classe container de tipo de exercício.}
00416 \textcolor{comment}{#   Este tipo básico contém um inteiro que representa um tipo de exercício.}
00417 \textcolor{comment}{#   Cada tipo de exercício funciona de uma forma diferente, e eles são}
00418 \textcolor{comment}{#   escaláveis.}
\hypertarget{BaseUnit_8py_source_l00419}{}\hyperlink{classELO_1_1BaseUnit_1_1ExType}{00419} \textcolor{keyword}{class }\hyperlink{classELO_1_1BaseUnit_1_1ExType}{ExType}(\hyperlink{classELO_1_1BaseUnit_1_1IfBaseType}{IfBaseType}):
00420     \_value = \textcolor{keywordtype}{None}
00421 
00422     \textcolor{comment}{## Contrutor da classe,.}
00423     \textcolor{comment}{#   Impede a criação do tipo básico para valores de entrada}
00424     \textcolor{comment}{#   inválidos.}
00425     \textcolor{comment}{#}
00426     \textcolor{comment}{#   @arg        value       Valor de entrada, que será}
00427     \textcolor{comment}{#                           armazenado no tipo básico.}
00428     \textcolor{comment}{#}
00429     \textcolor{comment}{#   @exception  ValueError  Será lançada no caso do valor}
00430     \textcolor{comment}{#                           de entrada ser inválido.}
\hypertarget{BaseUnit_8py_source_l00431}{}\hyperlink{classELO_1_1BaseUnit_1_1ExType_aca9932deaf8fe247d16ce17909d3dfbf}{00431}     \textcolor{keyword}{def }\hyperlink{classELO_1_1BaseUnit_1_1ExType_aca9932deaf8fe247d16ce17909d3dfbf}{\_\_init\_\_}(self, value):
00432         \textcolor{keywordflow}{try}:
00433             self.\hyperlink{classELO_1_1BaseUnit_1_1IfBaseType_a11c68b128a7069e27c1c2fcb782269ea}{\_validate}(value)
00434         \textcolor{keywordflow}{except} ValueError \textcolor{keyword}{as} exc:
00435             del self
00436             \textcolor{keywordflow}{raise} exc
00437         self.\hyperlink{classELO_1_1BaseUnit_1_1IfBaseType_ad05d9d377fc4b99743c022cc8f6019d7}{\_value} = value
00438 
00439     \textcolor{comment}{## Validador da classe.}
00440     \textcolor{comment}{#   Lança uma excessão no caso de value não atender aos requisitos}
00441     \textcolor{comment}{#   do sistema.}
00442     \textcolor{comment}{#}
00443     \textcolor{comment}{#   @arg        value       Inteiro que será testado.}
00444     \textcolor{comment}{#}
00445     \textcolor{comment}{#   @exception  ValueError  Será lançado no caso de value}
00446     \textcolor{comment}{#                           ser negativo ou não inteiro.}
00447     \textcolor{keyword}{def }\_validate(self, value):
00448         \textcolor{keywordflow}{try}: int(value)
00449         \textcolor{keywordflow}{except} ValueError: \textcolor{keywordflow}{raise} ValueError(lang.DICT[\textcolor{stringliteral}{'EXCEPTION\_INV\_ET\_F'}])
00450 
00451         \textcolor{keywordflow}{if} value <= 0:
00452             \textcolor{keywordflow}{raise} ValueError(lang.DICT[\textcolor{stringliteral}{'EXCEPTION\_INV\_ET\_S'}])
00453 
00454 \textcolor{comment}{## Classe container de Id.}
00455 \textcolor{comment}{#   Este tipo básico irá armazenar um inteiro, que identifica univocamente}
00456 \textcolor{comment}{#   diversas estruturas dentro do sistema.}
\hypertarget{BaseUnit_8py_source_l00457}{}\hyperlink{classELO_1_1BaseUnit_1_1Id}{00457} \textcolor{keyword}{class }\hyperlink{classELO_1_1BaseUnit_1_1Id}{Id}(\hyperlink{classELO_1_1BaseUnit_1_1IfBaseType}{IfBaseType}):
00458     \_value = \textcolor{keywordtype}{None}
00459 
00460     \textcolor{comment}{## Construtor da classe.}
00461     \textcolor{comment}{#   Responsável por impedir a criação de um tipo básico com}
00462     \textcolor{comment}{#   conteúdo inválido.}
00463     \textcolor{comment}{#}
00464     \textcolor{comment}{#   @arg        value       Inteiro a ser salvo no tipo}
00465     \textcolor{comment}{#                           básico.}
00466     \textcolor{comment}{#}
00467     \textcolor{comment}{#   @exception  ValueError  Será lançada no caso de value}
00468     \textcolor{comment}{#                           não corresponder aos requisitos}
00469     \textcolor{comment}{#                           do sistema.}
\hypertarget{BaseUnit_8py_source_l00470}{}\hyperlink{classELO_1_1BaseUnit_1_1Id_adc3e24ef020fb79dda540cdbec85b7a8}{00470}     \textcolor{keyword}{def }\hyperlink{classELO_1_1BaseUnit_1_1Id_adc3e24ef020fb79dda540cdbec85b7a8}{\_\_init\_\_}(self, value):
00471         \textcolor{keywordflow}{try}:
00472             self.\hyperlink{classELO_1_1BaseUnit_1_1IfBaseType_a11c68b128a7069e27c1c2fcb782269ea}{\_validate}(value)
00473         \textcolor{keywordflow}{except} ValueError \textcolor{keyword}{as} exc:
00474             del self
00475             \textcolor{keywordflow}{raise} exc
00476         self.\hyperlink{classELO_1_1BaseUnit_1_1IfBaseType_ad05d9d377fc4b99743c022cc8f6019d7}{\_value} = value
00477 
00478     \textcolor{comment}{## Validador da classe.}
00479     \textcolor{comment}{#   Lança uma exceção no caso do valor de entrada não bater com os}
00480     \textcolor{comment}{#   requisitos para um Id.}
00481     \textcolor{comment}{#}
00482     \textcolor{comment}{#   @arg        value       Inteiro a ser testado.}
00483     \textcolor{comment}{#}
00484     \textcolor{comment}{#   @exception  ValueError  Exceção a ser lançada no caso}
00485     \textcolor{comment}{#                           de value não ser inteiro ou,}
00486     \textcolor{comment}{#                           caso seja, negativo.}
00487     \textcolor{keyword}{def }\_validate(self, value):
00488         \textcolor{keywordflow}{try}: int(value)
00489         \textcolor{keywordflow}{except} ValueError: \textcolor{keywordflow}{raise} ValueError(lang.DICT[\textcolor{stringliteral}{'EXCEPTION\_INV\_ID\_F'}])
00490 
00491         \textcolor{keywordflow}{if} value < 1:
00492             \textcolor{keywordflow}{raise} ValueError(lang.DICT[\textcolor{stringliteral}{'EXCEPTION\_INV\_ID\_S'}])
00493 
00494 \textcolor{comment}{## Classe container de linguagem.}
00495 \textcolor{comment}{#   Irá armazenar um número que corresponde à língua de preferência do}
00496 \textcolor{comment}{#   usuário. A relação linguagem-id será armazenada numa tabela à parte.}
\hypertarget{BaseUnit_8py_source_l00497}{}\hyperlink{classELO_1_1BaseUnit_1_1Language}{00497} \textcolor{keyword}{class }\hyperlink{classELO_1_1BaseUnit_1_1Language}{Language}(\hyperlink{classELO_1_1BaseUnit_1_1IfBaseType}{IfBaseType}):
00498     \_value = \textcolor{keywordtype}{None}
00499 
00500     \textcolor{comment}{## Construtor da classe.}
00501     \textcolor{comment}{#   Impede a criação de tipos básicos com conteúdo inválido.}
00502     \textcolor{comment}{#}
00503     \textcolor{comment}{#   @arg        value       Inteiro a ser armazenado.}
00504     \textcolor{comment}{#}
00505     \textcolor{comment}{#   @exception  ValueError  Exceção a ser lançada no caso}
00506     \textcolor{comment}{#                           de value não corresponder às}
00507     \textcolor{comment}{#                           expectativas.}
\hypertarget{BaseUnit_8py_source_l00508}{}\hyperlink{classELO_1_1BaseUnit_1_1Language_a0b717ad014a17f87cbe20caf5be56fb8}{00508}     \textcolor{keyword}{def }\hyperlink{classELO_1_1BaseUnit_1_1Language_a0b717ad014a17f87cbe20caf5be56fb8}{\_\_init\_\_}(self, value):
00509         \textcolor{keywordflow}{try}:
00510             self.\hyperlink{classELO_1_1BaseUnit_1_1IfBaseType_a11c68b128a7069e27c1c2fcb782269ea}{\_validate}(value)
00511         \textcolor{keywordflow}{except} ValueError \textcolor{keyword}{as} exc:
00512             del self
00513             \textcolor{keywordflow}{raise} exc
00514         self.\hyperlink{classELO_1_1BaseUnit_1_1IfBaseType_ad05d9d377fc4b99743c022cc8f6019d7}{\_value} = value
00515 
00516     \textcolor{comment}{## Validador da classe.}
00517     \textcolor{comment}{#   Recebe um inteiro e verifica se ele pode ser utilizado como}
00518     \textcolor{comment}{#   Linguagem.}
00519     \textcolor{comment}{#}
00520     \textcolor{comment}{#   @arg        value       Inteiro a ser analisado.}
00521     \textcolor{comment}{#}
00522     \textcolor{comment}{#   @exception  ValueError  Exceção a ser lançada caso}
00523     \textcolor{comment}{#                           value não seja inteiro ou}
00524     \textcolor{comment}{#                           seja negativo.}
00525     \textcolor{keyword}{def }\_validate(self, value):
00526         \textcolor{keywordflow}{try}: int(value)
00527         \textcolor{keywordflow}{except} ValueError: \textcolor{keywordflow}{raise} ValueError(lang.DICT[\textcolor{stringliteral}{'EXCEPTION\_INV\_LG\_F'}])
00528 
00529         \textcolor{keywordflow}{if} value < 1:
00530             \textcolor{keywordflow}{raise} ValueError(lang.DICT[\textcolor{stringliteral}{'EXCEPTION\_INV\_LG\_F'}])
00531 
00532 \textcolor{comment}{## Classe container de Data.}
00533 \textcolor{comment}{#   Este tipo básico armazena uma data para os mais diversos usos.}
\hypertarget{BaseUnit_8py_source_l00534}{}\hyperlink{classELO_1_1BaseUnit_1_1Date}{00534} \textcolor{keyword}{class }\hyperlink{classELO_1_1BaseUnit_1_1Date}{Date}(\hyperlink{classELO_1_1BaseUnit_1_1IfBaseType}{IfBaseType}):
00535     \_value = \textcolor{keywordtype}{None}
00536     
00537     \textcolor{comment}{## Construtor da classe.}
00538     \textcolor{comment}{#   Responsável por impedir a criação de um tipo básico que não}
00539     \textcolor{comment}{#   atenda aos requisitos do sistema.}
00540     \textcolor{comment}{#}
00541     \textcolor{comment}{#   @arg        day     Dia a ser armazenado.}
00542     \textcolor{comment}{#}
00543     \textcolor{comment}{#   @arg        month       Mês a ser armazenado.}
00544     \textcolor{comment}{#}
00545     \textcolor{comment}{#   @arg        year        Ano a ser armazenado.}
00546     \textcolor{comment}{#}
00547     \textcolor{comment}{#   @exception  ValueError  Exceção a ser lançada caso a}
00548     \textcolor{comment}{#                           data seja inválida.}
\hypertarget{BaseUnit_8py_source_l00549}{}\hyperlink{classELO_1_1BaseUnit_1_1Date_a97d924fa5f1b2a1d8afbdbfe17b6a852}{00549}     \textcolor{keyword}{def }\hyperlink{classELO_1_1BaseUnit_1_1Date_a97d924fa5f1b2a1d8afbdbfe17b6a852}{\_\_init\_\_}(self, day, month, year):
00550         \textcolor{keywordflow}{try}:
00551             value[\textcolor{stringliteral}{'day'}] = day
00552             value[\textcolor{stringliteral}{'month'}] = month
00553             value[\textcolor{stringliteral}{'year'}] = year
00554             self.\hyperlink{classELO_1_1BaseUnit_1_1IfBaseType_a11c68b128a7069e27c1c2fcb782269ea}{\_validate}(value)
00555         \textcolor{keywordflow}{except} ValueError \textcolor{keyword}{as} exc:
00556             del self
00557             \textcolor{keywordflow}{raise} exc
00558         self.\hyperlink{classELO_1_1BaseUnit_1_1IfBaseType_ad05d9d377fc4b99743c022cc8f6019d7}{\_value} = value
00559 
00560     \textcolor{comment}{## Validator da classe.}
00561     \textcolor{comment}{#   Método que recebe um dicionário e verifica se a data nele}
00562     \textcolor{comment}{#   contida é válida.}
00563     \textcolor{comment}{#}
00564     \textcolor{comment}{#   @arg        value       Dicionário a ser testado. Deve}
00565     \textcolor{comment}{#                           conter no mínimo três campos:}
00566     \textcolor{comment}{#                           year, month e day.}
00567     \textcolor{comment}{#}
00568     \textcolor{comment}{#   @exception  ValueError  Exceção que será lançada na}
00569     \textcolor{comment}{#                           eventualidade da data inserida}
00570     \textcolor{comment}{#                           ser impossível de ocorrer, ou}
00571     \textcolor{comment}{#                           caso ela seja anterior a 2014.}
00572     \textcolor{keyword}{def }\_validate(self, value):
00573         year = value[\textcolor{stringliteral}{'year'}]
00574         month = value[\textcolor{stringliteral}{'month'}]
00575         day = value[\textcolor{stringliteral}{'day'}]
00576 
00577         \textcolor{keywordflow}{if} year < 2014:
00578             \textcolor{keywordflow}{raise} ValueError(lang.DICT[\textcolor{stringliteral}{'EXCEPTION\_INV\_DT\_Y'}])
00579 
00580         \textcolor{keywordflow}{if} month < 1 \textcolor{keywordflow}{or} month > 12:
00581             \textcolor{keywordflow}{raise} ValueError(lang.DICT[\textcolor{stringliteral}{'EXCEPTION\_INV\_DT\_M'}])
00582 
00583         \textcolor{keywordflow}{if} day < 1:
00584             \textcolor{keywordflow}{raise} ValueError(lang.DICT[\textcolor{stringliteral}{'EXCEPTION\_INV\_DT\_D'}])
00585 
00586         \textcolor{keywordflow}{if} month == 2:
00587             \textcolor{keywordflow}{if} day > 29:
00588                 \textcolor{keywordflow}{raise} ValueError(lang.DICT[\textcolor{stringliteral}{'EXCEPTION\_INV\_DT\_D'}])
00589         \textcolor{keywordflow}{elif} month <= 7:
00590             \textcolor{keywordflow}{if} day > 30 + (1 \textcolor{keywordflow}{if} month % 2 \textcolor{keywordflow}{else} 0):
00591                 \textcolor{keywordflow}{raise} ValueError(lang.DICT[\textcolor{stringliteral}{'EXCEPTION\_INV\_DT\_D'}])
00592         \textcolor{keywordflow}{else}:
00593             \textcolor{keywordflow}{if} day > 30 + (0 \textcolor{keywordflow}{if} month % 2 \textcolor{keywordflow}{else} 1):
00594                 \textcolor{keywordflow}{raise} ValueError(lang.DICT[\textcolor{stringliteral}{'EXCEPTION\_INV\_DT\_D'}])
\end{DoxyCode}
