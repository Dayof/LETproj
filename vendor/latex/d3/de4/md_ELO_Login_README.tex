\subsection*{Função}

Pasta que contém o módulo de login. Responsável pelo controle de login e logout dos usuários do sistema. É capaz de validar nomes e senhas.

\subsection*{Conteúdo}

As entradas em negrito indicam pastas, e as normais indicam arquivos isolados.


\begin{DoxyItemize}
\item {\bfseries init}.py
\item forms.\-py
\item \hyperlink{LoginUnit_8py}{Login\-Unit.\-py}
\end{DoxyItemize}

\subsection*{\-\_\-\-\_\-init\-\_\-\-\_\-.\-py}

Arquivo requisitado pelo python para reconhecer a pasta como um repositório de arquivos referenciável. Em outras palavras, é um arquivo que só é útil para a linguagem e deve ser ignorado em outras instâncias.

\subsection*{forms.\-py}

Arquivo que contém a definição das classes que modelam os formulários deste módulo. Neste caso, ele é responsável por modelar a form de login, ou seja, definir as caixas de entrada de nome e senha, sem as quais seria impossível realizar o login.

\subsection*{\hyperlink{LoginUnit_8py}{Login\-Unit.\-py}}

Arquivo principal do módulo. Contém as diferentes camadas, bem como suas devidas interfaces e variadas implementações. É o arquivo que recebe o controle do programa após a chamada da Factory (ver \hyperlink{MainUnit_8py}{Main\-Unit.\-py} dentro de L\-E\-T\-Project.\-E\-L\-O.\-E\-L\-O). É responsável por se utilizar das classes formulário (ver forms.\-py) bem como dos outros recursos do programa -\/ como o banco de dados e os cookies -\/ para criar a página que será mostrada para o usuário. No caso da Login\-Unit, ela deve ser responsável por barrar o acesso de usuários não registrados, ou que não consigam inserir a senha. 