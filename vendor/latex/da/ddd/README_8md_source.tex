\hypertarget{README_8md_source}{\subsection{R\-E\-A\-D\-M\-E.\-md}
}

\begin{DoxyCode}
00001 \textcolor{preprocessor}{# LETproject}
00002 \textcolor{preprocessor}{}
00003 \textcolor{preprocessor}{## Index}
00004 \textcolor{preprocessor}{}
00005 1. Definição
00006 2. Dados \textcolor{keywordflow}{do} projeto
00007 3. Documentação
00008   3.1. Doxygen (html)
00009 4. Instalação e execução
00010 5. Metas de desenvolvimento
00011 6. Log de Atividades
00012 
00013 \textcolor{preprocessor}{## Conteúdo}
00014 \textcolor{preprocessor}{}
00015 \textcolor{preprocessor}{### 1. Definição}
00016 \textcolor{preprocessor}{}
00017 O LETproject é uma iniciativa \textcolor{keywordflow}{do} laboratório \textcolor{keywordflow}{do} LET para o desenvolvimento de uma plataforma de ensino de 
      línguas online.
00018 Seu principal objetivo é criar uma \textcolor{keyword}{interface }amigável entre alunos e professores de forma a incluir o 
      estudo doméstico de línguas no mundo informatizado.
00019 
00020 ### 2. Dados do projeto
00021 
00022 Título do projeto:
00023 * **LETproject (nome tempor�
* **ELO (Ensino de Línguas Online) (nome temporário)**
* **SALiE (nome final)**

Orientador:
**Professor Cláudio Correa e Castro Gonçalves**

Unidade Acadêmica / Departamento:
**Instituto de Letras/IL - Departamento de Línguas Estrangeiras e Tradução/LET**

Alunos Envolvidos:
* Yurick Hauschild Caetano da Costa 12/0024136
* Dayanne Fernandes 13/0107191
* Bruna Luisa xx/xxxxxxx

### 3. Documentação

Toda a documentação do projeto pode ser encontrada na pasta doc/.

> **ATENÇÃO**:
> A documentação acima referenciada inclui explanações das funções de
> todas as classes e métodos implementados e tem por público alvo
> principalmente desenvolvedores interessados em contribuir com o projeto.

> Para uma versão específica para usuários, ou seja, pessoas que estão interessadas no funcionamento da
       ferramenta e não em seus mecanismos, leia o ponto *4. Instalação e execução*

#### 3.i. Doxyfile (html)

 No intuito de simplificar a navegação dentro da documentação do projeto, utilizamos a ferramenta Doxygen,
       gerando, assim, um arquivo HTML que contém uma interface amigável, bem como um arquivo .tex capaz de gerar
       um pdf.
 Tanto o pdf quanto o html possuem as mesmas informações.
 Estes arquivos estão contidos em vendor e para acessá-los basta abrir o arquivo doc/html/index.html ou
       doc/latex/refman.pdf com o seu navegador ou visualizador de pdf, respectivamente.
 
### 4. Instalação e execução

Para executar o programa, siga as instruções abaixo.

1. Baixe o código fonte do projeto.
  Para isso, basta clicar no botão "Download ZIP" ao lado deste arquivo, como na imagem abaixo.  
![example1](http://i.imgur.com/kJtzWwf.jpg)

2. Para instalar o python no Windows:

 Obs: Não desenvolvemos a plataforma em Windows, e não temos o hábito de testar neste ambiente. Então, caro
       usuário da microsoft, na ocasião de encontrar algum bug ocasionado por incompatibilidade, sintá-se
       convidado a abrir uma issue e nos avisar, para que o corrijamos o mais rápido possível.
 [Siga este tutorial](http://docs.python-guide.org/en/latest/starting/install/win/).

3. Para instalar o python no Linux:
> OBS: As últimas versões do Ubuntu e Fedora já vêm com o python 2.7 e as últimas versões do RHEL e CentOS
       já vêm com o python 2.6.

  * Instale o pip:
    `$ easy\_install pip`
  * Instale o virtualenv:
    `$ pip install virtualenv`
  * Selecione o diretório para a instalação do python e execute o virtualenv.
    `$ virtualenv --distributte venv`
  * Para executar o ambiente, rode:
    `$ source venv/bin/activate`
  * Para sair do ambiente:
    `$ deactivate`

4. Para instalar django no Windows:
  * Baixe [Download Django-1.7.1.tar.gz](https://www.djangoproject.com/download/1.7.1/tarball/). Então
       extraia o arquivo, inicie o DOS
 shell (ctrl+E, cmd) com permissão de administrador e execute o comando no diretório cujo nome inicie com
       "Django-":
    `$ python setup.py install`

5. Para instalar django no Ubuntu:
  * Pelo pip:
    `$ pip install Django==1.7`
  * "Manualmente":
    * Baixe [Download Django-1.7.1.tar.gz](https://www.djangoproject.com/download/1.7.1/tarball/). Então:
        `$ tar xzvf Django-1.6.2.tar.gz`
        `$ cd Django-1.6.2`
        `$ sudo python setup.py install`
�
* **ELO (Ensino de Línguas Online) (nome temporário)**
* **SALiE (nome final)**

Orientador:
**Professor Cláudio Correa e Castro Gonçalves**

Unidade Acadêmica / Departamento:
**Instituto de Letras/IL - Departamento de Línguas Estrangeiras e Tradução/LET**

Alunos Envolvidos:
* Yurick Hauschild Caetano da Costa 12/0024136
* Dayanne Fernandes 13/0107191
* Bruna Luisa xx/xxxxxxx

### 3. Documentação

Toda a documentação do projeto pode ser encontrada na pasta doc/.

> **ATENÇÃO**:
> A documentação acima referenciada inclui explanações das funções de
> todas as classes e métodos implementados e tem por público alvo
> principalmente desenvolvedores interessados em contribuir com o projeto.

> Para uma versão específica para usuários, ou seja, pessoas que estão interessadas no funcionamento da
       ferramenta e não em seus mecanismos, leia o ponto *4. Instalação e execução*

#### 3.i. Doxyfile (html)

 No intuito de simplificar a navegação dentro da documentação do projeto, utilizamos a ferramenta Doxygen,
       gerando, assim, um arquivo HTML que contém uma interface amigável, bem como um arquivo .tex capaz de gerar
       um pdf.
 Tanto o pdf quanto o html possuem as mesmas informações.
 Estes arquivos estão contidos em vendor e para acessá-los basta abrir o arquivo doc/html/index.html ou
       doc/latex/refman.pdf com o seu navegador ou visualizador de pdf, respectivamente.
 
### 4. Instalação e execução

Para executar o programa, siga as instruções abaixo.

1. Baixe o código fonte do projeto.
  Para isso, basta clicar no botão "Download ZIP" ao lado deste arquivo, como na imagem abaixo.  
![example1](http://i.imgur.com/kJtzWwf.jpg)

2. Para instalar o python no Windows:

 Obs: Não desenvolvemos a plataforma em Windows, e não temos o hábito de testar neste ambiente. Então, caro
       usuário da microsoft, na ocasião de encontrar algum bug ocasionado por incompatibilidade, sintá-se
       convidado a abrir uma issue e nos avisar, para que o corrijamos o mais rápido possível.
 [Siga este tutorial](http://docs.python-guide.org/en/latest/starting/install/win/).

3. Para instalar o python no Linux:
> OBS: As últimas versões do Ubuntu e Fedora já vêm com o python 2.7 e as últimas versões do RHEL e CentOS
       já vêm com o python 2.6.

  * Instale o pip:
    `$ easy\_install pip`
  * Instale o virtualenv:
    `$ pip install virtualenv`
  * Selecione o diretório para a instalação do python e execute o virtualenv.
    `$ virtualenv --distributte venv`
  * Para executar o ambiente, rode:
    `$ source venv/bin/activate`
  * Para sair do ambiente:
    `$ deactivate`

4. Para instalar django no Windows:
  * Baixe [Download Django-1.7.1.tar.gz](https://www.djangoproject.com/download/1.7.1/tarball/). Então
       extraia o arquivo, inicie o DOS
 shell (ctrl+E, cmd) com permissão de administrador e execute o comando no diretório cujo nome inicie com
       "Django-":
    `$ python setup.py install`

5. Para instalar django no Ubuntu:
  * Pelo pip:
    `$ pip install Django==1.7`
  * "Manualmente":
    * Baixe [Download Django-1.7.1.tar.gz](https://www.djangoproject.com/download/1.7.1/tarball/). Então:
        `$ tar xzvf Django-1.6.2.tar.gz`
        `$ cd Django-1.6.2`
        `$ sudo python setup.py install`
rio)**
00024 * **ELO (Ensino de L� Online) (nome temporário)**
* **SALiE (nome final)**

Orientador:
**Professor Cláudio Correa e Castro Gonçalves**

Unidade Acadêmica / Departamento:
**Instituto de Letras/IL - Departamento de Línguas Estrangeiras e Tradução/LET**

Alunos Envolvidos:
* Yurick Hauschild Caetano da Costa 12/0024136
* Dayanne Fernandes 13/0107191
* Bruna Luisa xx/xxxxxxx

### 3. Documentação

Toda a documentação do projeto pode ser encontrada na pasta doc/.

> **ATENÇÃO**:
> A documentação acima referenciada inclui explanações das funções de
> todas as classes e métodos implementados e tem por público alvo
> principalmente desenvolvedores interessados em contribuir com o projeto.

> Para uma versão específica para usuários, ou seja, pessoas que estão interessadas no funcionamento da
       ferramenta e não em seus mecanismos, leia o ponto *4. Instalação e execução*

#### 3.i. Doxyfile (html)

 No intuito de simplificar a navegação dentro da documentação do projeto, utilizamos a ferramenta Doxygen,
       gerando, assim, um arquivo HTML que contém uma interface amigável, bem como um arquivo .tex capaz de gerar
       um pdf.
 Tanto o pdf quanto o html possuem as mesmas informações.
 Estes arquivos estão contidos em vendor e para acessá-los basta abrir o arquivo doc/html/index.html ou
       doc/latex/refman.pdf com o seu navegador ou visualizador de pdf, respectivamente.
 
### 4. Instalação e execução

Para executar o programa, siga as instruções abaixo.

1. Baixe o código fonte do projeto.
  Para isso, basta clicar no botão "Download ZIP" ao lado deste arquivo, como na imagem abaixo.  
![example1](http://i.imgur.com/kJtzWwf.jpg)

2. Para instalar o python no Windows:

 Obs: Não desenvolvemos a plataforma em Windows, e não temos o hábito de testar neste ambiente. Então, caro
       usuário da microsoft, na ocasião de encontrar algum bug ocasionado por incompatibilidade, sintá-se
       convidado a abrir uma issue e nos avisar, para que o corrijamos o mais rápido possível.
 [Siga este tutorial](http://docs.python-guide.org/en/latest/starting/install/win/).

3. Para instalar o python no Linux:
> OBS: As últimas versões do Ubuntu e Fedora já vêm com o python 2.7 e as últimas versões do RHEL e CentOS
       já vêm com o python 2.6.

  * Instale o pip:
    `$ easy\_install pip`
  * Instale o virtualenv:
    `$ pip install virtualenv`
  * Selecione o diretório para a instalação do python e execute o virtualenv.
    `$ virtualenv --distributte venv`
  * Para executar o ambiente, rode:
    `$ source venv/bin/activate`
  * Para sair do ambiente:
    `$ deactivate`

4. Para instalar django no Windows:
  * Baixe [Download Django-1.7.1.tar.gz](https://www.djangoproject.com/download/1.7.1/tarball/). Então
       extraia o arquivo, inicie o DOS
 shell (ctrl+E, cmd) com permissão de administrador e execute o comando no diretório cujo nome inicie com
       "Django-":
    `$ python setup.py install`

5. Para instalar django no Ubuntu:
  * Pelo pip:
    `$ pip install Django==1.7`
  * "Manualmente":
    * Baixe [Download Django-1.7.1.tar.gz](https://www.djangoproject.com/download/1.7.1/tarball/). Então:
        `$ tar xzvf Django-1.6.2.tar.gz`
        `$ cd Django-1.6.2`
        `$ sudo python setup.py install`
� Online) (nome temporário)**
* **SALiE (nome final)**

Orientador:
**Professor Cláudio Correa e Castro Gonçalves**

Unidade Acadêmica / Departamento:
**Instituto de Letras/IL - Departamento de Línguas Estrangeiras e Tradução/LET**

Alunos Envolvidos:
* Yurick Hauschild Caetano da Costa 12/0024136
* Dayanne Fernandes 13/0107191
* Bruna Luisa xx/xxxxxxx

### 3. Documentação

Toda a documentação do projeto pode ser encontrada na pasta doc/.

> **ATENÇÃO**:
> A documentação acima referenciada inclui explanações das funções de
> todas as classes e métodos implementados e tem por público alvo
> principalmente desenvolvedores interessados em contribuir com o projeto.

> Para uma versão específica para usuários, ou seja, pessoas que estão interessadas no funcionamento da
       ferramenta e não em seus mecanismos, leia o ponto *4. Instalação e execução*

#### 3.i. Doxyfile (html)

 No intuito de simplificar a navegação dentro da documentação do projeto, utilizamos a ferramenta Doxygen,
       gerando, assim, um arquivo HTML que contém uma interface amigável, bem como um arquivo .tex capaz de gerar
       um pdf.
 Tanto o pdf quanto o html possuem as mesmas informações.
 Estes arquivos estão contidos em vendor e para acessá-los basta abrir o arquivo doc/html/index.html ou
       doc/latex/refman.pdf com o seu navegador ou visualizador de pdf, respectivamente.
 
### 4. Instalação e execução

Para executar o programa, siga as instruções abaixo.

1. Baixe o código fonte do projeto.
  Para isso, basta clicar no botão "Download ZIP" ao lado deste arquivo, como na imagem abaixo.  
![example1](http://i.imgur.com/kJtzWwf.jpg)

2. Para instalar o python no Windows:

 Obs: Não desenvolvemos a plataforma em Windows, e não temos o hábito de testar neste ambiente. Então, caro
       usuário da microsoft, na ocasião de encontrar algum bug ocasionado por incompatibilidade, sintá-se
       convidado a abrir uma issue e nos avisar, para que o corrijamos o mais rápido possível.
 [Siga este tutorial](http://docs.python-guide.org/en/latest/starting/install/win/).

3. Para instalar o python no Linux:
> OBS: As últimas versões do Ubuntu e Fedora já vêm com o python 2.7 e as últimas versões do RHEL e CentOS
       já vêm com o python 2.6.

  * Instale o pip:
    `$ easy\_install pip`
  * Instale o virtualenv:
    `$ pip install virtualenv`
  * Selecione o diretório para a instalação do python e execute o virtualenv.
    `$ virtualenv --distributte venv`
  * Para executar o ambiente, rode:
    `$ source venv/bin/activate`
  * Para sair do ambiente:
    `$ deactivate`

4. Para instalar django no Windows:
  * Baixe [Download Django-1.7.1.tar.gz](https://www.djangoproject.com/download/1.7.1/tarball/). Então
       extraia o arquivo, inicie o DOS
 shell (ctrl+E, cmd) com permissão de administrador e execute o comando no diretório cujo nome inicie com
       "Django-":
    `$ python setup.py install`

5. Para instalar django no Ubuntu:
  * Pelo pip:
    `$ pip install Django==1.7`
  * "Manualmente":
    * Baixe [Download Django-1.7.1.tar.gz](https://www.djangoproject.com/download/1.7.1/tarball/). Então:
        `$ tar xzvf Django-1.6.2.tar.gz`
        `$ cd Django-1.6.2`
        `$ sudo python setup.py install`
nguas Online) (nome tempor�
* **SALiE (nome final)**

Orientador:
**Professor Cláudio Correa e Castro Gonçalves**

Unidade Acadêmica / Departamento:
**Instituto de Letras/IL - Departamento de Línguas Estrangeiras e Tradução/LET**

Alunos Envolvidos:
* Yurick Hauschild Caetano da Costa 12/0024136
* Dayanne Fernandes 13/0107191
* Bruna Luisa xx/xxxxxxx

### 3. Documentação

Toda a documentação do projeto pode ser encontrada na pasta doc/.

> **ATENÇÃO**:
> A documentação acima referenciada inclui explanações das funções de
> todas as classes e métodos implementados e tem por público alvo
> principalmente desenvolvedores interessados em contribuir com o projeto.

> Para uma versão específica para usuários, ou seja, pessoas que estão interessadas no funcionamento da
       ferramenta e não em seus mecanismos, leia o ponto *4. Instalação e execução*

#### 3.i. Doxyfile (html)

 No intuito de simplificar a navegação dentro da documentação do projeto, utilizamos a ferramenta Doxygen,
       gerando, assim, um arquivo HTML que contém uma interface amigável, bem como um arquivo .tex capaz de gerar
       um pdf.
 Tanto o pdf quanto o html possuem as mesmas informações.
 Estes arquivos estão contidos em vendor e para acessá-los basta abrir o arquivo doc/html/index.html ou
       doc/latex/refman.pdf com o seu navegador ou visualizador de pdf, respectivamente.
 
### 4. Instalação e execução

Para executar o programa, siga as instruções abaixo.

1. Baixe o código fonte do projeto.
  Para isso, basta clicar no botão "Download ZIP" ao lado deste arquivo, como na imagem abaixo.  
![example1](http://i.imgur.com/kJtzWwf.jpg)

2. Para instalar o python no Windows:

 Obs: Não desenvolvemos a plataforma em Windows, e não temos o hábito de testar neste ambiente. Então, caro
       usuário da microsoft, na ocasião de encontrar algum bug ocasionado por incompatibilidade, sintá-se
       convidado a abrir uma issue e nos avisar, para que o corrijamos o mais rápido possível.
 [Siga este tutorial](http://docs.python-guide.org/en/latest/starting/install/win/).

3. Para instalar o python no Linux:
> OBS: As últimas versões do Ubuntu e Fedora já vêm com o python 2.7 e as últimas versões do RHEL e CentOS
       já vêm com o python 2.6.

  * Instale o pip:
    `$ easy\_install pip`
  * Instale o virtualenv:
    `$ pip install virtualenv`
  * Selecione o diretório para a instalação do python e execute o virtualenv.
    `$ virtualenv --distributte venv`
  * Para executar o ambiente, rode:
    `$ source venv/bin/activate`
  * Para sair do ambiente:
    `$ deactivate`

4. Para instalar django no Windows:
  * Baixe [Download Django-1.7.1.tar.gz](https://www.djangoproject.com/download/1.7.1/tarball/). Então
       extraia o arquivo, inicie o DOS
 shell (ctrl+E, cmd) com permissão de administrador e execute o comando no diretório cujo nome inicie com
       "Django-":
    `$ python setup.py install`

5. Para instalar django no Ubuntu:
  * Pelo pip:
    `$ pip install Django==1.7`
  * "Manualmente":
    * Baixe [Download Django-1.7.1.tar.gz](https://www.djangoproject.com/download/1.7.1/tarball/). Então:
        `$ tar xzvf Django-1.6.2.tar.gz`
        `$ cd Django-1.6.2`
        `$ sudo python setup.py install`
�
* **SALiE (nome final)**

Orientador:
**Professor Cláudio Correa e Castro Gonçalves**

Unidade Acadêmica / Departamento:
**Instituto de Letras/IL - Departamento de Línguas Estrangeiras e Tradução/LET**

Alunos Envolvidos:
* Yurick Hauschild Caetano da Costa 12/0024136
* Dayanne Fernandes 13/0107191
* Bruna Luisa xx/xxxxxxx

### 3. Documentação

Toda a documentação do projeto pode ser encontrada na pasta doc/.

> **ATENÇÃO**:
> A documentação acima referenciada inclui explanações das funções de
> todas as classes e métodos implementados e tem por público alvo
> principalmente desenvolvedores interessados em contribuir com o projeto.

> Para uma versão específica para usuários, ou seja, pessoas que estão interessadas no funcionamento da
       ferramenta e não em seus mecanismos, leia o ponto *4. Instalação e execução*

#### 3.i. Doxyfile (html)

 No intuito de simplificar a navegação dentro da documentação do projeto, utilizamos a ferramenta Doxygen,
       gerando, assim, um arquivo HTML que contém uma interface amigável, bem como um arquivo .tex capaz de gerar
       um pdf.
 Tanto o pdf quanto o html possuem as mesmas informações.
 Estes arquivos estão contidos em vendor e para acessá-los basta abrir o arquivo doc/html/index.html ou
       doc/latex/refman.pdf com o seu navegador ou visualizador de pdf, respectivamente.
 
### 4. Instalação e execução

Para executar o programa, siga as instruções abaixo.

1. Baixe o código fonte do projeto.
  Para isso, basta clicar no botão "Download ZIP" ao lado deste arquivo, como na imagem abaixo.  
![example1](http://i.imgur.com/kJtzWwf.jpg)

2. Para instalar o python no Windows:

 Obs: Não desenvolvemos a plataforma em Windows, e não temos o hábito de testar neste ambiente. Então, caro
       usuário da microsoft, na ocasião de encontrar algum bug ocasionado por incompatibilidade, sintá-se
       convidado a abrir uma issue e nos avisar, para que o corrijamos o mais rápido possível.
 [Siga este tutorial](http://docs.python-guide.org/en/latest/starting/install/win/).

3. Para instalar o python no Linux:
> OBS: As últimas versões do Ubuntu e Fedora já vêm com o python 2.7 e as últimas versões do RHEL e CentOS
       já vêm com o python 2.6.

  * Instale o pip:
    `$ easy\_install pip`
  * Instale o virtualenv:
    `$ pip install virtualenv`
  * Selecione o diretório para a instalação do python e execute o virtualenv.
    `$ virtualenv --distributte venv`
  * Para executar o ambiente, rode:
    `$ source venv/bin/activate`
  * Para sair do ambiente:
    `$ deactivate`

4. Para instalar django no Windows:
  * Baixe [Download Django-1.7.1.tar.gz](https://www.djangoproject.com/download/1.7.1/tarball/). Então
       extraia o arquivo, inicie o DOS
 shell (ctrl+E, cmd) com permissão de administrador e execute o comando no diretório cujo nome inicie com
       "Django-":
    `$ python setup.py install`

5. Para instalar django no Ubuntu:
  * Pelo pip:
    `$ pip install Django==1.7`
  * "Manualmente":
    * Baixe [Download Django-1.7.1.tar.gz](https://www.djangoproject.com/download/1.7.1/tarball/). Então:
        `$ tar xzvf Django-1.6.2.tar.gz`
        `$ cd Django-1.6.2`
        `$ sudo python setup.py install`
rio)**
00025 * **SALiE (nome final)**
00026 
00027 Orientador:
00028 **Professor Cláudio Correa e Castro Gonçalves**
00029 
00030 Unidade Acadêmica / Departamento:
00031 **Instituto de Letras/IL - Departamento de Línguas Estrangeiras e Tradução/LET**
00032 
00033 Alunos Envolvidos:
00034 * Yurick Hauschild Caetano da Costa 12/0024136
00035 * Dayanne Fernandes 13/0107191
00036 * Bruna Luisa xx/xxxxxxx
00037 
00038 \textcolor{preprocessor}{### 3. Documentação}
00039 \textcolor{preprocessor}{}
00040 Toda a documentação do projeto pode ser encontrada na pasta doc/.
00041 
00042 > **ATENÇÃO**:
00043 > A documentação acima referenciada inclui explanações das funções de
00044 > todas as classes e métodos implementados e tem por público alvo
00045 > principalmente desenvolvedores interessados em contribuir com o projeto.
00046 
00047 > Para uma versão específica para usuários, ou seja, pessoas que estão interessadas no funcionamento da 
      ferramenta e não em seus mecanismos, leia o ponto *4. Instalação e execução*
00048 
00049 \textcolor{preprocessor}{#### 3.i. Doxyfile (html)}
00050 \textcolor{preprocessor}{}
00051  No intuito de simplificar a navegação dentro da documentação do projeto, utilizamos a ferramenta Doxygen, 
      gerando, assim, um arquivo HTML que contém uma interface amigável, bem como um arquivo .tex capaz de gerar 
      um pdf.
00052  Tanto o pdf quanto o html possuem as mesmas informações.
00053  Estes arquivos estão contidos em vendor e para acessá-los basta abrir o arquivo doc/html/index.html ou doc
      /latex/refman.pdf com o seu navegador ou visualizador de pdf, respectivamente.
00054  
00055 \textcolor{preprocessor}{### 4. Instalação e execução}
00056 \textcolor{preprocessor}{}
00057 Para executar o programa, siga as instruções abaixo.
00058 
00059 1. Baixe o código fonte do projeto.
00060   Para isso, basta clicar no botão "Download ZIP" ao lado deste arquivo, como na imagem abaixo.  
00061 ![example1](http:\textcolor{comment}{//i.imgur.com/kJtzWwf.jpg)}
00062 
00063 2. Para instalar o python no Windows:
00064 
00065  Obs: N� desenvolvemos a plataforma em Windows, e não temos o hábito de testar neste ambiente. Então, caro
       usuário da microsoft, na ocasião de encontrar algum bug ocasionado por incompatibilidade, sintá-se
       convidado a abrir uma issue e nos avisar, para que o corrijamos o mais rápido possível.
 [Siga este tutorial](http://docs.python-guide.org/en/latest/starting/install/win/).

3. Para instalar o python no Linux:
> OBS: As últimas versões do Ubuntu e Fedora já vêm com o python 2.7 e as últimas versões do RHEL e CentOS
       já vêm com o python 2.6.

  * Instale o pip:
    `$ easy\_install pip`
  * Instale o virtualenv:
    `$ pip install virtualenv`
  * Selecione o diretório para a instalação do python e execute o virtualenv.
    `$ virtualenv --distributte venv`
  * Para executar o ambiente, rode:
    `$ source venv/bin/activate`
  * Para sair do ambiente:
    `$ deactivate`

4. Para instalar django no Windows:
  * Baixe [Download Django-1.7.1.tar.gz](https://www.djangoproject.com/download/1.7.1/tarball/). Então
       extraia o arquivo, inicie o DOS
 shell (ctrl+E, cmd) com permissão de administrador e execute o comando no diretório cujo nome inicie com
       "Django-":
    `$ python setup.py install`

5. Para instalar django no Ubuntu:
  * Pelo pip:
    `$ pip install Django==1.7`
  * "Manualmente":
    * Baixe [Download Django-1.7.1.tar.gz](https://www.djangoproject.com/download/1.7.1/tarball/). Então:
        `$ tar xzvf Django-1.6.2.tar.gz`
        `$ cd Django-1.6.2`
        `$ sudo python setup.py install`
� desenvolvemos a plataforma em Windows, e não temos o hábito de testar neste ambiente. Então, caro usuário
       da microsoft, na ocasião de encontrar algum bug ocasionado por incompatibilidade, sintá-se convidado a
       abrir uma issue e nos avisar, para que o corrijamos o mais rápido possível.
 [Siga este tutorial](http://docs.python-guide.org/en/latest/starting/install/win/).

3. Para instalar o python no Linux:
> OBS: As últimas versões do Ubuntu e Fedora já vêm com o python 2.7 e as últimas versões do RHEL e CentOS
       já vêm com o python 2.6.

  * Instale o pip:
    `$ easy\_install pip`
  * Instale o virtualenv:
    `$ pip install virtualenv`
  * Selecione o diretório para a instalação do python e execute o virtualenv.
    `$ virtualenv --distributte venv`
  * Para executar o ambiente, rode:
    `$ source venv/bin/activate`
  * Para sair do ambiente:
    `$ deactivate`

4. Para instalar django no Windows:
  * Baixe [Download Django-1.7.1.tar.gz](https://www.djangoproject.com/download/1.7.1/tarball/). Então
       extraia o arquivo, inicie o DOS
 shell (ctrl+E, cmd) com permissão de administrador e execute o comando no diretório cujo nome inicie com
       "Django-":
    `$ python setup.py install`

5. Para instalar django no Ubuntu:
  * Pelo pip:
    `$ pip install Django==1.7`
  * "Manualmente":
    * Baixe [Download Django-1.7.1.tar.gz](https://www.djangoproject.com/download/1.7.1/tarball/). Então:
        `$ tar xzvf Django-1.6.2.tar.gz`
        `$ cd Django-1.6.2`
        `$ sudo python setup.py install`
o desenvolvemos a plataforma em Windows, e n� temos o hábito de testar neste ambiente. Então, caro usuário
       da microsoft, na ocasião de encontrar algum bug ocasionado por incompatibilidade, sintá-se convidado a
       abrir uma issue e nos avisar, para que o corrijamos o mais rápido possível.
 [Siga este tutorial](http://docs.python-guide.org/en/latest/starting/install/win/).

3. Para instalar o python no Linux:
> OBS: As últimas versões do Ubuntu e Fedora já vêm com o python 2.7 e as últimas versões do RHEL e CentOS
       já vêm com o python 2.6.

  * Instale o pip:
    `$ easy\_install pip`
  * Instale o virtualenv:
    `$ pip install virtualenv`
  * Selecione o diretório para a instalação do python e execute o virtualenv.
    `$ virtualenv --distributte venv`
  * Para executar o ambiente, rode:
    `$ source venv/bin/activate`
  * Para sair do ambiente:
    `$ deactivate`

4. Para instalar django no Windows:
  * Baixe [Download Django-1.7.1.tar.gz](https://www.djangoproject.com/download/1.7.1/tarball/). Então
       extraia o arquivo, inicie o DOS
 shell (ctrl+E, cmd) com permissão de administrador e execute o comando no diretório cujo nome inicie com
       "Django-":
    `$ python setup.py install`

5. Para instalar django no Ubuntu:
  * Pelo pip:
    `$ pip install Django==1.7`
  * "Manualmente":
    * Baixe [Download Django-1.7.1.tar.gz](https://www.djangoproject.com/download/1.7.1/tarball/). Então:
        `$ tar xzvf Django-1.6.2.tar.gz`
        `$ cd Django-1.6.2`
        `$ sudo python setup.py install`
� temos o hábito de testar neste ambiente. Então, caro usuário da microsoft, na ocasião de encontrar algum
       bug ocasionado por incompatibilidade, sintá-se convidado a abrir uma issue e nos avisar, para que o
       corrijamos o mais rápido possível.
 [Siga este tutorial](http://docs.python-guide.org/en/latest/starting/install/win/).

3. Para instalar o python no Linux:
> OBS: As últimas versões do Ubuntu e Fedora já vêm com o python 2.7 e as últimas versões do RHEL e CentOS
       já vêm com o python 2.6.

  * Instale o pip:
    `$ easy\_install pip`
  * Instale o virtualenv:
    `$ pip install virtualenv`
  * Selecione o diretório para a instalação do python e execute o virtualenv.
    `$ virtualenv --distributte venv`
  * Para executar o ambiente, rode:
    `$ source venv/bin/activate`
  * Para sair do ambiente:
    `$ deactivate`

4. Para instalar django no Windows:
  * Baixe [Download Django-1.7.1.tar.gz](https://www.djangoproject.com/download/1.7.1/tarball/). Então
       extraia o arquivo, inicie o DOS
 shell (ctrl+E, cmd) com permissão de administrador e execute o comando no diretório cujo nome inicie com
       "Django-":
    `$ python setup.py install`

5. Para instalar django no Ubuntu:
  * Pelo pip:
    `$ pip install Django==1.7`
  * "Manualmente":
    * Baixe [Download Django-1.7.1.tar.gz](https://www.djangoproject.com/download/1.7.1/tarball/). Então:
        `$ tar xzvf Django-1.6.2.tar.gz`
        `$ cd Django-1.6.2`
        `$ sudo python setup.py install`
o temos o h� de testar neste ambiente. Então, caro usuário da microsoft, na ocasião de encontrar algum
       bug ocasionado por incompatibilidade, sintá-se convidado a abrir uma issue e nos avisar, para que o
       corrijamos o mais rápido possível.
 [Siga este tutorial](http://docs.python-guide.org/en/latest/starting/install/win/).

3. Para instalar o python no Linux:
> OBS: As últimas versões do Ubuntu e Fedora já vêm com o python 2.7 e as últimas versões do RHEL e CentOS
       já vêm com o python 2.6.

  * Instale o pip:
    `$ easy\_install pip`
  * Instale o virtualenv:
    `$ pip install virtualenv`
  * Selecione o diretório para a instalação do python e execute o virtualenv.
    `$ virtualenv --distributte venv`
  * Para executar o ambiente, rode:
    `$ source venv/bin/activate`
  * Para sair do ambiente:
    `$ deactivate`

4. Para instalar django no Windows:
  * Baixe [Download Django-1.7.1.tar.gz](https://www.djangoproject.com/download/1.7.1/tarball/). Então
       extraia o arquivo, inicie o DOS
 shell (ctrl+E, cmd) com permissão de administrador e execute o comando no diretório cujo nome inicie com
       "Django-":
    `$ python setup.py install`

5. Para instalar django no Ubuntu:
  * Pelo pip:
    `$ pip install Django==1.7`
  * "Manualmente":
    * Baixe [Download Django-1.7.1.tar.gz](https://www.djangoproject.com/download/1.7.1/tarball/). Então:
        `$ tar xzvf Django-1.6.2.tar.gz`
        `$ cd Django-1.6.2`
        `$ sudo python setup.py install`
� de testar neste ambiente. Então, caro usuário da microsoft, na ocasião de encontrar algum bug
       ocasionado por incompatibilidade, sintá-se convidado a abrir uma issue e nos avisar, para que o corrijamos o mais
       rápido possível.
 [Siga este tutorial](http://docs.python-guide.org/en/latest/starting/install/win/).

3. Para instalar o python no Linux:
> OBS: As últimas versões do Ubuntu e Fedora já vêm com o python 2.7 e as últimas versões do RHEL e CentOS
       já vêm com o python 2.6.

  * Instale o pip:
    `$ easy\_install pip`
  * Instale o virtualenv:
    `$ pip install virtualenv`
  * Selecione o diretório para a instalação do python e execute o virtualenv.
    `$ virtualenv --distributte venv`
  * Para executar o ambiente, rode:
    `$ source venv/bin/activate`
  * Para sair do ambiente:
    `$ deactivate`

4. Para instalar django no Windows:
  * Baixe [Download Django-1.7.1.tar.gz](https://www.djangoproject.com/download/1.7.1/tarball/). Então
       extraia o arquivo, inicie o DOS
 shell (ctrl+E, cmd) com permissão de administrador e execute o comando no diretório cujo nome inicie com
       "Django-":
    `$ python setup.py install`

5. Para instalar django no Ubuntu:
  * Pelo pip:
    `$ pip install Django==1.7`
  * "Manualmente":
    * Baixe [Download Django-1.7.1.tar.gz](https://www.djangoproject.com/download/1.7.1/tarball/). Então:
        `$ tar xzvf Django-1.6.2.tar.gz`
        `$ cd Django-1.6.2`
        `$ sudo python setup.py install`
bito de testar neste ambiente. Ent� caro usuário da microsoft, na ocasião de encontrar algum bug
       ocasionado por incompatibilidade, sintá-se convidado a abrir uma issue e nos avisar, para que o corrijamos o mais
       rápido possível.
 [Siga este tutorial](http://docs.python-guide.org/en/latest/starting/install/win/).

3. Para instalar o python no Linux:
> OBS: As últimas versões do Ubuntu e Fedora já vêm com o python 2.7 e as últimas versões do RHEL e CentOS
       já vêm com o python 2.6.

  * Instale o pip:
    `$ easy\_install pip`
  * Instale o virtualenv:
    `$ pip install virtualenv`
  * Selecione o diretório para a instalação do python e execute o virtualenv.
    `$ virtualenv --distributte venv`
  * Para executar o ambiente, rode:
    `$ source venv/bin/activate`
  * Para sair do ambiente:
    `$ deactivate`

4. Para instalar django no Windows:
  * Baixe [Download Django-1.7.1.tar.gz](https://www.djangoproject.com/download/1.7.1/tarball/). Então
       extraia o arquivo, inicie o DOS
 shell (ctrl+E, cmd) com permissão de administrador e execute o comando no diretório cujo nome inicie com
       "Django-":
    `$ python setup.py install`

5. Para instalar django no Ubuntu:
  * Pelo pip:
    `$ pip install Django==1.7`
  * "Manualmente":
    * Baixe [Download Django-1.7.1.tar.gz](https://www.djangoproject.com/download/1.7.1/tarball/). Então:
        `$ tar xzvf Django-1.6.2.tar.gz`
        `$ cd Django-1.6.2`
        `$ sudo python setup.py install`
� caro usuário da microsoft, na ocasião de encontrar algum bug ocasionado por incompatibilidade, sintá-se
       convidado a abrir uma issue e nos avisar, para que o corrijamos o mais rápido possível.
 [Siga este tutorial](http://docs.python-guide.org/en/latest/starting/install/win/).

3. Para instalar o python no Linux:
> OBS: As últimas versões do Ubuntu e Fedora já vêm com o python 2.7 e as últimas versões do RHEL e CentOS
       já vêm com o python 2.6.

  * Instale o pip:
    `$ easy\_install pip`
  * Instale o virtualenv:
    `$ pip install virtualenv`
  * Selecione o diretório para a instalação do python e execute o virtualenv.
    `$ virtualenv --distributte venv`
  * Para executar o ambiente, rode:
    `$ source venv/bin/activate`
  * Para sair do ambiente:
    `$ deactivate`

4. Para instalar django no Windows:
  * Baixe [Download Django-1.7.1.tar.gz](https://www.djangoproject.com/download/1.7.1/tarball/). Então
       extraia o arquivo, inicie o DOS
 shell (ctrl+E, cmd) com permissão de administrador e execute o comando no diretório cujo nome inicie com
       "Django-":
    `$ python setup.py install`

5. Para instalar django no Ubuntu:
  * Pelo pip:
    `$ pip install Django==1.7`
  * "Manualmente":
    * Baixe [Download Django-1.7.1.tar.gz](https://www.djangoproject.com/download/1.7.1/tarball/). Então:
        `$ tar xzvf Django-1.6.2.tar.gz`
        `$ cd Django-1.6.2`
        `$ sudo python setup.py install`
o, caro usu� da microsoft, na ocasião de encontrar algum bug ocasionado por incompatibilidade, sintá-se
       convidado a abrir uma issue e nos avisar, para que o corrijamos o mais rápido possível.
 [Siga este tutorial](http://docs.python-guide.org/en/latest/starting/install/win/).

3. Para instalar o python no Linux:
> OBS: As últimas versões do Ubuntu e Fedora já vêm com o python 2.7 e as últimas versões do RHEL e CentOS
       já vêm com o python 2.6.

  * Instale o pip:
    `$ easy\_install pip`
  * Instale o virtualenv:
    `$ pip install virtualenv`
  * Selecione o diretório para a instalação do python e execute o virtualenv.
    `$ virtualenv --distributte venv`
  * Para executar o ambiente, rode:
    `$ source venv/bin/activate`
  * Para sair do ambiente:
    `$ deactivate`

4. Para instalar django no Windows:
  * Baixe [Download Django-1.7.1.tar.gz](https://www.djangoproject.com/download/1.7.1/tarball/). Então
       extraia o arquivo, inicie o DOS
 shell (ctrl+E, cmd) com permissão de administrador e execute o comando no diretório cujo nome inicie com
       "Django-":
    `$ python setup.py install`

5. Para instalar django no Ubuntu:
  * Pelo pip:
    `$ pip install Django==1.7`
  * "Manualmente":
    * Baixe [Download Django-1.7.1.tar.gz](https://www.djangoproject.com/download/1.7.1/tarball/). Então:
        `$ tar xzvf Django-1.6.2.tar.gz`
        `$ cd Django-1.6.2`
        `$ sudo python setup.py install`
� da microsoft, na ocasião de encontrar algum bug ocasionado por incompatibilidade, sintá-se convidado a
       abrir uma issue e nos avisar, para que o corrijamos o mais rápido possível.
 [Siga este tutorial](http://docs.python-guide.org/en/latest/starting/install/win/).

3. Para instalar o python no Linux:
> OBS: As últimas versões do Ubuntu e Fedora já vêm com o python 2.7 e as últimas versões do RHEL e CentOS
       já vêm com o python 2.6.

  * Instale o pip:
    `$ easy\_install pip`
  * Instale o virtualenv:
    `$ pip install virtualenv`
  * Selecione o diretório para a instalação do python e execute o virtualenv.
    `$ virtualenv --distributte venv`
  * Para executar o ambiente, rode:
    `$ source venv/bin/activate`
  * Para sair do ambiente:
    `$ deactivate`

4. Para instalar django no Windows:
  * Baixe [Download Django-1.7.1.tar.gz](https://www.djangoproject.com/download/1.7.1/tarball/). Então
       extraia o arquivo, inicie o DOS
 shell (ctrl+E, cmd) com permissão de administrador e execute o comando no diretório cujo nome inicie com
       "Django-":
    `$ python setup.py install`

5. Para instalar django no Ubuntu:
  * Pelo pip:
    `$ pip install Django==1.7`
  * "Manualmente":
    * Baixe [Download Django-1.7.1.tar.gz](https://www.djangoproject.com/download/1.7.1/tarball/). Então:
        `$ tar xzvf Django-1.6.2.tar.gz`
        `$ cd Django-1.6.2`
        `$ sudo python setup.py install`
rio da microsoft, na ocasi� de encontrar algum bug ocasionado por incompatibilidade, sintá-se convidado a
       abrir uma issue e nos avisar, para que o corrijamos o mais rápido possível.
 [Siga este tutorial](http://docs.python-guide.org/en/latest/starting/install/win/).

3. Para instalar o python no Linux:
> OBS: As últimas versões do Ubuntu e Fedora já vêm com o python 2.7 e as últimas versões do RHEL e CentOS
       já vêm com o python 2.6.

  * Instale o pip:
    `$ easy\_install pip`
  * Instale o virtualenv:
    `$ pip install virtualenv`
  * Selecione o diretório para a instalação do python e execute o virtualenv.
    `$ virtualenv --distributte venv`
  * Para executar o ambiente, rode:
    `$ source venv/bin/activate`
  * Para sair do ambiente:
    `$ deactivate`

4. Para instalar django no Windows:
  * Baixe [Download Django-1.7.1.tar.gz](https://www.djangoproject.com/download/1.7.1/tarball/). Então
       extraia o arquivo, inicie o DOS
 shell (ctrl+E, cmd) com permissão de administrador e execute o comando no diretório cujo nome inicie com
       "Django-":
    `$ python setup.py install`

5. Para instalar django no Ubuntu:
  * Pelo pip:
    `$ pip install Django==1.7`
  * "Manualmente":
    * Baixe [Download Django-1.7.1.tar.gz](https://www.djangoproject.com/download/1.7.1/tarball/). Então:
        `$ tar xzvf Django-1.6.2.tar.gz`
        `$ cd Django-1.6.2`
        `$ sudo python setup.py install`
� de encontrar algum bug ocasionado por incompatibilidade, sintá-se convidado a abrir uma issue e nos
       avisar, para que o corrijamos o mais rápido possível.
 [Siga este tutorial](http://docs.python-guide.org/en/latest/starting/install/win/).

3. Para instalar o python no Linux:
> OBS: As últimas versões do Ubuntu e Fedora já vêm com o python 2.7 e as últimas versões do RHEL e CentOS
       já vêm com o python 2.6.

  * Instale o pip:
    `$ easy\_install pip`
  * Instale o virtualenv:
    `$ pip install virtualenv`
  * Selecione o diretório para a instalação do python e execute o virtualenv.
    `$ virtualenv --distributte venv`
  * Para executar o ambiente, rode:
    `$ source venv/bin/activate`
  * Para sair do ambiente:
    `$ deactivate`

4. Para instalar django no Windows:
  * Baixe [Download Django-1.7.1.tar.gz](https://www.djangoproject.com/download/1.7.1/tarball/). Então
       extraia o arquivo, inicie o DOS
 shell (ctrl+E, cmd) com permissão de administrador e execute o comando no diretório cujo nome inicie com
       "Django-":
    `$ python setup.py install`

5. Para instalar django no Ubuntu:
  * Pelo pip:
    `$ pip install Django==1.7`
  * "Manualmente":
    * Baixe [Download Django-1.7.1.tar.gz](https://www.djangoproject.com/download/1.7.1/tarball/). Então:
        `$ tar xzvf Django-1.6.2.tar.gz`
        `$ cd Django-1.6.2`
        `$ sudo python setup.py install`
o de encontrar algum bug ocasionado por incompatibilidade, sint� convidado a abrir uma issue e nos
       avisar, para que o corrijamos o mais rápido possível.
 [Siga este tutorial](http://docs.python-guide.org/en/latest/starting/install/win/).

3. Para instalar o python no Linux:
> OBS: As últimas versões do Ubuntu e Fedora já vêm com o python 2.7 e as últimas versões do RHEL e CentOS
       já vêm com o python 2.6.

  * Instale o pip:
    `$ easy\_install pip`
  * Instale o virtualenv:
    `$ pip install virtualenv`
  * Selecione o diretório para a instalação do python e execute o virtualenv.
    `$ virtualenv --distributte venv`
  * Para executar o ambiente, rode:
    `$ source venv/bin/activate`
  * Para sair do ambiente:
    `$ deactivate`

4. Para instalar django no Windows:
  * Baixe [Download Django-1.7.1.tar.gz](https://www.djangoproject.com/download/1.7.1/tarball/). Então
       extraia o arquivo, inicie o DOS
 shell (ctrl+E, cmd) com permissão de administrador e execute o comando no diretório cujo nome inicie com
       "Django-":
    `$ python setup.py install`

5. Para instalar django no Ubuntu:
  * Pelo pip:
    `$ pip install Django==1.7`
  * "Manualmente":
    * Baixe [Download Django-1.7.1.tar.gz](https://www.djangoproject.com/download/1.7.1/tarball/). Então:
        `$ tar xzvf Django-1.6.2.tar.gz`
        `$ cd Django-1.6.2`
        `$ sudo python setup.py install`
� convidado a abrir uma issue e nos avisar, para que o corrijamos o mais rápido possível.
 [Siga este tutorial](http://docs.python-guide.org/en/latest/starting/install/win/).

3. Para instalar o python no Linux:
> OBS: As últimas versões do Ubuntu e Fedora já vêm com o python 2.7 e as últimas versões do RHEL e CentOS
       já vêm com o python 2.6.

  * Instale o pip:
    `$ easy\_install pip`
  * Instale o virtualenv:
    `$ pip install virtualenv`
  * Selecione o diretório para a instalação do python e execute o virtualenv.
    `$ virtualenv --distributte venv`
  * Para executar o ambiente, rode:
    `$ source venv/bin/activate`
  * Para sair do ambiente:
    `$ deactivate`

4. Para instalar django no Windows:
  * Baixe [Download Django-1.7.1.tar.gz](https://www.djangoproject.com/download/1.7.1/tarball/). Então
       extraia o arquivo, inicie o DOS
 shell (ctrl+E, cmd) com permissão de administrador e execute o comando no diretório cujo nome inicie com
       "Django-":
    `$ python setup.py install`

5. Para instalar django no Ubuntu:
  * Pelo pip:
    `$ pip install Django==1.7`
  * "Manualmente":
    * Baixe [Download Django-1.7.1.tar.gz](https://www.djangoproject.com/download/1.7.1/tarball/). Então:
        `$ tar xzvf Django-1.6.2.tar.gz`
        `$ cd Django-1.6.2`
        `$ sudo python setup.py install`
-se convidado a abrir uma issue e nos avisar, para que o corrijamos o mais r� possível.
 [Siga este tutorial](http://docs.python-guide.org/en/latest/starting/install/win/).

3. Para instalar o python no Linux:
> OBS: As últimas versões do Ubuntu e Fedora já vêm com o python 2.7 e as últimas versões do RHEL e CentOS
       já vêm com o python 2.6.

  * Instale o pip:
    `$ easy\_install pip`
  * Instale o virtualenv:
    `$ pip install virtualenv`
  * Selecione o diretório para a instalação do python e execute o virtualenv.
    `$ virtualenv --distributte venv`
  * Para executar o ambiente, rode:
    `$ source venv/bin/activate`
  * Para sair do ambiente:
    `$ deactivate`

4. Para instalar django no Windows:
  * Baixe [Download Django-1.7.1.tar.gz](https://www.djangoproject.com/download/1.7.1/tarball/). Então
       extraia o arquivo, inicie o DOS
 shell (ctrl+E, cmd) com permissão de administrador e execute o comando no diretório cujo nome inicie com
       "Django-":
    `$ python setup.py install`

5. Para instalar django no Ubuntu:
  * Pelo pip:
    `$ pip install Django==1.7`
  * "Manualmente":
    * Baixe [Download Django-1.7.1.tar.gz](https://www.djangoproject.com/download/1.7.1/tarball/). Então:
        `$ tar xzvf Django-1.6.2.tar.gz`
        `$ cd Django-1.6.2`
        `$ sudo python setup.py install`
� possível.
 [Siga este tutorial](http://docs.python-guide.org/en/latest/starting/install/win/).

3. Para instalar o python no Linux:
> OBS: As últimas versões do Ubuntu e Fedora já vêm com o python 2.7 e as últimas versões do RHEL e CentOS
       já vêm com o python 2.6.

  * Instale o pip:
    `$ easy\_install pip`
  * Instale o virtualenv:
    `$ pip install virtualenv`
  * Selecione o diretório para a instalação do python e execute o virtualenv.
    `$ virtualenv --distributte venv`
  * Para executar o ambiente, rode:
    `$ source venv/bin/activate`
  * Para sair do ambiente:
    `$ deactivate`

4. Para instalar django no Windows:
  * Baixe [Download Django-1.7.1.tar.gz](https://www.djangoproject.com/download/1.7.1/tarball/). Então
       extraia o arquivo, inicie o DOS
 shell (ctrl+E, cmd) com permissão de administrador e execute o comando no diretório cujo nome inicie com
       "Django-":
    `$ python setup.py install`

5. Para instalar django no Ubuntu:
  * Pelo pip:
    `$ pip install Django==1.7`
  * "Manualmente":
    * Baixe [Download Django-1.7.1.tar.gz](https://www.djangoproject.com/download/1.7.1/tarball/). Então:
        `$ tar xzvf Django-1.6.2.tar.gz`
        `$ cd Django-1.6.2`
        `$ sudo python setup.py install`
pido poss�
 [Siga este tutorial](http://docs.python-guide.org/en/latest/starting/install/win/).

3. Para instalar o python no Linux:
> OBS: As últimas versões do Ubuntu e Fedora já vêm com o python 2.7 e as últimas versões do RHEL e CentOS
       já vêm com o python 2.6.

  * Instale o pip:
    `$ easy\_install pip`
  * Instale o virtualenv:
    `$ pip install virtualenv`
  * Selecione o diretório para a instalação do python e execute o virtualenv.
    `$ virtualenv --distributte venv`
  * Para executar o ambiente, rode:
    `$ source venv/bin/activate`
  * Para sair do ambiente:
    `$ deactivate`

4. Para instalar django no Windows:
  * Baixe [Download Django-1.7.1.tar.gz](https://www.djangoproject.com/download/1.7.1/tarball/). Então
       extraia o arquivo, inicie o DOS
 shell (ctrl+E, cmd) com permissão de administrador e execute o comando no diretório cujo nome inicie com
       "Django-":
    `$ python setup.py install`

5. Para instalar django no Ubuntu:
  * Pelo pip:
    `$ pip install Django==1.7`
  * "Manualmente":
    * Baixe [Download Django-1.7.1.tar.gz](https://www.djangoproject.com/download/1.7.1/tarball/). Então:
        `$ tar xzvf Django-1.6.2.tar.gz`
        `$ cd Django-1.6.2`
        `$ sudo python setup.py install`
�
 [Siga este tutorial](http://docs.python-guide.org/en/latest/starting/install/win/).

3. Para instalar o python no Linux:
> OBS: As últimas versões do Ubuntu e Fedora já vêm com o python 2.7 e as últimas versões do RHEL e CentOS
       já vêm com o python 2.6.

  * Instale o pip:
    `$ easy\_install pip`
  * Instale o virtualenv:
    `$ pip install virtualenv`
  * Selecione o diretório para a instalação do python e execute o virtualenv.
    `$ virtualenv --distributte venv`
  * Para executar o ambiente, rode:
    `$ source venv/bin/activate`
  * Para sair do ambiente:
    `$ deactivate`

4. Para instalar django no Windows:
  * Baixe [Download Django-1.7.1.tar.gz](https://www.djangoproject.com/download/1.7.1/tarball/). Então
       extraia o arquivo, inicie o DOS
 shell (ctrl+E, cmd) com permissão de administrador e execute o comando no diretório cujo nome inicie com
       "Django-":
    `$ python setup.py install`

5. Para instalar django no Ubuntu:
  * Pelo pip:
    `$ pip install Django==1.7`
  * "Manualmente":
    * Baixe [Download Django-1.7.1.tar.gz](https://www.djangoproject.com/download/1.7.1/tarball/). Então:
        `$ tar xzvf Django-1.6.2.tar.gz`
        `$ cd Django-1.6.2`
        `$ sudo python setup.py install`
vel.
00066  [Siga este tutorial](http:\textcolor{comment}{//docs.python-guide.org/en/latest/starting/install/win/).}
00067 
00068 3. Para instalar o python no Linux:
00069 > OBS: As � versões do Ubuntu e Fedora já vêm com o python 2.7 e as últimas versões do RHEL e CentOS
       já vêm com o python 2.6.

  * Instale o pip:
    `$ easy\_install pip`
  * Instale o virtualenv:
    `$ pip install virtualenv`
  * Selecione o diretório para a instalação do python e execute o virtualenv.
    `$ virtualenv --distributte venv`
  * Para executar o ambiente, rode:
    `$ source venv/bin/activate`
  * Para sair do ambiente:
    `$ deactivate`

4. Para instalar django no Windows:
  * Baixe [Download Django-1.7.1.tar.gz](https://www.djangoproject.com/download/1.7.1/tarball/). Então
       extraia o arquivo, inicie o DOS
 shell (ctrl+E, cmd) com permissão de administrador e execute o comando no diretório cujo nome inicie com
       "Django-":
    `$ python setup.py install`

5. Para instalar django no Ubuntu:
  * Pelo pip:
    `$ pip install Django==1.7`
  * "Manualmente":
    * Baixe [Download Django-1.7.1.tar.gz](https://www.djangoproject.com/download/1.7.1/tarball/). Então:
        `$ tar xzvf Django-1.6.2.tar.gz`
        `$ cd Django-1.6.2`
        `$ sudo python setup.py install`
� versões do Ubuntu e Fedora já vêm com o python 2.7 e as últimas versões do RHEL e CentOS já vêm com
       o python 2.6.

  * Instale o pip:
    `$ easy\_install pip`
  * Instale o virtualenv:
    `$ pip install virtualenv`
  * Selecione o diretório para a instalação do python e execute o virtualenv.
    `$ virtualenv --distributte venv`
  * Para executar o ambiente, rode:
    `$ source venv/bin/activate`
  * Para sair do ambiente:
    `$ deactivate`

4. Para instalar django no Windows:
  * Baixe [Download Django-1.7.1.tar.gz](https://www.djangoproject.com/download/1.7.1/tarball/). Então
       extraia o arquivo, inicie o DOS
 shell (ctrl+E, cmd) com permissão de administrador e execute o comando no diretório cujo nome inicie com
       "Django-":
    `$ python setup.py install`

5. Para instalar django no Ubuntu:
  * Pelo pip:
    `$ pip install Django==1.7`
  * "Manualmente":
    * Baixe [Download Django-1.7.1.tar.gz](https://www.djangoproject.com/download/1.7.1/tarball/). Então:
        `$ tar xzvf Django-1.6.2.tar.gz`
        `$ cd Django-1.6.2`
        `$ sudo python setup.py install`
ltimas vers� do Ubuntu e Fedora já vêm com o python 2.7 e as últimas versões do RHEL e CentOS já vêm com
       o python 2.6.

  * Instale o pip:
    `$ easy\_install pip`
  * Instale o virtualenv:
    `$ pip install virtualenv`
  * Selecione o diretório para a instalação do python e execute o virtualenv.
    `$ virtualenv --distributte venv`
  * Para executar o ambiente, rode:
    `$ source venv/bin/activate`
  * Para sair do ambiente:
    `$ deactivate`

4. Para instalar django no Windows:
  * Baixe [Download Django-1.7.1.tar.gz](https://www.djangoproject.com/download/1.7.1/tarball/). Então
       extraia o arquivo, inicie o DOS
 shell (ctrl+E, cmd) com permissão de administrador e execute o comando no diretório cujo nome inicie com
       "Django-":
    `$ python setup.py install`

5. Para instalar django no Ubuntu:
  * Pelo pip:
    `$ pip install Django==1.7`
  * "Manualmente":
    * Baixe [Download Django-1.7.1.tar.gz](https://www.djangoproject.com/download/1.7.1/tarball/). Então:
        `$ tar xzvf Django-1.6.2.tar.gz`
        `$ cd Django-1.6.2`
        `$ sudo python setup.py install`
� do Ubuntu e Fedora já vêm com o python 2.7 e as últimas versões do RHEL e CentOS já vêm com o python
       2.6.

  * Instale o pip:
    `$ easy\_install pip`
  * Instale o virtualenv:
    `$ pip install virtualenv`
  * Selecione o diretório para a instalação do python e execute o virtualenv.
    `$ virtualenv --distributte venv`
  * Para executar o ambiente, rode:
    `$ source venv/bin/activate`
  * Para sair do ambiente:
    `$ deactivate`

4. Para instalar django no Windows:
  * Baixe [Download Django-1.7.1.tar.gz](https://www.djangoproject.com/download/1.7.1/tarball/). Então
       extraia o arquivo, inicie o DOS
 shell (ctrl+E, cmd) com permissão de administrador e execute o comando no diretório cujo nome inicie com
       "Django-":
    `$ python setup.py install`

5. Para instalar django no Ubuntu:
  * Pelo pip:
    `$ pip install Django==1.7`
  * "Manualmente":
    * Baixe [Download Django-1.7.1.tar.gz](https://www.djangoproject.com/download/1.7.1/tarball/). Então:
        `$ tar xzvf Django-1.6.2.tar.gz`
        `$ cd Django-1.6.2`
        `$ sudo python setup.py install`
es do Ubuntu e Fedora j� vêm com o python 2.7 e as últimas versões do RHEL e CentOS já vêm com o python
       2.6.

  * Instale o pip:
    `$ easy\_install pip`
  * Instale o virtualenv:
    `$ pip install virtualenv`
  * Selecione o diretório para a instalação do python e execute o virtualenv.
    `$ virtualenv --distributte venv`
  * Para executar o ambiente, rode:
    `$ source venv/bin/activate`
  * Para sair do ambiente:
    `$ deactivate`

4. Para instalar django no Windows:
  * Baixe [Download Django-1.7.1.tar.gz](https://www.djangoproject.com/download/1.7.1/tarball/). Então
       extraia o arquivo, inicie o DOS
 shell (ctrl+E, cmd) com permissão de administrador e execute o comando no diretório cujo nome inicie com
       "Django-":
    `$ python setup.py install`

5. Para instalar django no Ubuntu:
  * Pelo pip:
    `$ pip install Django==1.7`
  * "Manualmente":
    * Baixe [Download Django-1.7.1.tar.gz](https://www.djangoproject.com/download/1.7.1/tarball/). Então:
        `$ tar xzvf Django-1.6.2.tar.gz`
        `$ cd Django-1.6.2`
        `$ sudo python setup.py install`
� com o python 2.7 e as últimas versões do RHEL e CentOS já vêm com o python 2.6.

  * Instale o pip:
    `$ easy\_install pip`
  * Instale o virtualenv:
    `$ pip install virtualenv`
  * Selecione o diretório para a instalação do python e execute o virtualenv.
    `$ virtualenv --distributte venv`
  * Para executar o ambiente, rode:
    `$ source venv/bin/activate`
  * Para sair do ambiente:
    `$ deactivate`

4. Para instalar django no Windows:
  * Baixe [Download Django-1.7.1.tar.gz](https://www.djangoproject.com/download/1.7.1/tarball/). Então
       extraia o arquivo, inicie o DOS
 shell (ctrl+E, cmd) com permissão de administrador e execute o comando no diretório cujo nome inicie com
       "Django-":
    `$ python setup.py install`

5. Para instalar django no Ubuntu:
  * Pelo pip:
    `$ pip install Django==1.7`
  * "Manualmente":
    * Baixe [Download Django-1.7.1.tar.gz](https://www.djangoproject.com/download/1.7.1/tarball/). Então:
        `$ tar xzvf Django-1.6.2.tar.gz`
        `$ cd Django-1.6.2`
        `$ sudo python setup.py install`
 v� com o python 2.7 e as últimas versões do RHEL e CentOS já vêm com o python 2.6.

  * Instale o pip:
    `$ easy\_install pip`
  * Instale o virtualenv:
    `$ pip install virtualenv`
  * Selecione o diretório para a instalação do python e execute o virtualenv.
    `$ virtualenv --distributte venv`
  * Para executar o ambiente, rode:
    `$ source venv/bin/activate`
  * Para sair do ambiente:
    `$ deactivate`

4. Para instalar django no Windows:
  * Baixe [Download Django-1.7.1.tar.gz](https://www.djangoproject.com/download/1.7.1/tarball/). Então
       extraia o arquivo, inicie o DOS
 shell (ctrl+E, cmd) com permissão de administrador e execute o comando no diretório cujo nome inicie com
       "Django-":
    `$ python setup.py install`

5. Para instalar django no Ubuntu:
  * Pelo pip:
    `$ pip install Django==1.7`
  * "Manualmente":
    * Baixe [Download Django-1.7.1.tar.gz](https://www.djangoproject.com/download/1.7.1/tarball/). Então:
        `$ tar xzvf Django-1.6.2.tar.gz`
        `$ cd Django-1.6.2`
        `$ sudo python setup.py install`
� com o python 2.7 e as últimas versões do RHEL e CentOS já vêm com o python 2.6.

  * Instale o pip:
    `$ easy\_install pip`
  * Instale o virtualenv:
    `$ pip install virtualenv`
  * Selecione o diretório para a instalação do python e execute o virtualenv.
    `$ virtualenv --distributte venv`
  * Para executar o ambiente, rode:
    `$ source venv/bin/activate`
  * Para sair do ambiente:
    `$ deactivate`

4. Para instalar django no Windows:
  * Baixe [Download Django-1.7.1.tar.gz](https://www.djangoproject.com/download/1.7.1/tarball/). Então
       extraia o arquivo, inicie o DOS
 shell (ctrl+E, cmd) com permissão de administrador e execute o comando no diretório cujo nome inicie com
       "Django-":
    `$ python setup.py install`

5. Para instalar django no Ubuntu:
  * Pelo pip:
    `$ pip install Django==1.7`
  * "Manualmente":
    * Baixe [Download Django-1.7.1.tar.gz](https://www.djangoproject.com/download/1.7.1/tarball/). Então:
        `$ tar xzvf Django-1.6.2.tar.gz`
        `$ cd Django-1.6.2`
        `$ sudo python setup.py install`
m com o python 2.7 e as � versões do RHEL e CentOS já vêm com o python 2.6.

  * Instale o pip:
    `$ easy\_install pip`
  * Instale o virtualenv:
    `$ pip install virtualenv`
  * Selecione o diretório para a instalação do python e execute o virtualenv.
    `$ virtualenv --distributte venv`
  * Para executar o ambiente, rode:
    `$ source venv/bin/activate`
  * Para sair do ambiente:
    `$ deactivate`

4. Para instalar django no Windows:
  * Baixe [Download Django-1.7.1.tar.gz](https://www.djangoproject.com/download/1.7.1/tarball/). Então
       extraia o arquivo, inicie o DOS
 shell (ctrl+E, cmd) com permissão de administrador e execute o comando no diretório cujo nome inicie com
       "Django-":
    `$ python setup.py install`

5. Para instalar django no Ubuntu:
  * Pelo pip:
    `$ pip install Django==1.7`
  * "Manualmente":
    * Baixe [Download Django-1.7.1.tar.gz](https://www.djangoproject.com/download/1.7.1/tarball/). Então:
        `$ tar xzvf Django-1.6.2.tar.gz`
        `$ cd Django-1.6.2`
        `$ sudo python setup.py install`
� versões do RHEL e CentOS já vêm com o python 2.6.

  * Instale o pip:
    `$ easy\_install pip`
  * Instale o virtualenv:
    `$ pip install virtualenv`
  * Selecione o diretório para a instalação do python e execute o virtualenv.
    `$ virtualenv --distributte venv`
  * Para executar o ambiente, rode:
    `$ source venv/bin/activate`
  * Para sair do ambiente:
    `$ deactivate`

4. Para instalar django no Windows:
  * Baixe [Download Django-1.7.1.tar.gz](https://www.djangoproject.com/download/1.7.1/tarball/). Então
       extraia o arquivo, inicie o DOS
 shell (ctrl+E, cmd) com permissão de administrador e execute o comando no diretório cujo nome inicie com
       "Django-":
    `$ python setup.py install`

5. Para instalar django no Ubuntu:
  * Pelo pip:
    `$ pip install Django==1.7`
  * "Manualmente":
    * Baixe [Download Django-1.7.1.tar.gz](https://www.djangoproject.com/download/1.7.1/tarball/). Então:
        `$ tar xzvf Django-1.6.2.tar.gz`
        `$ cd Django-1.6.2`
        `$ sudo python setup.py install`
ltimas vers� do RHEL e CentOS já vêm com o python 2.6.

  * Instale o pip:
    `$ easy\_install pip`
  * Instale o virtualenv:
    `$ pip install virtualenv`
  * Selecione o diretório para a instalação do python e execute o virtualenv.
    `$ virtualenv --distributte venv`
  * Para executar o ambiente, rode:
    `$ source venv/bin/activate`
  * Para sair do ambiente:
    `$ deactivate`

4. Para instalar django no Windows:
  * Baixe [Download Django-1.7.1.tar.gz](https://www.djangoproject.com/download/1.7.1/tarball/). Então
       extraia o arquivo, inicie o DOS
 shell (ctrl+E, cmd) com permissão de administrador e execute o comando no diretório cujo nome inicie com
       "Django-":
    `$ python setup.py install`

5. Para instalar django no Ubuntu:
  * Pelo pip:
    `$ pip install Django==1.7`
  * "Manualmente":
    * Baixe [Download Django-1.7.1.tar.gz](https://www.djangoproject.com/download/1.7.1/tarball/). Então:
        `$ tar xzvf Django-1.6.2.tar.gz`
        `$ cd Django-1.6.2`
        `$ sudo python setup.py install`
� do RHEL e CentOS já vêm com o python 2.6.

  * Instale o pip:
    `$ easy\_install pip`
  * Instale o virtualenv:
    `$ pip install virtualenv`
  * Selecione o diretório para a instalação do python e execute o virtualenv.
    `$ virtualenv --distributte venv`
  * Para executar o ambiente, rode:
    `$ source venv/bin/activate`
  * Para sair do ambiente:
    `$ deactivate`

4. Para instalar django no Windows:
  * Baixe [Download Django-1.7.1.tar.gz](https://www.djangoproject.com/download/1.7.1/tarball/). Então
       extraia o arquivo, inicie o DOS
 shell (ctrl+E, cmd) com permissão de administrador e execute o comando no diretório cujo nome inicie com
       "Django-":
    `$ python setup.py install`

5. Para instalar django no Ubuntu:
  * Pelo pip:
    `$ pip install Django==1.7`
  * "Manualmente":
    * Baixe [Download Django-1.7.1.tar.gz](https://www.djangoproject.com/download/1.7.1/tarball/). Então:
        `$ tar xzvf Django-1.6.2.tar.gz`
        `$ cd Django-1.6.2`
        `$ sudo python setup.py install`
es do RHEL e CentOS j� vêm com o python 2.6.

  * Instale o pip:
    `$ easy\_install pip`
  * Instale o virtualenv:
    `$ pip install virtualenv`
  * Selecione o diretório para a instalação do python e execute o virtualenv.
    `$ virtualenv --distributte venv`
  * Para executar o ambiente, rode:
    `$ source venv/bin/activate`
  * Para sair do ambiente:
    `$ deactivate`

4. Para instalar django no Windows:
  * Baixe [Download Django-1.7.1.tar.gz](https://www.djangoproject.com/download/1.7.1/tarball/). Então
       extraia o arquivo, inicie o DOS
 shell (ctrl+E, cmd) com permissão de administrador e execute o comando no diretório cujo nome inicie com
       "Django-":
    `$ python setup.py install`

5. Para instalar django no Ubuntu:
  * Pelo pip:
    `$ pip install Django==1.7`
  * "Manualmente":
    * Baixe [Download Django-1.7.1.tar.gz](https://www.djangoproject.com/download/1.7.1/tarball/). Então:
        `$ tar xzvf Django-1.6.2.tar.gz`
        `$ cd Django-1.6.2`
        `$ sudo python setup.py install`
� com o python 2.6.

  * Instale o pip:
    `$ easy\_install pip`
  * Instale o virtualenv:
    `$ pip install virtualenv`
  * Selecione o diretório para a instalação do python e execute o virtualenv.
    `$ virtualenv --distributte venv`
  * Para executar o ambiente, rode:
    `$ source venv/bin/activate`
  * Para sair do ambiente:
    `$ deactivate`

4. Para instalar django no Windows:
  * Baixe [Download Django-1.7.1.tar.gz](https://www.djangoproject.com/download/1.7.1/tarball/). Então
       extraia o arquivo, inicie o DOS
 shell (ctrl+E, cmd) com permissão de administrador e execute o comando no diretório cujo nome inicie com
       "Django-":
    `$ python setup.py install`

5. Para instalar django no Ubuntu:
  * Pelo pip:
    `$ pip install Django==1.7`
  * "Manualmente":
    * Baixe [Download Django-1.7.1.tar.gz](https://www.djangoproject.com/download/1.7.1/tarball/). Então:
        `$ tar xzvf Django-1.6.2.tar.gz`
        `$ cd Django-1.6.2`
        `$ sudo python setup.py install`
 v� com o python 2.6.

  * Instale o pip:
    `$ easy\_install pip`
  * Instale o virtualenv:
    `$ pip install virtualenv`
  * Selecione o diretório para a instalação do python e execute o virtualenv.
    `$ virtualenv --distributte venv`
  * Para executar o ambiente, rode:
    `$ source venv/bin/activate`
  * Para sair do ambiente:
    `$ deactivate`

4. Para instalar django no Windows:
  * Baixe [Download Django-1.7.1.tar.gz](https://www.djangoproject.com/download/1.7.1/tarball/). Então
       extraia o arquivo, inicie o DOS
 shell (ctrl+E, cmd) com permissão de administrador e execute o comando no diretório cujo nome inicie com
       "Django-":
    `$ python setup.py install`

5. Para instalar django no Ubuntu:
  * Pelo pip:
    `$ pip install Django==1.7`
  * "Manualmente":
    * Baixe [Download Django-1.7.1.tar.gz](https://www.djangoproject.com/download/1.7.1/tarball/). Então:
        `$ tar xzvf Django-1.6.2.tar.gz`
        `$ cd Django-1.6.2`
        `$ sudo python setup.py install`
� com o python 2.6.

  * Instale o pip:
    `$ easy\_install pip`
  * Instale o virtualenv:
    `$ pip install virtualenv`
  * Selecione o diretório para a instalação do python e execute o virtualenv.
    `$ virtualenv --distributte venv`
  * Para executar o ambiente, rode:
    `$ source venv/bin/activate`
  * Para sair do ambiente:
    `$ deactivate`

4. Para instalar django no Windows:
  * Baixe [Download Django-1.7.1.tar.gz](https://www.djangoproject.com/download/1.7.1/tarball/). Então
       extraia o arquivo, inicie o DOS
 shell (ctrl+E, cmd) com permissão de administrador e execute o comando no diretório cujo nome inicie com
       "Django-":
    `$ python setup.py install`

5. Para instalar django no Ubuntu:
  * Pelo pip:
    `$ pip install Django==1.7`
  * "Manualmente":
    * Baixe [Download Django-1.7.1.tar.gz](https://www.djangoproject.com/download/1.7.1/tarball/). Então:
        `$ tar xzvf Django-1.6.2.tar.gz`
        `$ cd Django-1.6.2`
        `$ sudo python setup.py install`
m com o python 2.6.
00070 
00071   * Instale o pip:
00072     `$ easy\_install pip`
00073   * Instale o virtualenv:
00074     `$ pip install virtualenv`
00075   * Selecione o diret� para a instalação do python e execute o virtualenv.
    `$ virtualenv --distributte venv`
  * Para executar o ambiente, rode:
    `$ source venv/bin/activate`
  * Para sair do ambiente:
    `$ deactivate`

4. Para instalar django no Windows:
  * Baixe [Download Django-1.7.1.tar.gz](https://www.djangoproject.com/download/1.7.1/tarball/). Então
       extraia o arquivo, inicie o DOS
 shell (ctrl+E, cmd) com permissão de administrador e execute o comando no diretório cujo nome inicie com
       "Django-":
    `$ python setup.py install`

5. Para instalar django no Ubuntu:
  * Pelo pip:
    `$ pip install Django==1.7`
  * "Manualmente":
    * Baixe [Download Django-1.7.1.tar.gz](https://www.djangoproject.com/download/1.7.1/tarball/). Então:
        `$ tar xzvf Django-1.6.2.tar.gz`
        `$ cd Django-1.6.2`
        `$ sudo python setup.py install`
� para a instalação do python e execute o virtualenv.
    `$ virtualenv --distributte venv`
  * Para executar o ambiente, rode:
    `$ source venv/bin/activate`
  * Para sair do ambiente:
    `$ deactivate`

4. Para instalar django no Windows:
  * Baixe [Download Django-1.7.1.tar.gz](https://www.djangoproject.com/download/1.7.1/tarball/). Então
       extraia o arquivo, inicie o DOS
 shell (ctrl+E, cmd) com permissão de administrador e execute o comando no diretório cujo nome inicie com
       "Django-":
    `$ python setup.py install`

5. Para instalar django no Ubuntu:
  * Pelo pip:
    `$ pip install Django==1.7`
  * "Manualmente":
    * Baixe [Download Django-1.7.1.tar.gz](https://www.djangoproject.com/download/1.7.1/tarball/). Então:
        `$ tar xzvf Django-1.6.2.tar.gz`
        `$ cd Django-1.6.2`
        `$ sudo python setup.py install`
rio para a instala� do python e execute o virtualenv.
    `$ virtualenv --distributte venv`
  * Para executar o ambiente, rode:
    `$ source venv/bin/activate`
  * Para sair do ambiente:
    `$ deactivate`

4. Para instalar django no Windows:
  * Baixe [Download Django-1.7.1.tar.gz](https://www.djangoproject.com/download/1.7.1/tarball/). Então
       extraia o arquivo, inicie o DOS
 shell (ctrl+E, cmd) com permissão de administrador e execute o comando no diretório cujo nome inicie com
       "Django-":
    `$ python setup.py install`

5. Para instalar django no Ubuntu:
  * Pelo pip:
    `$ pip install Django==1.7`
  * "Manualmente":
    * Baixe [Download Django-1.7.1.tar.gz](https://www.djangoproject.com/download/1.7.1/tarball/). Então:
        `$ tar xzvf Django-1.6.2.tar.gz`
        `$ cd Django-1.6.2`
        `$ sudo python setup.py install`
� do python e execute o virtualenv.
    `$ virtualenv --distributte venv`
  * Para executar o ambiente, rode:
    `$ source venv/bin/activate`
  * Para sair do ambiente:
    `$ deactivate`

4. Para instalar django no Windows:
  * Baixe [Download Django-1.7.1.tar.gz](https://www.djangoproject.com/download/1.7.1/tarball/). Então
       extraia o arquivo, inicie o DOS
 shell (ctrl+E, cmd) com permissão de administrador e execute o comando no diretório cujo nome inicie com
       "Django-":
    `$ python setup.py install`

5. Para instalar django no Ubuntu:
  * Pelo pip:
    `$ pip install Django==1.7`
  * "Manualmente":
    * Baixe [Download Django-1.7.1.tar.gz](https://www.djangoproject.com/download/1.7.1/tarball/). Então:
        `$ tar xzvf Django-1.6.2.tar.gz`
        `$ cd Django-1.6.2`
        `$ sudo python setup.py install`
� do python e execute o virtualenv.
    `$ virtualenv --distributte venv`
  * Para executar o ambiente, rode:
    `$ source venv/bin/activate`
  * Para sair do ambiente:
    `$ deactivate`

4. Para instalar django no Windows:
  * Baixe [Download Django-1.7.1.tar.gz](https://www.djangoproject.com/download/1.7.1/tarball/). Então
       extraia o arquivo, inicie o DOS
 shell (ctrl+E, cmd) com permissão de administrador e execute o comando no diretório cujo nome inicie com
       "Django-":
    `$ python setup.py install`

5. Para instalar django no Ubuntu:
  * Pelo pip:
    `$ pip install Django==1.7`
  * "Manualmente":
    * Baixe [Download Django-1.7.1.tar.gz](https://www.djangoproject.com/download/1.7.1/tarball/). Então:
        `$ tar xzvf Django-1.6.2.tar.gz`
        `$ cd Django-1.6.2`
        `$ sudo python setup.py install`
� do python e execute o virtualenv.
    `$ virtualenv --distributte venv`
  * Para executar o ambiente, rode:
    `$ source venv/bin/activate`
  * Para sair do ambiente:
    `$ deactivate`

4. Para instalar django no Windows:
  * Baixe [Download Django-1.7.1.tar.gz](https://www.djangoproject.com/download/1.7.1/tarball/). Então
       extraia o arquivo, inicie o DOS
 shell (ctrl+E, cmd) com permissão de administrador e execute o comando no diretório cujo nome inicie com
       "Django-":
    `$ python setup.py install`

5. Para instalar django no Ubuntu:
  * Pelo pip:
    `$ pip install Django==1.7`
  * "Manualmente":
    * Baixe [Download Django-1.7.1.tar.gz](https://www.djangoproject.com/download/1.7.1/tarball/). Então:
        `$ tar xzvf Django-1.6.2.tar.gz`
        `$ cd Django-1.6.2`
        `$ sudo python setup.py install`
o do python e execute o virtualenv.
00076     `$ virtualenv --distributte venv`
00077   * Para executar o ambiente, rode:
00078     `$ source venv/bin/activate`
00079   * Para sair do ambiente:
00080     `$ deactivate`
00081 
00082 4. Para instalar django no Windows:
00083   * Baixe [Download Django-1.7.1.tar.gz](https:\textcolor{comment}{//www.djangoproject.com/download/1.7.1/tarball/). Então
       extraia o arquivo, inicie o DOS}
00084  shell (ctrl+E, cmd) com permiss� de administrador e execute o comando no diretório cujo nome inicie com
       "Django-":
    `$ python setup.py install`

5. Para instalar django no Ubuntu:
  * Pelo pip:
    `$ pip install Django==1.7`
  * "Manualmente":
    * Baixe [Download Django-1.7.1.tar.gz](https://www.djangoproject.com/download/1.7.1/tarball/). Então:
        `$ tar xzvf Django-1.6.2.tar.gz`
        `$ cd Django-1.6.2`
        `$ sudo python setup.py install`
� de administrador e execute o comando no diretório cujo nome inicie com "Django-":
    `$ python setup.py install`

5. Para instalar django no Ubuntu:
  * Pelo pip:
    `$ pip install Django==1.7`
  * "Manualmente":
    * Baixe [Download Django-1.7.1.tar.gz](https://www.djangoproject.com/download/1.7.1/tarball/). Então:
        `$ tar xzvf Django-1.6.2.tar.gz`
        `$ cd Django-1.6.2`
        `$ sudo python setup.py install`
o de administrador e execute o comando no diret� cujo nome inicie com "Django-":
    `$ python setup.py install`

5. Para instalar django no Ubuntu:
  * Pelo pip:
    `$ pip install Django==1.7`
  * "Manualmente":
    * Baixe [Download Django-1.7.1.tar.gz](https://www.djangoproject.com/download/1.7.1/tarball/). Então:
        `$ tar xzvf Django-1.6.2.tar.gz`
        `$ cd Django-1.6.2`
        `$ sudo python setup.py install`
� cujo nome inicie com "Django-":
    `$ python setup.py install`

5. Para instalar django no Ubuntu:
  * Pelo pip:
    `$ pip install Django==1.7`
  * "Manualmente":
    * Baixe [Download Django-1.7.1.tar.gz](https://www.djangoproject.com/download/1.7.1/tarball/). Então:
        `$ tar xzvf Django-1.6.2.tar.gz`
        `$ cd Django-1.6.2`
        `$ sudo python setup.py install`
rio cujo nome inicie com "Django-":
00085     `$ python setup.py install`
00086 
00087 5. Para instalar django no Ubuntu:
00088   * Pelo pip:
00089     `$ pip install Django==1.7`
00090   * "Manualmente":
00091     * Baixe [Download Django-1.7.1.tar.gz](https:\textcolor{comment}{//www.djangoproject.com/download/1.7.1/tarball/). Então:}
00092         `$ tar xzvf Django-1.6.2.tar.gz`
00093         `$ cd Django-1.6.2`
00094         `$ sudo python setup.py install`
\end{DoxyCode}
