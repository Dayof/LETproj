\hypertarget{README_8md_source}{}\subsection{R\+E\+A\+D\+M\+E.\+md}

\begin{DoxyCode}
00001 # LETproject
00002 
00003 ## Index
00004 
00005 1. Definição
00006 2. Dados do projeto
00007 3. Documentação
00008   3.1. Doxygen (html)
00009 4. Instalação e execução
00010 5. Metas de desenvolvimento
00011 6. Log de Atividades
00012 
00013 ## Conteúdo
00014 
00015 ### 1. Definição
00016 
00017 O LETproject é uma iniciativa do laboratório do LET para o desenvolvimento de uma plataforma de ensino
       de línguas online.
00018 Seu principal objetivo é criar uma interface amigável entre alunos e professores de forma a incluir o
       estudo doméstico de línguas no mundo informatizado.
00019 
00020 ### 2. Dados do projeto
00021 
00022 Título do projeto:
00023 * **LETproject (nome temporário)**
00024 * **ELO (Ensino de Línguas Online) (nome temporário)**
00025 * **SALiE (nome final)**
00026 
00027 Orientador:
00028 **Professor Cláudio Correa e Castro Gonçalves**
00029 
00030 Unidade Acadêmica / Departamento:
00031 **Instituto de Letras/IL - Departamento de Línguas Estrangeiras e Tradução/LET**
00032 
00033 Alunos Envolvidos:
00034 * Yurick Hauschild Caetano da Costa 12/0024136
00035 * Dayanne Fernandes 13/0107191
00036 * Bruna Luisa xx/xxxxxxx
00037 
00038 ### 3. Documentação
00039 
00040 Toda a documentação do projeto pode ser encontrada na pasta doc/.
00041 
00042 > **ATENÇÃO**:
00043 > A documentação acima referenciada inclui explanações das funções de
00044 > todas as classes e métodos implementados e tem por público alvo
00045 > principalmente desenvolvedores interessados em contribuir com o projeto.
00046 
00047 > Para uma versão específica para usuários, ou seja, pessoas que estão interessadas no funcionamento
       da ferramenta e não em seus mecanismos, leia o ponto *4. Instalação e execução*
00048 
00049 #### 3.i. Doxyfile (html)
00050 
00051  No intuito de simplificar a navegação dentro da documentação do projeto, utilizamos a ferramenta
       Doxygen, gerando, assim, um arquivo HTML que contém uma interface amigável, bem como um arquivo .tex capaz de
       gerar um pdf.
00052  Tanto o pdf quanto o html possuem as mesmas informações.
00053  Estes arquivos estão contidos em vendor e para acessá-los basta abrir o arquivo doc/html/index.html
       ou doc/latex/refman.pdf com o seu navegador ou visualizador de pdf, respectivamente.
00054  
00055 ### 4. Instalação e execução
00056 
00057 Para executar o programa, siga as instruções abaixo.
00058 
00059 1. Baixe o código fonte do projeto.
00060   Para isso, basta clicar no botão "Download ZIP" ao lado deste arquivo, como na imagem abaixo.  
00061 ![example1](http://i.imgur.com/kJtzWwf.jpg)
00062 
00063 2. Para instalar o python no Windows:
00064 
00065  Obs: Não desenvolvemos a plataforma em Windows, e não temos o hábito de testar neste ambiente. Então,
       caro usuário da microsoft, na ocasião de encontrar algum bug ocasionado por incompatibilidade, sintá-se
       convidado a abrir uma issue e nos avisar, para que o corrijamos o mais rápido possível.
00066  [Siga este tutorial](http://docs.python-guide.org/en/latest/starting/install/win/).
00067 
00068 3. Para instalar o python no Linux:
00069 > OBS: As últimas versões do Ubuntu e Fedora já vêm com o python 2.7 e as últimas versões do RHEL e
       CentOS já vêm com o python 2.6.
00070 
00071   * Instale o pip:
00072     `$ easy\_install pip`
00073   * Instale o virtualenv:
00074    `$ pip install virtualenv`
00075   * Selecione o diretório para a instalação do python e execute o virtualenv.
00076    `$ virtualenv --distributte venv`
00077   * Para executar o ambiente, rode:
00078    `$ source venv/bin/activate`
00079   * Para sair do ambiente:
00080    `$ deactivate`
00081 
00082 4. Para instalar django no Windows:
00083   * Baixe [Download Django-1.7.1.tar.gz](https://www.djangoproject.com/download/1.7.1/tarball/). Então
       extraia o arquivo, inicie o DOS
00084  shell (ctrl+E, cmd) com permissão de administrador e execute o comando no diretório cujo nome inicie
       com "Django-":
00085    `$ python setup.py install`
00086 
00087 5. Para instalar django no Ubuntu:
00088   * Pelo pip:
00089    `$ pip install Django==1.7`
00090   * "Manualmente":
00091    * Baixe [Download Django-1.7.1.tar.gz](https://www.djangoproject.com/download/1.7.1/tarball/).
       Então:
00092        `$ tar xzvf Django-1.6.2.tar.gz`
00093        `$ cd Django-1.6.2`
00094        `$ sudo python setup.py install`
\end{DoxyCode}
