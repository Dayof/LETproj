\hypertarget{README_8md_source}{\subsection{R\-E\-A\-D\-M\-E.\-md}
}

\begin{DoxyCode}
00001 \textcolor{preprocessor}{# LETproject  \{#index\}}
00002 \textcolor{preprocessor}{}
00003 \textcolor{preprocessor}{## Conteúdo}
00004 \textcolor{preprocessor}{}
00005 [TOC]
00006 
00007 \textcolor{preprocessor}{## Index}
00008 \textcolor{preprocessor}{}
00009 1. Definição
00010 2. Dados \textcolor{keywordflow}{do} projeto
00011 3. Documentação
00012   3.1. Doxygen (html)
00013 4. Instalação e execução
00014 5. Metas de desenvolvimento
00015 6. Log de Atividades
00016 
00017 \textcolor{preprocessor}{## Conteúdo}
00018 \textcolor{preprocessor}{}
00019 \textcolor{preprocessor}{### 1. Definição}
00020 \textcolor{preprocessor}{}
00021 O LETproject é uma iniciativa \textcolor{keywordflow}{do} laboratório \textcolor{keywordflow}{do} LET para o desenvolvimento de uma plataforma de ensino de 
      línguas online.
00022 Seu principal objetivo é criar uma \textcolor{keyword}{interface }amigável entre alunos e professores de forma a incluir o 
      estudo doméstico de línguas no mundo informatizado.
00023 
00024 ### 2. Dados do projeto
00025 
00026 Título do projeto:
00027 * **LETproject (nome tempor�
* **ELO (Ensino de Línguas Online) (nome temporário)**

Orientador:
**Professor Cláudio Correa e Castro Gonçalves**

Unidade Acadêmica / Departamento:
**Instituto de Letras/IL - Departamento de Línguas Estrangeiras e Tradução/LET**

Alunos Envolvidos:
* Yurick Hauschild Caetano da Costa 12/0024136
* André Accioly Lima 12/0059908
* Diego Santos da Silva 11/0027892

### 3. Documentação

Toda a documentação do projeto pode ser encontrada na pasta doc/.

> **ATENÇÃO**:
> A documentação acima referenciada inclui explanações das funções de
> todas as classes e métodos implementados e tem por público alvo
> principalmente desenvolvedores interessados em contribuir com o projeto.

> Para uma versão específica para usuários, ou seja, pessoas que estão interessadas no funcionamento da
       ferramenta e não em seus mecanismos, leia o ponto *4. Instalação e execução*

#### 3.i. Doxyfile (html)

 No intuito de simplificar a navegação dentro da documentação do projeto, utilizamos a ferramenta Doxygen,
       gerando, assim, um arquivo HTML que contém uma interface amigável.
 Este arquivo está contido em doc/html/ e para acessá-lo basta abrir o arquivo doc/html/index.html com o
       seu navegador.
 
### 4. Instalação e execução

Para executar o programa, siga as instruções abaixo.

1. Baixe o código fonte do projeto.
  Para isso, basta clicar no botão "Download ZIP" ao lado deste arquivo, como na imagem abaixo.  
![example1](http://i.imgur.com/kJtzWwf.jpg)

2. Para instalar o python no Windows:


3. Para instalar o python no Linux:
> OBS: As últimas versões do Ubuntu e Fedora já vêm com o python 2.7 e as últimas versões do RHEL e CentOS
       já vêm com o python 2.6.

[BAIXE O PYTHON SCRIPT AQUI, E EXECUTE](http://python-distribute.org/distribute\_setup.py). Para executar,
       use no terminal `python distributte\_setup.py`
  * Instale o pip:
    `$ easy\_install pip`
  * Instale o virtualenv:
    `$ pip install virtualenv`
  * Selecione o diretório para a instalação do python e execute o virtualenv.
    `$ virtualenv --distributte venv`
  * Para executar o ambiente, rode:
    `$ source venv/bin/activate`
  * Para sair do ambiente:
    `$ deactivate`

4. Para instalar django no Windows:
  * Baixe [Download Django-1.6.2.tar.gz](https://www.djangoproject.com/download/1.6.2/tarball/). Então
       extraia o arquivo, inicie o DOS
 shell com permissão de administrador e execute o comando no diretório cujo nome inicie com "Django-":
    `$ python setup.py install`

5. Para instalar django no Ubuntu:
  * Pelo pip:
    `$ pip install Django==1.6.2`
  * "Manualmente":
    * Baixe [Download Django-1.6.2.tar.gz](https://www.djangoproject.com/download/1.6.2/tarball/). Então:
        `$ tar xzvf Django-1.6.2.tar.gz`
        `$ cd Django-1.6.2`
        `$ sudo python setup.py install`�
* **ELO (Ensino de Línguas Online) (nome temporário)**

Orientador:
**Professor Cláudio Correa e Castro Gonçalves**

Unidade Acadêmica / Departamento:
**Instituto de Letras/IL - Departamento de Línguas Estrangeiras e Tradução/LET**

Alunos Envolvidos:
* Yurick Hauschild Caetano da Costa 12/0024136
* André Accioly Lima 12/0059908
* Diego Santos da Silva 11/0027892

### 3. Documentação

Toda a documentação do projeto pode ser encontrada na pasta doc/.

> **ATENÇÃO**:
> A documentação acima referenciada inclui explanações das funções de
> todas as classes e métodos implementados e tem por público alvo
> principalmente desenvolvedores interessados em contribuir com o projeto.

> Para uma versão específica para usuários, ou seja, pessoas que estão interessadas no funcionamento da
       ferramenta e não em seus mecanismos, leia o ponto *4. Instalação e execução*

#### 3.i. Doxyfile (html)

 No intuito de simplificar a navegação dentro da documentação do projeto, utilizamos a ferramenta Doxygen,
       gerando, assim, um arquivo HTML que contém uma interface amigável.
 Este arquivo está contido em doc/html/ e para acessá-lo basta abrir o arquivo doc/html/index.html com o
       seu navegador.
 
### 4. Instalação e execução

Para executar o programa, siga as instruções abaixo.

1. Baixe o código fonte do projeto.
  Para isso, basta clicar no botão "Download ZIP" ao lado deste arquivo, como na imagem abaixo.  
![example1](http://i.imgur.com/kJtzWwf.jpg)

2. Para instalar o python no Windows:


3. Para instalar o python no Linux:
> OBS: As últimas versões do Ubuntu e Fedora já vêm com o python 2.7 e as últimas versões do RHEL e CentOS
       já vêm com o python 2.6.

[BAIXE O PYTHON SCRIPT AQUI, E EXECUTE](http://python-distribute.org/distribute\_setup.py). Para executar,
       use no terminal `python distributte\_setup.py`
  * Instale o pip:
    `$ easy\_install pip`
  * Instale o virtualenv:
    `$ pip install virtualenv`
  * Selecione o diretório para a instalação do python e execute o virtualenv.
    `$ virtualenv --distributte venv`
  * Para executar o ambiente, rode:
    `$ source venv/bin/activate`
  * Para sair do ambiente:
    `$ deactivate`

4. Para instalar django no Windows:
  * Baixe [Download Django-1.6.2.tar.gz](https://www.djangoproject.com/download/1.6.2/tarball/). Então
       extraia o arquivo, inicie o DOS
 shell com permissão de administrador e execute o comando no diretório cujo nome inicie com "Django-":
    `$ python setup.py install`

5. Para instalar django no Ubuntu:
  * Pelo pip:
    `$ pip install Django==1.6.2`
  * "Manualmente":
    * Baixe [Download Django-1.6.2.tar.gz](https://www.djangoproject.com/download/1.6.2/tarball/). Então:
        `$ tar xzvf Django-1.6.2.tar.gz`
        `$ cd Django-1.6.2`
        `$ sudo python setup.py install`rio)**
00028 * **ELO (Ensino de L� Online) (nome temporário)**

Orientador:
**Professor Cláudio Correa e Castro Gonçalves**

Unidade Acadêmica / Departamento:
**Instituto de Letras/IL - Departamento de Línguas Estrangeiras e Tradução/LET**

Alunos Envolvidos:
* Yurick Hauschild Caetano da Costa 12/0024136
* André Accioly Lima 12/0059908
* Diego Santos da Silva 11/0027892

### 3. Documentação

Toda a documentação do projeto pode ser encontrada na pasta doc/.

> **ATENÇÃO**:
> A documentação acima referenciada inclui explanações das funções de
> todas as classes e métodos implementados e tem por público alvo
> principalmente desenvolvedores interessados em contribuir com o projeto.

> Para uma versão específica para usuários, ou seja, pessoas que estão interessadas no funcionamento da
       ferramenta e não em seus mecanismos, leia o ponto *4. Instalação e execução*

#### 3.i. Doxyfile (html)

 No intuito de simplificar a navegação dentro da documentação do projeto, utilizamos a ferramenta Doxygen,
       gerando, assim, um arquivo HTML que contém uma interface amigável.
 Este arquivo está contido em doc/html/ e para acessá-lo basta abrir o arquivo doc/html/index.html com o
       seu navegador.
 
### 4. Instalação e execução

Para executar o programa, siga as instruções abaixo.

1. Baixe o código fonte do projeto.
  Para isso, basta clicar no botão "Download ZIP" ao lado deste arquivo, como na imagem abaixo.  
![example1](http://i.imgur.com/kJtzWwf.jpg)

2. Para instalar o python no Windows:


3. Para instalar o python no Linux:
> OBS: As últimas versões do Ubuntu e Fedora já vêm com o python 2.7 e as últimas versões do RHEL e CentOS
       já vêm com o python 2.6.

[BAIXE O PYTHON SCRIPT AQUI, E EXECUTE](http://python-distribute.org/distribute\_setup.py). Para executar,
       use no terminal `python distributte\_setup.py`
  * Instale o pip:
    `$ easy\_install pip`
  * Instale o virtualenv:
    `$ pip install virtualenv`
  * Selecione o diretório para a instalação do python e execute o virtualenv.
    `$ virtualenv --distributte venv`
  * Para executar o ambiente, rode:
    `$ source venv/bin/activate`
  * Para sair do ambiente:
    `$ deactivate`

4. Para instalar django no Windows:
  * Baixe [Download Django-1.6.2.tar.gz](https://www.djangoproject.com/download/1.6.2/tarball/). Então
       extraia o arquivo, inicie o DOS
 shell com permissão de administrador e execute o comando no diretório cujo nome inicie com "Django-":
    `$ python setup.py install`

5. Para instalar django no Ubuntu:
  * Pelo pip:
    `$ pip install Django==1.6.2`
  * "Manualmente":
    * Baixe [Download Django-1.6.2.tar.gz](https://www.djangoproject.com/download/1.6.2/tarball/). Então:
        `$ tar xzvf Django-1.6.2.tar.gz`
        `$ cd Django-1.6.2`
        `$ sudo python setup.py install`� Online) (nome temporário)**

Orientador:
**Professor Cláudio Correa e Castro Gonçalves**

Unidade Acadêmica / Departamento:
**Instituto de Letras/IL - Departamento de Línguas Estrangeiras e Tradução/LET**

Alunos Envolvidos:
* Yurick Hauschild Caetano da Costa 12/0024136
* André Accioly Lima 12/0059908
* Diego Santos da Silva 11/0027892

### 3. Documentação

Toda a documentação do projeto pode ser encontrada na pasta doc/.

> **ATENÇÃO**:
> A documentação acima referenciada inclui explanações das funções de
> todas as classes e métodos implementados e tem por público alvo
> principalmente desenvolvedores interessados em contribuir com o projeto.

> Para uma versão específica para usuários, ou seja, pessoas que estão interessadas no funcionamento da
       ferramenta e não em seus mecanismos, leia o ponto *4. Instalação e execução*

#### 3.i. Doxyfile (html)

 No intuito de simplificar a navegação dentro da documentação do projeto, utilizamos a ferramenta Doxygen,
       gerando, assim, um arquivo HTML que contém uma interface amigável.
 Este arquivo está contido em doc/html/ e para acessá-lo basta abrir o arquivo doc/html/index.html com o
       seu navegador.
 
### 4. Instalação e execução

Para executar o programa, siga as instruções abaixo.

1. Baixe o código fonte do projeto.
  Para isso, basta clicar no botão "Download ZIP" ao lado deste arquivo, como na imagem abaixo.  
![example1](http://i.imgur.com/kJtzWwf.jpg)

2. Para instalar o python no Windows:


3. Para instalar o python no Linux:
> OBS: As últimas versões do Ubuntu e Fedora já vêm com o python 2.7 e as últimas versões do RHEL e CentOS
       já vêm com o python 2.6.

[BAIXE O PYTHON SCRIPT AQUI, E EXECUTE](http://python-distribute.org/distribute\_setup.py). Para executar,
       use no terminal `python distributte\_setup.py`
  * Instale o pip:
    `$ easy\_install pip`
  * Instale o virtualenv:
    `$ pip install virtualenv`
  * Selecione o diretório para a instalação do python e execute o virtualenv.
    `$ virtualenv --distributte venv`
  * Para executar o ambiente, rode:
    `$ source venv/bin/activate`
  * Para sair do ambiente:
    `$ deactivate`

4. Para instalar django no Windows:
  * Baixe [Download Django-1.6.2.tar.gz](https://www.djangoproject.com/download/1.6.2/tarball/). Então
       extraia o arquivo, inicie o DOS
 shell com permissão de administrador e execute o comando no diretório cujo nome inicie com "Django-":
    `$ python setup.py install`

5. Para instalar django no Ubuntu:
  * Pelo pip:
    `$ pip install Django==1.6.2`
  * "Manualmente":
    * Baixe [Download Django-1.6.2.tar.gz](https://www.djangoproject.com/download/1.6.2/tarball/). Então:
        `$ tar xzvf Django-1.6.2.tar.gz`
        `$ cd Django-1.6.2`
        `$ sudo python setup.py install`nguas Online) (nome tempor�

Orientador:
**Professor Cláudio Correa e Castro Gonçalves**

Unidade Acadêmica / Departamento:
**Instituto de Letras/IL - Departamento de Línguas Estrangeiras e Tradução/LET**

Alunos Envolvidos:
* Yurick Hauschild Caetano da Costa 12/0024136
* André Accioly Lima 12/0059908
* Diego Santos da Silva 11/0027892

### 3. Documentação

Toda a documentação do projeto pode ser encontrada na pasta doc/.

> **ATENÇÃO**:
> A documentação acima referenciada inclui explanações das funções de
> todas as classes e métodos implementados e tem por público alvo
> principalmente desenvolvedores interessados em contribuir com o projeto.

> Para uma versão específica para usuários, ou seja, pessoas que estão interessadas no funcionamento da
       ferramenta e não em seus mecanismos, leia o ponto *4. Instalação e execução*

#### 3.i. Doxyfile (html)

 No intuito de simplificar a navegação dentro da documentação do projeto, utilizamos a ferramenta Doxygen,
       gerando, assim, um arquivo HTML que contém uma interface amigável.
 Este arquivo está contido em doc/html/ e para acessá-lo basta abrir o arquivo doc/html/index.html com o
       seu navegador.
 
### 4. Instalação e execução

Para executar o programa, siga as instruções abaixo.

1. Baixe o código fonte do projeto.
  Para isso, basta clicar no botão "Download ZIP" ao lado deste arquivo, como na imagem abaixo.  
![example1](http://i.imgur.com/kJtzWwf.jpg)

2. Para instalar o python no Windows:


3. Para instalar o python no Linux:
> OBS: As últimas versões do Ubuntu e Fedora já vêm com o python 2.7 e as últimas versões do RHEL e CentOS
       já vêm com o python 2.6.

[BAIXE O PYTHON SCRIPT AQUI, E EXECUTE](http://python-distribute.org/distribute\_setup.py). Para executar,
       use no terminal `python distributte\_setup.py`
  * Instale o pip:
    `$ easy\_install pip`
  * Instale o virtualenv:
    `$ pip install virtualenv`
  * Selecione o diretório para a instalação do python e execute o virtualenv.
    `$ virtualenv --distributte venv`
  * Para executar o ambiente, rode:
    `$ source venv/bin/activate`
  * Para sair do ambiente:
    `$ deactivate`

4. Para instalar django no Windows:
  * Baixe [Download Django-1.6.2.tar.gz](https://www.djangoproject.com/download/1.6.2/tarball/). Então
       extraia o arquivo, inicie o DOS
 shell com permissão de administrador e execute o comando no diretório cujo nome inicie com "Django-":
    `$ python setup.py install`

5. Para instalar django no Ubuntu:
  * Pelo pip:
    `$ pip install Django==1.6.2`
  * "Manualmente":
    * Baixe [Download Django-1.6.2.tar.gz](https://www.djangoproject.com/download/1.6.2/tarball/). Então:
        `$ tar xzvf Django-1.6.2.tar.gz`
        `$ cd Django-1.6.2`
        `$ sudo python setup.py install`�

Orientador:
**Professor Cláudio Correa e Castro Gonçalves**

Unidade Acadêmica / Departamento:
**Instituto de Letras/IL - Departamento de Línguas Estrangeiras e Tradução/LET**

Alunos Envolvidos:
* Yurick Hauschild Caetano da Costa 12/0024136
* André Accioly Lima 12/0059908
* Diego Santos da Silva 11/0027892

### 3. Documentação

Toda a documentação do projeto pode ser encontrada na pasta doc/.

> **ATENÇÃO**:
> A documentação acima referenciada inclui explanações das funções de
> todas as classes e métodos implementados e tem por público alvo
> principalmente desenvolvedores interessados em contribuir com o projeto.

> Para uma versão específica para usuários, ou seja, pessoas que estão interessadas no funcionamento da
       ferramenta e não em seus mecanismos, leia o ponto *4. Instalação e execução*

#### 3.i. Doxyfile (html)

 No intuito de simplificar a navegação dentro da documentação do projeto, utilizamos a ferramenta Doxygen,
       gerando, assim, um arquivo HTML que contém uma interface amigável.
 Este arquivo está contido em doc/html/ e para acessá-lo basta abrir o arquivo doc/html/index.html com o
       seu navegador.
 
### 4. Instalação e execução

Para executar o programa, siga as instruções abaixo.

1. Baixe o código fonte do projeto.
  Para isso, basta clicar no botão "Download ZIP" ao lado deste arquivo, como na imagem abaixo.  
![example1](http://i.imgur.com/kJtzWwf.jpg)

2. Para instalar o python no Windows:


3. Para instalar o python no Linux:
> OBS: As últimas versões do Ubuntu e Fedora já vêm com o python 2.7 e as últimas versões do RHEL e CentOS
       já vêm com o python 2.6.

[BAIXE O PYTHON SCRIPT AQUI, E EXECUTE](http://python-distribute.org/distribute\_setup.py). Para executar,
       use no terminal `python distributte\_setup.py`
  * Instale o pip:
    `$ easy\_install pip`
  * Instale o virtualenv:
    `$ pip install virtualenv`
  * Selecione o diretório para a instalação do python e execute o virtualenv.
    `$ virtualenv --distributte venv`
  * Para executar o ambiente, rode:
    `$ source venv/bin/activate`
  * Para sair do ambiente:
    `$ deactivate`

4. Para instalar django no Windows:
  * Baixe [Download Django-1.6.2.tar.gz](https://www.djangoproject.com/download/1.6.2/tarball/). Então
       extraia o arquivo, inicie o DOS
 shell com permissão de administrador e execute o comando no diretório cujo nome inicie com "Django-":
    `$ python setup.py install`

5. Para instalar django no Ubuntu:
  * Pelo pip:
    `$ pip install Django==1.6.2`
  * "Manualmente":
    * Baixe [Download Django-1.6.2.tar.gz](https://www.djangoproject.com/download/1.6.2/tarball/). Então:
        `$ tar xzvf Django-1.6.2.tar.gz`
        `$ cd Django-1.6.2`
        `$ sudo python setup.py install`rio)**
00029 
00030 Orientador:
00031 **Professor Cláudio Correa e Castro Gonçalves**
00032 
00033 Unidade Acadêmica / Departamento:
00034 **Instituto de Letras/IL - Departamento de Línguas Estrangeiras e Tradução/LET**
00035 
00036 Alunos Envolvidos:
00037 * Yurick Hauschild Caetano da Costa 12/0024136
00038 * André Accioly Lima 12/0059908
00039 * Diego Santos da Silva 11/0027892
00040 
00041 \textcolor{preprocessor}{### 3. Documentação}
00042 \textcolor{preprocessor}{}
00043 Toda a documentação do projeto pode ser encontrada na pasta doc/.
00044 
00045 > **ATENÇÃO**:
00046 > A documentação acima referenciada inclui explanações das funções de
00047 > todas as classes e métodos implementados e tem por público alvo
00048 > principalmente desenvolvedores interessados em contribuir com o projeto.
00049 
00050 > Para uma versão específica para usuários, ou seja, pessoas que estão interessadas no funcionamento da 
      ferramenta e não em seus mecanismos, leia o ponto *4. Instalação e execução*
00051 
00052 \textcolor{preprocessor}{#### 3.i. Doxyfile (html)}
00053 \textcolor{preprocessor}{}
00054  No intuito de simplificar a navegação dentro da documentação do projeto, utilizamos a ferramenta Doxygen, 
      gerando, assim, um arquivo HTML que contém uma interface amigável.
00055  Este arquivo está contido em doc/html/ e para acessá-lo basta abrir o arquivo doc/html/index.html com o 
      seu navegador.
00056  
00057 \textcolor{preprocessor}{### 4. Instalação e execução}
00058 \textcolor{preprocessor}{}
00059 Para executar o programa, siga as instruções abaixo.
00060 
00061 1. Baixe o código fonte do projeto.
00062   Para isso, basta clicar no botão "Download ZIP" ao lado deste arquivo, como na imagem abaixo.  
00063 ![example1](http:\textcolor{comment}{//i.imgur.com/kJtzWwf.jpg)}
00064 
00065 2. Para instalar o python no Windows:
00066 
00067 
00068 3. Para instalar o python no Linux:
00069 > OBS: As últimas versões do Ubuntu e Fedora já vêm com o python 2.7 e as últimas versões do RHEL e CentOS 
      já vêm com o python 2.6.
00070 
00071 [BAIXE O PYTHON SCRIPT AQUI, E EXECUTE](http:\textcolor{comment}{//python-distribute.org/distribute\_setup.py). Para executar,
       use no terminal `python distributte\_setup.py`}
00072   * Instale o pip:
00073     `$ easy\_install pip`
00074   * Instale o virtualenv:
00075     `$ pip install virtualenv`
00076   * Selecione o diret� para a instalação do python e execute o virtualenv.
    `$ virtualenv --distributte venv`
  * Para executar o ambiente, rode:
    `$ source venv/bin/activate`
  * Para sair do ambiente:
    `$ deactivate`

4. Para instalar django no Windows:
  * Baixe [Download Django-1.6.2.tar.gz](https://www.djangoproject.com/download/1.6.2/tarball/). Então
       extraia o arquivo, inicie o DOS
 shell com permissão de administrador e execute o comando no diretório cujo nome inicie com "Django-":
    `$ python setup.py install`

5. Para instalar django no Ubuntu:
  * Pelo pip:
    `$ pip install Django==1.6.2`
  * "Manualmente":
    * Baixe [Download Django-1.6.2.tar.gz](https://www.djangoproject.com/download/1.6.2/tarball/). Então:
        `$ tar xzvf Django-1.6.2.tar.gz`
        `$ cd Django-1.6.2`
        `$ sudo python setup.py install`� para a instalação do python e execute o virtualenv.
    `$ virtualenv --distributte venv`
  * Para executar o ambiente, rode:
    `$ source venv/bin/activate`
  * Para sair do ambiente:
    `$ deactivate`

4. Para instalar django no Windows:
  * Baixe [Download Django-1.6.2.tar.gz](https://www.djangoproject.com/download/1.6.2/tarball/). Então
       extraia o arquivo, inicie o DOS
 shell com permissão de administrador e execute o comando no diretório cujo nome inicie com "Django-":
    `$ python setup.py install`

5. Para instalar django no Ubuntu:
  * Pelo pip:
    `$ pip install Django==1.6.2`
  * "Manualmente":
    * Baixe [Download Django-1.6.2.tar.gz](https://www.djangoproject.com/download/1.6.2/tarball/). Então:
        `$ tar xzvf Django-1.6.2.tar.gz`
        `$ cd Django-1.6.2`
        `$ sudo python setup.py install`rio para a instala� do python e execute o virtualenv.
    `$ virtualenv --distributte venv`
  * Para executar o ambiente, rode:
    `$ source venv/bin/activate`
  * Para sair do ambiente:
    `$ deactivate`

4. Para instalar django no Windows:
  * Baixe [Download Django-1.6.2.tar.gz](https://www.djangoproject.com/download/1.6.2/tarball/). Então
       extraia o arquivo, inicie o DOS
 shell com permissão de administrador e execute o comando no diretório cujo nome inicie com "Django-":
    `$ python setup.py install`

5. Para instalar django no Ubuntu:
  * Pelo pip:
    `$ pip install Django==1.6.2`
  * "Manualmente":
    * Baixe [Download Django-1.6.2.tar.gz](https://www.djangoproject.com/download/1.6.2/tarball/). Então:
        `$ tar xzvf Django-1.6.2.tar.gz`
        `$ cd Django-1.6.2`
        `$ sudo python setup.py install`� do python e execute o virtualenv.
    `$ virtualenv --distributte venv`
  * Para executar o ambiente, rode:
    `$ source venv/bin/activate`
  * Para sair do ambiente:
    `$ deactivate`

4. Para instalar django no Windows:
  * Baixe [Download Django-1.6.2.tar.gz](https://www.djangoproject.com/download/1.6.2/tarball/). Então
       extraia o arquivo, inicie o DOS
 shell com permissão de administrador e execute o comando no diretório cujo nome inicie com "Django-":
    `$ python setup.py install`

5. Para instalar django no Ubuntu:
  * Pelo pip:
    `$ pip install Django==1.6.2`
  * "Manualmente":
    * Baixe [Download Django-1.6.2.tar.gz](https://www.djangoproject.com/download/1.6.2/tarball/). Então:
        `$ tar xzvf Django-1.6.2.tar.gz`
        `$ cd Django-1.6.2`
        `$ sudo python setup.py install`� do python e execute o virtualenv.
    `$ virtualenv --distributte venv`
  * Para executar o ambiente, rode:
    `$ source venv/bin/activate`
  * Para sair do ambiente:
    `$ deactivate`

4. Para instalar django no Windows:
  * Baixe [Download Django-1.6.2.tar.gz](https://www.djangoproject.com/download/1.6.2/tarball/). Então
       extraia o arquivo, inicie o DOS
 shell com permissão de administrador e execute o comando no diretório cujo nome inicie com "Django-":
    `$ python setup.py install`

5. Para instalar django no Ubuntu:
  * Pelo pip:
    `$ pip install Django==1.6.2`
  * "Manualmente":
    * Baixe [Download Django-1.6.2.tar.gz](https://www.djangoproject.com/download/1.6.2/tarball/). Então:
        `$ tar xzvf Django-1.6.2.tar.gz`
        `$ cd Django-1.6.2`
        `$ sudo python setup.py install`� do python e execute o virtualenv.
    `$ virtualenv --distributte venv`
  * Para executar o ambiente, rode:
    `$ source venv/bin/activate`
  * Para sair do ambiente:
    `$ deactivate`

4. Para instalar django no Windows:
  * Baixe [Download Django-1.6.2.tar.gz](https://www.djangoproject.com/download/1.6.2/tarball/). Então
       extraia o arquivo, inicie o DOS
 shell com permissão de administrador e execute o comando no diretório cujo nome inicie com "Django-":
    `$ python setup.py install`

5. Para instalar django no Ubuntu:
  * Pelo pip:
    `$ pip install Django==1.6.2`
  * "Manualmente":
    * Baixe [Download Django-1.6.2.tar.gz](https://www.djangoproject.com/download/1.6.2/tarball/). Então:
        `$ tar xzvf Django-1.6.2.tar.gz`
        `$ cd Django-1.6.2`
        `$ sudo python setup.py install`o do python e execute o virtualenv.
00077     `$ virtualenv --distributte venv`
00078   * Para executar o ambiente, rode:
00079     `$ source venv/bin/activate`
00080   * Para sair do ambiente:
00081     `$ deactivate`
00082 
00083 4. Para instalar django no Windows:
00084   * Baixe [Download Django-1.6.2.tar.gz](https:\textcolor{comment}{//www.djangoproject.com/download/1.6.2/tarball/). Então
       extraia o arquivo, inicie o DOS}
00085  shell com permiss� de administrador e execute o comando no diretório cujo nome inicie com "Django-":
    `$ python setup.py install`

5. Para instalar django no Ubuntu:
  * Pelo pip:
    `$ pip install Django==1.6.2`
  * "Manualmente":
    * Baixe [Download Django-1.6.2.tar.gz](https://www.djangoproject.com/download/1.6.2/tarball/). Então:
        `$ tar xzvf Django-1.6.2.tar.gz`
        `$ cd Django-1.6.2`
        `$ sudo python setup.py install`� de administrador e execute o comando no diretório cujo nome
       inicie com "Django-":
    `$ python setup.py install`

5. Para instalar django no Ubuntu:
  * Pelo pip:
    `$ pip install Django==1.6.2`
  * "Manualmente":
    * Baixe [Download Django-1.6.2.tar.gz](https://www.djangoproject.com/download/1.6.2/tarball/). Então:
        `$ tar xzvf Django-1.6.2.tar.gz`
        `$ cd Django-1.6.2`
        `$ sudo python setup.py install`o de administrador e execute o comando no diret� cujo nome
       inicie com "Django-":
    `$ python setup.py install`

5. Para instalar django no Ubuntu:
  * Pelo pip:
    `$ pip install Django==1.6.2`
  * "Manualmente":
    * Baixe [Download Django-1.6.2.tar.gz](https://www.djangoproject.com/download/1.6.2/tarball/). Então:
        `$ tar xzvf Django-1.6.2.tar.gz`
        `$ cd Django-1.6.2`
        `$ sudo python setup.py install`� cujo nome inicie com "Django-":
    `$ python setup.py install`

5. Para instalar django no Ubuntu:
  * Pelo pip:
    `$ pip install Django==1.6.2`
  * "Manualmente":
    * Baixe [Download Django-1.6.2.tar.gz](https://www.djangoproject.com/download/1.6.2/tarball/). Então:
        `$ tar xzvf Django-1.6.2.tar.gz`
        `$ cd Django-1.6.2`
        `$ sudo python setup.py install`rio cujo nome inicie com "Django-":
00086     `$ python setup.py install`
00087 
00088 5. Para instalar django no Ubuntu:
00089   * Pelo pip:
00090     `$ pip install Django==1.6.2`
00091   * "Manualmente":
00092     * Baixe [Download Django-1.6.2.tar.gz](https:\textcolor{comment}{//www.djangoproject.com/download/1.6.2/tarball/). Então:}
00093         `$ tar xzvf Django-1.6.2.tar.gz`
00094         `$ cd Django-1.6.2`
00095         `$ sudo python setup.py install`
\end{DoxyCode}
