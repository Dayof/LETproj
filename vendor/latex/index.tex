\subsubsection*{Index}


\begin{DoxyEnumerate}
\item Definição
\item Dados do projeto
\item Documentação 3.\-1. Doxygen (html)
\item Instalação e execução
\item Metas de desenvolvimento
\item Log de Atividades
\end{DoxyEnumerate}

\subsubsection*{Conteúdo}

\paragraph*{1. Definição}

O L\-E\-Tproject é uma iniciativa do laboratório do L\-E\-T para o desenvolvimento de uma plataforma de ensino de línguas online. Seu principal objetivo é criar uma interface amigável entre alunos e professores de forma a incluir o estudo doméstico de línguas no mundo informatizado.

\paragraph*{2. Dados do projeto}

Título do projeto\-:
\begin{DoxyItemize}
\item {\bfseries L\-E\-Tproject (nome temporário)}
\item {\bfseries \hyperlink{namespaceELO}{E\-L\-O} (Ensino de Línguas Online) (nome temporário)}
\item {\bfseries S\-A\-Li\-E (nome final)}
\end{DoxyItemize}

Orientador\-: {\bfseries Professor Cláudio Correa e Castro Gonçalves}

Unidade Acadêmica / Departamento\-: {\bfseries Instituto de Letras/\-I\-L -\/ Departamento de Línguas Estrangeiras e Tradução/\-L\-E\-T}

Alunos Envolvidos\-:
\begin{DoxyItemize}
\item Yurick Hauschild Caetano da Costa 12/0024136
\item Dayanne Fernandes 13/0107191
\item Bruna Luisa xx/xxxxxxx
\end{DoxyItemize}

\paragraph*{3. Documentação}

Toda a documentação do projeto pode ser encontrada na pasta doc/.

\begin{quotation}
{\bfseries A\-T\-E\-NÇÃ\-O}\-: A documentação acima referenciada inclui explanações das funções de todas as classes e métodos implementados e tem por público alvo principalmente desenvolvedores interessados em contribuir com o projeto.

\end{quotation}


\begin{quotation}
Para uma versão específica para usuários, ou seja, pessoas que estão interessadas no funcionamento da ferramenta e não em seus mecanismos, leia o ponto {\itshape 4. Instalação e execução}

\end{quotation}


\subparagraph*{3.\-i. Doxyfile (html)}

No intuito de simplificar a navegação dentro da documentação do projeto, utilizamos a ferramenta Doxygen, gerando, assim, um arquivo H\-T\-M\-L que contém uma interface amigável, bem como um arquivo .tex capaz de gerar um pdf. Tanto o pdf quanto o html possuem as mesmas informações. Estes arquivos estão contidos em vendor e para acessá-\/los basta abrir o arquivo doc/html/index.\-html ou doc/latex/refman.\-pdf com o seu navegador ou visualizador de pdf, respectivamente.

\paragraph*{4. Instalação e execução}

Para executar o programa, siga as instruções abaixo.


\begin{DoxyEnumerate}
\item Baixe o código fonte do projeto. Para isso, basta clicar no botão \char`\"{}\-Download Z\-I\-P\char`\"{} ao lado deste arquivo, como na imagem abaixo. 
\item Para instalar o python no Windows\-:

Obs\-: Não desenvolvemos a plataforma em Windows, e não temos o hábito de testar neste ambiente. Então, caro usuário da microsoft, na ocasião de encontrar algum bug ocasionado por incompatibilidade, sintá-\/se convidado a abrir uma issue e nos avisar, para que o corrijamos o mais rápido possível. \href{http://docs.python-guide.org/en/latest/starting/install/win/}{\tt Siga este tutorial}.
\item Para instalar o python no Linux\-: \begin{quotation}
O\-B\-S\-: As últimas versões do Ubuntu e Fedora já vêm com o python 2.\-7 e as últimas versões do R\-H\-E\-L e Cent\-O\-S já vêm com o python 2.\-6.

\end{quotation}

\begin{DoxyItemize}
\item Instale o pip\-: {\ttfamily \$ easy\-\_\-install pip}
\item Instale o virtualenv\-: {\ttfamily \$ pip install virtualenv}
\item Selecione o diretório para a instalação do python e execute o virtualenv. {\ttfamily \$ virtualenv -\/-\/distributte venv}
\item Para executar o ambiente, rode\-: {\ttfamily \$ source venv/bin/activate}
\item Para sair do ambiente\-: {\ttfamily \$ deactivate}
\end{DoxyItemize}
\item Para instalar django no Windows\-:
\begin{DoxyItemize}
\item Baixe \href{https://www.djangoproject.com/download/1.6.2/tarball/}{\tt Download Django-\/1.\-6.\-2.\-tar.\-gz}. Então extraia o arquivo, inicie o D\-O\-S shell com permissão de administrador e execute o comando no diretório cujo nome inicie com \char`\"{}\-Django-\/\char`\"{}\-: {\ttfamily \$ python setup.\-py install}
\end{DoxyItemize}
\item Para instalar django no Ubuntu\-:
\begin{DoxyItemize}
\item Pelo pip\-: {\ttfamily \$ pip install Django==1.\-7}
\item \char`\"{}\-Manualmente\char`\"{}\-:
\begin{DoxyItemize}
\item Baixe \href{https://www.djangoproject.com/download/1.6.2/tarball/}{\tt Download Django-\/1.\-6.\-2.\-tar.\-gz}. Então\-: {\ttfamily \$ tar xzvf Django-\/1.\-6.\-2.\-tar.\-gz} {\ttfamily \$ cd Django-\/1.\-6.\-2} {\ttfamily \$ sudo python setup.\-py install} 
\end{DoxyItemize}
\end{DoxyItemize}
\end{DoxyEnumerate}