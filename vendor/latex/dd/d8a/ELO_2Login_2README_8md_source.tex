\hypertarget{ELO_2Login_2README_8md_source}{\subsection{E\-L\-O/\-Login/\-R\-E\-A\-D\-M\-E.md}
}

\begin{DoxyCode}
00001 \textcolor{preprocessor}{# LETProject.ELO.Login}
00002 \textcolor{preprocessor}{}
00003 \textcolor{preprocessor}{## Função}
00004 \textcolor{preprocessor}{}
00005 Pasta que contém o módulo de login.
00006 Responsável pelo controle de login e logout dos usuários \textcolor{keywordflow}{do} sistema. É capaz de validar nomes e senhas.
00007 
00008 ## Conteúdo
00009 
00010 As entradas em negrito indicam pastas, e as normais indicam arquivos isolados.
00011 
00012 * \_\_init\_\_.py
00013 * forms.py
00014 * LoginUnit.py
00015 
00016 \textcolor{preprocessor}{## \_\_init\_\_.py}
00017 \textcolor{preprocessor}{}
00018 Arquivo requisitado pelo python para reconhecer a pasta como um repositório de arquivos referenciável.
00019 Em outras palavras, é um arquivo que só é útil para a linguagem e deve ser ignorado em outras instâncias.
00020 
00021 ## forms.py
00022 
00023 Arquivo que contém a definição das classes que modelam os formulários deste módulo.
00024 Neste caso, ele é responsável por modelar a form de login, ou seja, definir as caixas de entrada de nome e 
      senha, sem as quais seria impossível realizar o login.
00025 
00026 ## LoginUnit.py
00027 
00028 Arquivo principal \textcolor{keywordflow}{do} módulo.
00029 Contém as diferentes camadas, bem como suas devidas interfaces e variadas implementações.
00030 É o arquivo que recebe o controle \textcolor{keywordflow}{do} programa após a chamada da Factory (ver MainUnit.py dentro de 
      LETProject.ELO.ELO). 
00031 É responsável por se utilizar das classes formulário (ver forms.py) bem como dos outros recursos do 
      programa - como o banco de dados e os cookies - para criar a página que será mostrada para o usuário.
00032 No caso da LoginUnit, ela deve ser responsável por barrar o acesso de usuários não registrados, ou que não 
      consigam inserir a senha.
\end{DoxyCode}
