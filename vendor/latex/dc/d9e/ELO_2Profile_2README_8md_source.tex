\hypertarget{ELO_2Profile_2README_8md_source}{}\subsection{E\+L\+O/\+Profile/\+R\+E\+A\+D\+M\+E.md}

\begin{DoxyCode}
00001 # LETProject.ELO.Profile
00002 
00003 ## Função
00004 
00005 Pasta que contém o módulo de perfil.
00006 Responsável pelas páginas onde se encontram os dados do usuário, seja ela a Home ou a página de dados
       detalhados (referenciada internamente como 'perfil completo'). Também permite ao usuário editar seus
       próprios dados. A uma primeira versão, é inacessível para administradores.
00007 
00008 ## Conteúdo
00009 
00010 As entradas em negrito indicam pastas, e as normais indicam arquivos isolados.
00011 
00012 * \_\_init\_\_.py
00013 * forms.py
00014 * ProfileUnit.py
00015 
00016 ## \_\_init\_\_.py
00017 
00018 Arquivo requisitado pelo python para reconhecer a pasta como um repositório de arquivos referenciável.
00019 Em outras palavras, é um arquivo que só é útil para a linguagem e deve ser ignorado em outras
       instâncias.
00020 
00021 ## forms.py
00022 
00023 Arquivo que contém a definição das classes que modelam os formulários deste módulo.
00024 Neste caso, ele é responsável por modelar as forms de edição de dados de usuário, possbilitando ao
       usuário comum (estudante ou professor) editar seus dados pessoais, como nome e sexo.
00025 
00026 ## LoginUnit.py
00027 
00028 Arquivo principal do módulo.
00029 Contém as diferentes camadas, bem como suas devidas interfaces e variadas implementações.
00030 É o arquivo que recebe o controle do programa após a chamada da Factory (ver MainUnit.py dentro de
       LETProject.ELO.ELO). 
00031 É responsável por se utilizar das classes formulário (ver forms.py) bem como dos outros recursos do
       programa - como o banco de dados e os cookies - para criar a página que será mostrada para o usuário.
00032 No caso da ProfileUnit, ele será responsável por possibilitar a visualização e edição dos dados
       pessoais - sensíveis ou não - ao usuário comum.
\end{DoxyCode}
