\hypertarget{ELO_2ELO_2README_8md_source}{\subsection{E\-L\-O/\-E\-L\-O/\-R\-E\-A\-D\-M\-E.md}
}

\begin{DoxyCode}
00001 \textcolor{preprocessor}{# LETProject.ELO.ELO}
00002 \textcolor{preprocessor}{}
00003 \textcolor{preprocessor}{## Função}
00004 \textcolor{preprocessor}{}
00005 Pasta que contém os arquivos que são comuns a todos os módulos, assim como as pastas de \textcolor{keyword}{template}, \textcolor{keyword}{static} e 
      media.
00006 São arquivos responsáveis, em geral, pelo controle \textcolor{keywordflow}{do} fluxo e a organização intermodular.
00007 
00008 ## Conteúdo
00009 
00010 Entradas em negrito indicam pastas, entradas normais indicam arquivos isolados.
00011 
00012 * \_\_init\_\_.py
00013 * BaseUnit.py
00014 * **database**
00015 * EntityUnit.py
00016 * **lang**
00017 * MainUnit.py
00018 * **media**
00019 * models.py
00020 * settings.py
00021 * **\textcolor{keyword}{static}**
00022 * **templates**
00023 * **Trash**
00024 * urls.py
00025 * wsgi.py
00026 
00027 \textcolor{preprocessor}{### \_\_init\_\_.py}
00028 \textcolor{preprocessor}{}
00029 Arquivo requisitado pelo python para reconhecer a pasta como um repositório de arquivos referenciável.
00030 Em outras palavras, é um arquivo que só é útil para a linguagem e deve ser ignorado em outras instâncias.
00031 
00032 ### BaseUnit.py
00033 
00034 Arquivo que contém os tipos básicos \textcolor{keywordflow}{do} sistema.
00035 Um tipo básico é a unidade mínima de dado \textcolor{keywordflow}{do} programa (e.g. Nome, Senha, Matrícula e afins). 
00036 É o tipo básico quem executa a validação dos dados recebidos e/ou enviados; isso quer dizer que só é 
      possível criar um Nome, se ele atender aos requisitos dados para tal, ou seja, ser uma sequência de caracteres 
      alfanuméricos de até 32 caracteres. Caso tais requisitos não sejam atendidos, o tipo básico não será criado e 
      uma exceção \textcolor{keywordflow}{do} tipo ValueError será lançada.
00037 
00038 ### database
00039 
00040 Pasta responsável pelo armazenamento \textcolor{keywordflow}{do} banco de dados.
00041 Contém o arquivo binário correspondente ao banco de dados e um arquivo SQL que corresponde ao DUMP da 
      última versão daquele. Vale salientar que o GIT somente irá armazenar e atualizar este último.
00042 
00043 ### EntityUnit.py
00044 
00045 Arquivo que contém as entidades \textcolor{keywordflow}{do} sistema.
00046 Uma entidade é uma classe que só possui tipos básicos por atributos. Ela é responsável por armazenar a 
      validar os dados de todas as entidades \textcolor{keywordflow}{do} sistema (e.g. Aluno, Professor, Administrador, Curso etc).
00047 Analogamente a um Tipo Básico, só é possível criar uma entidade no caso de todos os seus atributos serem 
      válidos. Esse tipo de assertiva garante que somente sobreviverão dentro \textcolor{keywordflow}{do} sistema entidades com todos os 
      tipos básicos válidos.
00048 
00049 ### lang
00050 
00051 Pasta que contém os arquivos-dicionário e o arquivo de configuração de linguagem.
00052 Esta pasta é a responsável por converter todo o texto \textcolor{keywordflow}{do} programa para diferentes línguas.
00053 
00054 ### MainUnit.py
00055 
00056 Arquivo de montagem dos módulos \textcolor{keywordflow}{do} programa.
00057 Contém procedimentos de inicialização \textcolor{keywordflow}{do} programa e a classe Factory.
00058 Esta classe tem a responsabilidade de administrar o fluxo intermodular, ou seja, é ela quem monta e 
      desmonta as camadas de um mesmo módulo de forma a viabilizar o tráfego entre eles.
00059 
00060 ### media
00061 
00062 Pasta que contém os arquivos de media \textcolor{keywordflow}{do} sistema.
00063 É ela quem contém os dados de upload dos usuários, como os avatares dos alunos ou os vídeos dos professores
      .
00064 
00065 ### models.py
00066 
00067 Arquivo que contém as definições dos modelos \textcolor{keywordflow}{do} programa.
00068 Este arquivo é o responsável por modelar as tabelas \textcolor{keywordflow}{do} banco de dados de forma a viabilizar as consultas 
      das camadas de persistência.
00069 Note que aqui utilizamos a filosofia de triplestore em detrimento da organização relacional.
00070 
00071 ### settings.py
00072 
00073 Arquivo de configuração \textcolor{keywordflow}{do} Django.
00074 Será a primeira coisa a desaparecer após a publicação \textcolor{keywordflow}{do} sistema, por conter alguns dados de segurança.
00075 
00076 ### \textcolor{keyword}{static}
00077 
00078 Pasta que contém os arquivos estáticos \textcolor{keywordflow}{do} sistema.
00079 Arquivos CSS e Javascript são armazenados aqui para simplificar a busca por eles. Diferentemente dos 
      templates, os arquivos estáticos nem sempre são carregados, sendo guardados preferencialmente no cache \textcolor{keywordflow}{do} 
      navegador.
00080 
00081 ### templates
00082 
00083 Pasta que contém os templates \textcolor{keywordflow}{do} sistema.
00084 Os templates são os arquivos HTML que serão manipulados pelos módulos e corresponderão as páginas vistas 
      pelo usuário \textcolor{keywordflow}{do} sistema.
00085 
00086 ### urls.py
00087 
00088 Arquivo conhecido como URLConf. Ele é o guia da Factory (ver MainUnit.py), apontando-a para a direção 
      correta, baseando-se na URL que lhe foi passada.
00089 É o primeiro arquivo a ser chamado, sendo o primeiro divisor de águas no fluxo \textcolor{keywordflow}{do} programa.
00090 
00091 ### wsgi.py
00092 
00093 Arquivo \textcolor{keywordflow}{do} Django que configura o Web-Server. Não deve ser modificado por hora.
\end{DoxyCode}
